\chapter*{Introduction}
The book is divided into the following sections:
\begin{enumerate}
\item \textbf{preliminary topics} such as basic logic and set theory, 
\item \textbf{abstract algebra} which follows \cite{dummit-foote}, 
\item \textbf{linear algebra} which follows \cite{hoffman-kunze}, 
\item \textbf{real analysis} which follows \cite{rudin,apostol}, and
\item \textbf{complex analysis} which follows \cite{ahlfors},
\item \textbf{topology} which follows \cite{munkres}.
\end{enumerate}

The chapters in this book are structured as follows:
\begin{itemize}
\item A \textbf{theoretical portion}, which starts off with a couple of definitions coupled with examples, followed by theorems and propositions built upon the definitions.
\item A series of \textbf{exercises}.
\item Full \textbf{solutions} to the exercises.
\end{itemize}

The reader is not assumed to have any mathematical prerequisites, althrough some experience with proofs may be helpful.

\section*{Problem Solving}
In \cite{polya}, George P\'{o}lya outlined the following problem solving cycle:
\begin{enumerate}
\item \textbf{Understand the problem}

Ask yourself the following questions:
\begin{itemize}
\item Do you understand all the words used in stating the problem?
\item Is it possible to satisfy the condition? Is the condition sufficient to determine the unknown? Or is it insufficient? Or redundant? Or contradictory?
\item What are you asked to find or show? Can you restate the problem in your own words?
\item Draw a figure. Introduce suitable notation.
\item Is there enough information to enable you to find a solution?
\end{itemize}

\item \textbf{Devise a plan}

A partial list of heuristics -- good rules of thumb to solve problems -- is included:
\begin{multicols}{2}
\begin{itemize}
\item Guess and check
\item Look for a pattern
\item Make an orderly list
\item Draw a picture
\item Eliminate possibilities
\item Solve a simpler problem
\item Use symmetry
\item Use a model
\item Consider special cases
\item Work backwards
\item Use direct reasoning
\item Use a formula
\item Solve an equation
\item Be ingenious
\end{itemize}
\end{multicols}

\item \textbf{Execute the plan}

This step is usually easier than devising the plan. In general, all you need is care and patience, given that you have the necessary skills. Persist with the plan that you have chosen. If it continues not to work discard it and choose another. Don't be misled, this is how mathematics is done, even by professionals.

\begin{itemize}
\item Carrying out your plan of the solution, check each step. Can you see clearly that the step is correct? Can you prove that it is correct?
\end{itemize}

\item \textbf{Check and expand}

P\'{o}lya mentions that much can be gained by taking the time to reflect and look back at what you have done, what worked, and what didn't. Doing this will enable you to predict what strategy to use to solve future problems.

Look back reviewing and checking your results. Ask yourself the following questions:
\begin{itemize}
\item Can you check the result? Can you check the argument?
\item Can you derive the solution differently? Can you see it at a glance?
\item Can you use the result, or the method, for some other problem?
\end{itemize}
\end{enumerate}

Building on P\'{o}lya's problem solving strategy, Schoenfeld \cite{schoenfeld} came up with the following framework for problem solving, consisting of four components:
\begin{enumerate}
\item \textbf{Cognitive resources}: the body of facts and procedures at one's disposal.
\item \textbf{Heuristics}: `rules of thumb' for making progress in difficult situations.
\item \textbf{Control}: having to do with the efficiency with which individuals utilise the knowledge at their disposal. Sometimes, this is referred to as metacognition, which can be roughly translated as `thinking about one's own thinking'.
\begin{enumerate}
\item These are questions to ask oneself to monitor one's thinking.
\begin{itemize}
    \item What (exactly) am I doing? [Describe it precisely.] Be clear what I am doing NOW. Why am I doing it? [Tell how it fits into the solution.]
    \item Be clear what I am doing in the context of the BIG picture -- the solution. Be clear what I am going to do NEXT.
\end{itemize}

\item Stop and reassess your options when you
\begin{itemize}
    \item cannot answer the questions satisfactorily [probably you are on the wrong track]; OR
    \item are stuck in what you are doing [the track may not be right or it is right but it is at that moment too difficult for you].
\end{itemize}

\item Decide if you want to
\begin{itemize}
    \item carry on with the plan,
    \item abandon the plan, OR
    \item put on hold and try another plan.
\end{itemize}
\end{enumerate}

\item \textbf{Belief system}: one's perspectives regarding the nature of a discipline and how one goes about working on it.
\end{enumerate}

\section*{Study Skills}
The Faculty of Mathematics of the University of Cambridge has produced a leaflet called ``\href{https://www.maths.cam.ac.uk/undergrad/files/studyskills.pdf}{Study Skills in Mathematics}''. The Faculty also has \href{https://www.maths.cam.ac.uk/undergrad/files/Advice%20on%20Preparing%20for%20Exams%202022.pdf}{guidance notes} intended to help students prepare for exams.

Similarly, the Mathematical Institute of the University of Oxford has a \href{https://www.maths.ox.ac.uk/system/files/attachments/study_public_0.pdf}{study guide} and \href{https://www.maths.ox.ac.uk/system/files/attachments/Revision_advice_final_0.pdf}{thoughts on preparing for exams}.