\part{Topology}
\chapter{$n$-dimensional Euclidean space}
\section{Topology in Euclidean Space}
\begin{itemize}
\item Basic constructions
\item 
The \textbf{interior} of a set $A$, denoted by $A^\circ$, is the set of all interior points in $A$.

A set $A\subset\RR^n$ is called an \textbf{open set} if $A^\circ=A$, i.e. all points in $A$ are interior points.

\item Limit points, closure and closed sets

An element $x\in A$ is called a limit point of $A$ if $B_0(x,\epsilon)\cap A\neq\emptyset$ for all $\epsilon>0$.

The induced set of a set, denoted by $A^\prime$, is the set of all limit points of $A$.

The closure of a set $A$, denoted by $\bar{A}$, is the union set $A\cup A^\prime$.

A set $A\subset RR^n$ is called a closed set if $\bar{A}=A$, i.e. all limit points of $A$ are contained in $A$

\item Further topological constructions of points

An element $x\in A$ is called an isolated point of $A$ if it is not a limit point of $A$.

The boundary of a set $A$, denoted by $\partial A$, is the set difference $\bar{A}\setminus A^\circ$.

An element $x\in\RR^n$ is called a boundary point of $A$ if it is in $\partial A$.

An element $x\in\RR^n$ is called an exterior point of $A$ if it is an interior point of $A^c$.

\item Further topological constructions of sets

A set $A\subset\RR^n$ is compact if it is a bounded closed set.

A subset $B\subset A$ is called a dense subset of $A$ if $\bar{B}=A$.

A set $A\subset\RR^n$ is nowhere dense its closure has no interior, i.e. $(\bar{A})^\circ=\emptyset$.
\end{itemize}

In general topology, these are the axioms used to define open and closed sets. At the moment we only consider them to be certain properties regarding open and closed sets in $\RR^n$.
\begin{enumerate}[label=\textbf{P\arabic*}]
\item $A$ is open if and only if $A^c$ is closed.

\begin{proof}
\textbf{Forward direction}:
Let $A$ be open, we consider the punctured balls of $x \notin A$ (if $x \notin A$, we consider the punctured balls centered at $x$).

Our goal is to show that $B_0(x,r)$ always intersects with $A^c$

So suppose otherwise that $B_0(x,\epsilon)$ is a subset of A for some $\epsilon>0$

Ah no sorry, we consider x not in $A^c$

The thing is we want to show that $A^c$ is closed, i.e. all limit points of $A^c$ are in $A^c$

So suppose otherwise that $x$ is a limit point of $A^c$ that is not in $A^c$

$x$ is a limit point of $A^c$, hence for all $\epsilon>0$, $B_0(x,\epsilon)$ always intersects with $A^c$

This is equivalent to saying that $B_0(x,\epsilon)$ is never a subset of $(A^c)^c=A$

However, $x$ is not in $x \notin A^c$, so $x \in A$.

But if A is open, then there exists $\epsilon>0$ such that $B(x,\epsilon)$ is a subset of $A$, a contradiction

\textbf{Backward direction}: Let $A^c$ be closed. Suppose otherwise that $A$ is not open, i.e. there is a point $x\in A$ such that $B(x,\epsilon)$ is never a subset of $A$; that is to say, $B(x,\epsilon)$ always intersects with $A^c$

Since $x \in A$, then $B(x,\epsilon) \cap A^c = B_0(x,\epsilon) \cap A^c$

But this means that $B_0(x,\epsilon) \cap A^c$ is never empty, hence $x$ is a limit point of $A^c$.

However, $x \in A$, contradictory to $A^c$ being closed and thus should contain all of its limit points
\end{proof}

\item An arbitrary union of open sets is open; a finite intersection of open sets is open.

\begin{proof}
Let $A$ be an arbitrary union of open sets $\{U_i\}_{i \in I}$.

Then for any $x \in A$, suppose that $x \in U_i$, then since $U_i$ is open we can pick $B(x,\epsilon)$ subset of $U_i$ subset of $A$

On the other hand, let $U$ and $V$ be open sets and let $x \in U \cap V$. 
Since $U$ and $V$ are open, we can pick $\epsilon_1$ and $\epsilon_2$ such that $B(x,\epsilon_1)$ is in $U$ whereas $B(x,\epsilon_2)$ is in $V$. 
Then we simply pick $\epsilon=\min\{\epsilon_1,\epsilon_2\}$ so that $B(x,\epsilon)$ is in $U \cap V$.
\end{proof}

\item An arbitrary intersection of closed sets is closed; a finite union of closed sets is closed.

\begin{proof}
This follows from de Morgan's Law on P1 and P2.
\end{proof}
\end{enumerate}

\begin{prbm}\label{sizes}
Compare the sizes of the following pairs of sets, i.e. determine if they are equal, or if one set may be a subset of the other.
\begin{enumerate}
\item $(A\cup B)^\circ$, $A^\circ\cup B^\circ$
\item $(A\cap B)^\circ$, $A^\circ\cap B^\circ$
\item \label{size} $\overline{A\cup B}$, $\bar{A}\cup\bar{B}$
\item $\overline{A\cap B}$, $\bar{A}\cap\bar{B}$
\end{enumerate}
\end{prbm}

\begin{proof} \
\begin{enumerate}
\item $(A\cup B)^\circ$ may be bigger

In $\RR$ we consider the intervals $A=(-1,0]$ and $B=[0,1)$, then
\[ A^\circ\cup B^\circ=(-1,0)\cup(0,1), \quad (A\cup B)^\circ=(-1,1) \]

For $x \in A^\circ \cup B^\circ$, we have either $x \in A^\circ$ or $x \in B^\circ$, so there is some ball centered at $x$ that is contained in either $A$ or $B$ and thus must be contained in $A\cup B$ as well.

\item Equal

If $x \in (A\cap B)^\circ$, then there exists a ball $U$ centered at $x$ such that $U$ is in both $A$ and $B$, so $x$ is in both $A^\circ$ and $B^\circ$.

On the other hand, $A^\circ\cap B^\circ$ is a subset of $A\cap B$; taking the interior of both sides, then since the intersection between two open sets is open we find that $A^\circ\cap B^\circ$ is a subset of $(A\cap B)^\circ$.

\item Equal

\item $\bar{A}\cap\bar{B}$ may be bigger
\end{enumerate}
\end{proof}

\begin{prbm}
Prove that the set of exterior points, $(A^c)^\circ$ is the same as $(\bar{A})^c$.
\end{prbm}

\begin{proof}
\begin{align*}
x \in (A^c)^\circ 
&\iff \exists \epsilon>0 \text{ such that } B(x,\epsilon) \subset A^c \\
&\iff B(x,\epsilon) \cap A = \emptyset \\
&\iff x \notin A \text{ and } B_0(x,\epsilon) \cap A=\emptyset \\
&\iff x \notin A \cup A^\prime = \bar A \\
&\iff x \in (\bar A^c)
\end{align*}
\end{proof}

\begin{prbm}
Regarding alternative descriptions:
\begin{enumerate}
\item $A$ is a neighbourhood of x if and only if there exists an open set U such that x is in U, U is subset of A (trivial except you'll actually need to prove that balls are open sets).
\item If $x$ is a limit point of $A$, then in fact for any $\epsilon>0$, $B(x,\epsilon)$ contains infinitely many elements of $A$ (you don't need to mention the punctured ball here because of obvious reasons; converse is trivial but a good and intuitive description).
\item $x$ is a boundary point of $A$ if and only if for all $\epsilon>0$, $B(x,\epsilon)$ intersects with both $A$ and $A^c$.
\end{enumerate}
\end{prbm}

\begin{proof} \
\begin{enumerate}
\item We show that $B(x,\epsilon)$ is open:

$\forall y \in B(x,\epsilon)$, 
\[ |y-x|<\epsilon \]

$\forall z \in B(y,\epsilon-|y-x|)$, 
\[ |z-x|\le|z-y|+|y-x|<\epsilon-|y-x|+|y-x|=\epsilon \]

$\therefore\:B(y,\epsilon-|y-x|) \subset B(x,\epsilon)$

\item We construct a sequence $\{x_n\}$ recursively as follows:
\begin{itemize}
\item Pick $x_1 \in B_0(x,\epsilon) \cap A$
\item Pick $x_{n+1} \in B_0(x,|x_n-x|) \cap A$
\end{itemize}
It is easy to see that the balls above are getting smaller so all $x_n$ are both mutually distinct and all contained in $B(x,\epsilon)$.

\item $x$ is a boundary point if and only if $x \in \bar{A} \setminus A^\circ$

\textbf{Forward direction}: 

We consider two cases
\begin{itemize}
\item $x \in A$, then all $B(x,\epsilon)$ intersects with $A$ at $x$, but since x is not in $A^\circ$ they must always intersect with $A^c$ as well.
\item $x \notin A$, then all $B(x,\epsilon)$ intersect with $A^c$ at $x$, but since $x \in \bar{A}$, $x$ is a limit point of $A$ and thus $B(x,\epsilon)$ always intersects with $A$.
\end{itemize}

\textbf{Backward direction}:

We consider two cases
\begin{itemize}
\item $x \in A$, then since $B(x,\epsilon)$ always intersects with $A^c$, $x$ cannot be in $A^\circ$.
\item $x \notin A$, then since $B(x,\epsilon)$ always intersects with $A$, $x$ must be in $\bar{A}$.
\end{itemize}

In fact we can describe the closure without referring to punctured balls and induced sets:
$x \in \bar{A}$ if and only if $B(x,\epsilon)$ always intersects with $A$

Also as a side note, $A\circ\cup \partial A\cup (A^c)\circ=\RR^n$
\end{enumerate}
\end{proof}

\begin{prbm}
Regarding closures (The following properties are relatively nontrivial compared to its 'open-set' counterparts):
\begin{enumerate}[label=(\alph*)]
\item $A^\prime$ is closed.
\item $\bar{A}$ is closed, i.e. bar(barA)=barA
\end{enumerate}
\end{prbm}

\begin{proof} \
\begin{enumerate}[label=(\alph*)]
\item In order to show that $A^\prime$ is closed, we need to show that if $x$ is a limit point of $A^\prime$, then $x\in A^\prime$, i.e. $x$ is a limit point of $A$.

So we need to show that limit points of $A^\prime$ are always limit points of $A$: 
Let $x$ be a limit point of $A^\prime$, then for all $\epsilon>0$, $B_0(x,\epsilon/2)$ intersects with $A^\prime$ and we may pick $y \in B_0(x,\epsilon/2)\cap A^\prime$

Now here's the tricky part
Since $y \in A^\prime$, y is a limit point of $A$, hence $B_0(y,|y-x|)$ intersects with $A$ and thus we may pick $z \in B_0(y,|y-x|)\cap A$.

We show that $z \in B_0(x,\epsilon)$:
\[ |z-x|\le|z-y|+|y-x|<2|y-x|<\epsilon, \]
hence $z \in B(x,\epsilon)$.
\[ |z-y|<|x-y|, \]
hence $z \neq x$

$\therefore\:z \in B_0(x,\epsilon)$

\item As for (b), it is just (a) and \cref{sizes} \cref{size}.
\end{enumerate}
\end{proof}

For homework, you'll work out some properties regarding dense sets

1. $A$ is a dense set in $X$ if and only if $A$ intersects with all open sets in $X$.
2. If $A$ is dense in $X$ and $B$ is dense in $A$, then $B$ is dense in $X$
3. If $A$ and $B$ are dense in $X$ where $A$ is open, then $A\cap B$ is dense in $X$

\section{Important Theorems}
\begin{thrm}{Cantor's Intersection Theorem}{}
Given a decreasing sequence of compact sets $A_1\supset A_2 \supset \cdots$, there exists a point $x\in\RR^n$ such that $x$ belongs to all $A_i$. In other words, $\bigcap_{i=1}^\infty A_i\neq\emptyset$. Moreover, if for all $i\in\NN$ we have $\diam A_{i+1}\le c\cdot\diam A_k$ for some constant $c<1$, then such a point must be unique, i.e. $\bigcap_{i=1}^\infty A_k=\{x\}$ for some $x\in\RR^n$.
\end{thrm}

\begin{thrm}{Heine--Borel Theorem}{}
A set $A\subset\RR^n$ is compact if and only if every open covering has a finite subcover, i.e. for any family of open sets $\mathscr{U}=\{U_i\}_{i\in I}$ satisfying $A\subset\bigcup_{i\in I}U_i$, there exists $\{U_1,\dots,U_n\}\subset\mathscr{U}$ such that $A\subset\bigcup_{i=1}^n U_i$.
\end{thrm}

\begin{thrm}{Bolzano--Weierstrass Theorem}{}
Infinite bounded sets in $\RR^n$ must contain limit points.
\end{thrm}

We will follow a very specific sequence of steps to prove these three theorems:
\begin{enumerate}[label=(\alph*)]
\item Cantor Intersection for $n=1$
\item Bolzano--Weierstrass for $n=1$
\item Bolzano--Weierstrass for general $n$
\item Cantor Intersection for general $n$
\item Heine--Borel for general $n$
\end{enumerate}

\begin{proof} \
\begin{enumerate}[label=(\alph*)]
\item Suppose that there is a decreasing sequence of compact sets $A_1, A_2, \dots$ in the real numbers

Since $A_k$ are bounded, we may let $a_k=\inf A_k$
Also since $A_k$ are closed, $a_k \in A_k$

Note that since $A_k$ is a decreasing sequence of sets we have $a_1\le a_2\le\dots$

Also, whenever we have $n>k$, we have $a_n \in A_n$, but $A_n \subset A_k$ and thus $a_n \in A_k$.

Let $b_1=\sup A_1$, then $a_k \in A_1$ and thus $a_k\le b_1$ for all $k$.

This tells us that the sequence $\{a_k\}$ is bounded above, and thus we may let $a=\sup a_k$.

Our goal is to show that the number $a$ appears in all $A_k$, thus showing that the entire intersection $\bigcap A_k$ contains $a$ and thus must be non-empty.

Now we split this in two cases, which asks whether a is simply made from isolated points, or if it is actually some nontrivial point obtained from the boundaries of $A_k$

\textbf{Case 1:} $a_k=a$ for some $k$
In this case we see that $a_k\le a_n\le a$ for all $n>k$ and thus $a_n=a$ in this case, therefore a is an element in $A_n$ for all $n$

In this case you can imagine that there is a possibility where a is an isolated minimum point of $A_n$ which stays there forever in the decreasing sequence of sets

\textbf{Case 2:} $a_k<a$ for all $k$; in this case we see that $a$ is the limit point of the increasing sequence $\{a_k\}$

Exercise 1: Show that $a$ is a limit point of each $A_k$.

Note that $a_n$ is in $A_k$ for each $n>k$, and since $a=\sup\{a_k\}$ where $a_k$ is increasing, we can actually show that a is a limit point of $\{a_n \mid n \le k\}$:
For every $\epsilon>0$, we pick $n_0$ such that $0 < a-a_{n_0} < \epsilon$
Pick $n\prime > \max\{k,n_0\}$, then $a_{n^\prime} \ge a_{n_0}$ and so
\[ 0<a-a_n\prime \le a_{n_0} < \epsilon \]
This shows that there exists $a_n^\prime$ in $B_0(a,\epsilon) \cap \{a_n \mid n>k\}$ for all $\epsilon$, and so $a$ is a limit point of $\{a_n \mid n>k\}$.

Now since $\{a_n|n \ge k\}$ is a subset of $A_k$ we also see that a is a limit point of $A_k$
Finally, since $A_k$ is closed, we conclude that $a$ is in $A_k$ for all $k$, and we are done

Wait hold on, I forgot about the second part

Now we consider a decreasing sequence of compact sets $A_1, A_2, \dots$ such that $\diam A_{k+1} \le c \diam A_k$ for $c<1$.

Suppose otherwise that there exists $x, y$ in $\bigcap A_k$

You can imagine that this will form a fixed distance between two points, and thus there is a constant positive lower bound for the diameters:
\[ \diam A_k \ge |x-y| > 0 \forall k \]

But this cannot be true because $\diam A_{k+1} \le c \diam A_k$ and so the diameter is controlled by a decreasing geometric sequence:
\[ \diam A_{k+1} \le c^k \diam A_1 \]

So we can simply pick a natural number $k$ such that
\[ k > \log_c \frac{|x-y|}{\diam A_1} \]

\item We consider an infinite bounded set $A$ in the real numbers. Since $A$ is bounded, we can pick a closed interval $[a_1,b_1]$ containing $A$.

We then perform a series of binary cuts: Consider the two halves of $[a_1,b_1]$. We know that at least one of these two must contain infinitely many elements in $A$, otherwise $A$ cannot be infinite. We pick this half of the interval and denote it by $[a_2,b_2]$. We continue this to pick a decreasing sequence of closed intervals $[a_n,b_n]$.

Now $\diam [a_{n+1},b_{n+1}] = \frac{1}{2} \diam [a_n,b_n]$, so by the Cantor Intersection Theorem, there exists a unique real number $c$ in the intersection $\bigcap[a_n,b_n]$.

We show that this $c$ is in fact a limit point of $A$.

For any $\epsilon>0$, we need to show that $B_0(c,\epsilon) \cap A \neq \emptyset$, i.e. we need to find an element $x \neq c$ in $A$ that is less than $\epsilon$ apart from $c$.

We then realize that we can simply exploit the decreasing sequence $[a_n,b_n]$
Since $\diam [a_n,b_n]$ is controlled by a decreasing sequence:
\[ \diam [a_{n+1},b_{n+1}] \le 1/2^n \diam [a_1,b_1] \]
We take a sufficiently large n so that $b_n-a_n<\epsilon$
Since $c$ is in $[a_n,b_n]$, for all $x$ in $[a_n,b_n]$ we have $|x-c|\le b_n-a_n<\epsilon$ and therefore $[a_n,b_n]$ is within $B(c,\epsilon)$.

Here's the funny part: $[a_n,b_n]$ contains infinitely many elements of $A$, so it must contain at least one element in A that is not $c$.

Therefore this element $x \neq c$ is in $B_0(c,\epsilon)$.

\item Now we have an infinte bounded set $A$ in $\RR^n$

The idea here is to consecutively come up with better and better sequences of points in $A$. We denote $x_i$ to be the $i$-th coordinate in $\RR^n$.

Our first wish is to pick some elements in $A$ so that they sort of converge at $x_1$.

Because such considerations of 'restricting to a single coordinate' is important here, we define the projection map to the $i$-th coordinate by
\[ f_i(x_1,\dots,x_n)=x_i \]

So, we look at $f_i(A)$ and try to apply BW for the case where $n=1$.

However, the problem is that $f_i(A)$ need not be infinite. For example, the set $\{(0,0),(0,1),(0,2),\dots\}$ projected onto the first coordinate is simply $\{0\}$.

This forces us to consider two cases

Exercise 2: Show that $f_i(A)$ is bounded
This is simple
1. $f_1(A)$ is infinite, then we can apply BW(n=1) to find a real number $c_1$ which is a limit point in $f_1(A)$

Here we can construct a sequence of points 
\[ \{x^{(1),1},x^{(1),2},...\} \]
so that their first coordinates satisfy
\[ |x^{(1),n}_1-c_1| < 1/n \]
for all natural number n
(I know this notation is cumbersome but the problem is that we need multiple sequences for this proof)

2. $f_1(A)$ is finite, then by the Pigeonhole Principle there exists a real number $c_1$ such that its preimage $f_1^{-1}(c_1)$ in $A$ is infinite

In this case we can randomly pick a sequence $\{x^{(1),1},x^{(1),2},\dots\}$ in $A$ so that their first coordinate is equal to $c_1$

I forgot to mention something that is implied, but we actually do have the need to emphasize that the sequence $\{x^{(1),1},x^{(1),2},\dots\}$ can be chosen to contain mutually distinct entries

Now that we have a sequence that behaves nice on the first coordinate, we may then move on to the second coordinate

Let $A_1=\{x^{(1),1},x^{(1),2},\dots\}$
We again consider $f_2(A_1)$ in two cases, infinite or finite

In any case, we are able to find a subsequence $\{x^{(2),1},x^{(2),2},\dots\}$, where
$x^{(2),k}=x^{(1),n_k}$ for some strictly increasing sequence of natural numbers $n_k$

So that, for the limit point/point with infinite preimage $c_2$, this sequence satisfies
\[ |f_2(x^{(2),n})-c_2| < \frac{1}{n} \]
Note that the property we have for the second case (we in fact have $f_2(x^{(2),n})=c_2$) is just a better version of this.

Now, take note that picking this subsequence does no harm whatsoever towards the first coordinate (if anything it would turn out to be better) since
\[ |f_1(x^{(2),k})-c_1| = |f_1(x^{(1),n_k}-c_1| < \frac{1}{n_k} \le \frac{1}{k} \]
($n_1<\dots<n_k$ is a strictly increasing sequence of natural numbers so $n_k \ge k$)

This continues on until we obtain a sequence of points $\{x^{(n),1},x^{(n),2},\dots\}$ in $A$ so that
\[ |f_i(x^{(n),k}-c_i|<\frac{1}{k} \quad \forall i,k \]

As we can see, the point $c=(c_1,\dots,c_n)$ is in fact a limit point of $A$ as we can always choose a big enough $k$ so that $x^{(n),k}$ is in $B(c,\epsilon) \cap A$.

Since $\{x^{(n),k}\}$ was always chosen to be a sequence of distinct entries, there is no danger for this sequence to always be c, and so c must be a limit point of $A$.

\item We may now return to the general case of Cantor.

Suppose that there is a sequence of decreasing compact sets $A_1,A_2,\dots$ in $\RR^n$. 
Note that every point is contained in $A_1$, so boundedness will never be an issue here.

Since $A_k$ are all nonempty, we can simply pick any element $a_k$ from $A_k$.

For the uncannily specific case that there are only finitely many $\{a_k\}$ chosen, we simply note that, again by Pigeonhole Principle, one of the $a_k$ appears infinitely often; thus for each $A_n$ we simply pick $n_k>n$ so that $A_{n_k}$ contains $a_k$, then $a_k$ is in $A_{n_k}$ which is a subset of $A_n$.

Otherwise, we can then note that $\{a_k\}$ is an infinite bounded set of points, so there must exist a limit point a of $\{a_k\}$.

We can now see that $a$ is always an element of $A_k$:
Using the same technique as Exercise 1, we see that a is a limit point of $\{a_n \mid n>k\}$ and so is a limit point of $A_k$, therefore a is in $A_k$ as $A_k$ is closed.

This proves the first part of the statement
The second part is completely identical to the second part of the $n=1$ case so we don't need to waste our time there either

\item We now consider a compact set A with some open covering $\mathscr{U}$.

This theorem is proved by contradiction: 
Suppose otherwise that set $A$ cannot be covered by any finite collection of open sets in $\mathscr{U}$

Since $A$ is compact, we may enclose it in a closed cube $Q_1$ (whose edges are parallel to the axes)

Now, for each step, we partition $Q$ into $2^n$ cubes by cutting it in half from each direction.

Then, starting from $Q_1$, there must exist one of these smaller cubes, denoted by $Q_2$, such that $A \cap Q_2$ cannot be covered by a finite collection of open sets in $\mathscr{U}$. 
Otherwise, if each $A \cap Q$ has a finite cover, then we simply collect all of these open sets together to form a finite cover of $A$, which violates our assumption.

We continue on to partition $Q_n$ and pick $Q_{n+1}$ so that $A_{n+1}$ has no finite cover (denote $A_n = A \cap Q_n$).

Note that $A$ and $Q_n$ are both compact, so $A_n$ is compact
Also we see that there is a decreasing sequence $A_1,A_2,\dots$
(we can't exactly obtain a relation between $\diam A_n$ and $\diam A_{n+1}$ here)

By Cantor Intersection Theorem we can always find a point $x$ in $A$ located in the intersection $\bigcap A_k$.

Now, since $\mathscr{U}$ is an open covering of $A$, there exists an open set $U$ in $\mathscr{U}$ such that $x\in U$.

The final key step is to exploit the sequence of decreasing cubes $Q_n$. So even though there isn't a clear cut way to control the sizes of $\diam A_n$, we do in fact have the property that $\diam Q_{n+1} = \frac{1}{2^n} \diam Q_1$.

Therefore, by picking a sufficiently large $n$, we can obtain $Q_n$ that is contained in $U$.

But this is a contradiction. 
This is because we've specifically chosen the sequence $A_n$ to be sets that do not possess any finite cover $\{U_1,...,U_n\}$ in $\mathscr{U}$. But here $A_n$ simply would have a one-element cover $\{U\}$.

This completes our proof.
\end{enumerate}
\end{proof}
\pagebreak

\section{Structures}\todo{to incorporate into other sections}
Let $X$ be a metric space. All points and sets mentioned below are understood to be elements and subsets of $X$ respectively.
\begin{itemize}

$A$ is \emph{open} if every point of $A$ is an interior point of $A$, i.e. $A^\circ=A$.

\item A point $x\in A$ is a \emph{limit point} of $A$ if every neighborhood of $x$ contains a point $y \neq x$ such that $y \in A$.

This means $B_0(x,\epsilon) \cap A \neq \emptyset$ for all $\epsilon>0$.

The \emph{induced set} of $A$, denoted by $A^\prime$, is the set of all limit points of $A$.

The \emph{closure} of $A$, denoted by $\bar{A}$, is the union set $A\cup A^\prime$. $A$ is \emph{closed} if all limit points of $A$ are contained in $A$, i.e. $\bar{A}=A$.

\item A point $x \in A$ is an \emph{isolated point} of $A$ if $x$ is not a limit point of $A$.

\item The \emph{boundary} of a set $A$, denoted by $\partial A$, is the set difference $\bar{A}\setminus A^\circ$.

A point $x$ is a \emph{boundary point} of $A$ if $x\in\partial A$.

\item A point $x$ is an \emph{exterior point} of $A$ if it is an interior point of $A^c$.

\item $A$ is \emph{perfect} if $A$ is closed and if every point of $A$ is a limit point of $A$.

\item $A$ is \emph{bounded} if there is a real number $M$ and a point $p \in X$ such that $d(x,p) < M$ for all $x \in A$.

\item A subset $B\subset A$ is a \emph{dense subset} of $A$ if $\bar{B}=A$.

\item $A$ is \emph{nowhere dense} if its closure has no interior, i.e. $(\bar{A})^\circ=\emptyset$.
\end{itemize}

\begin{defn}{Open set}{}
A subset $U \subset X$ is open if, for every point $x \in U$, there exists $\epsilon > 0$ such that $B_{\epsilon}(x) \subset U$.
\end{defn}

The idea is that, in a open set, there exists a ``safety margin" around every point. Given a point $p$, one can \emph{move around in the set a certain distance and remain} in the sense.

\begin{figure}[H]
    \centering
    \includegraphics[width=8cm]{images/open_set.png}
    \caption{Open set}
\end{figure}