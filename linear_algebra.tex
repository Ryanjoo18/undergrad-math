\part{Linear Algebra}
\chapter{Linear Equations}
\section{Systems of Linear Equations}
Suppose $F$ is a field. We consider tbe problem of finding $n$ scalars (elements of $F$) $x_1,\dots,x_n$ which satisfy the conditions
\begin{equation}\label{eqn:system-linear-eqn}
\begin{split}
a_{11}x_1+a_{12}x_2+\cdots+a_{1n}x_n&=b_1\\
a_{21}x_1+a_{22}x_2+\cdots+a_{2n}x_n&=b_2\\
\vdots&=\vdots\\
a_{m1}x_1+a_{m2}x_2+\cdots+a_{mn}x_n&=b_m
\end{split}
\end{equation}
where $b_1,\dots,b_m$ and $a_{ij},1\le i\le m,1\le j\le n$ are given elements of $F$. We call \ref{eqn:system-linear-eqn} a \vocab{system of $m$ linear equations in $n$ unknowns}.

Any $n$-tuple $(x_1,\dots,x_n)$ of elements of $F$ which satisfies each of the equations in \ref{eqn:system-linear-eqn} is called a \vocab{solution} of the system.

If $b_1=\cdots=b_m=0$, we say that the system is \vocab{homogeneous}, or that each of the equations is homogeneous.

For the general system \ref{eqn:system-linear-eqn}, suppose we select $m$ scalars $c_1,\dots,c_m$, multiply the $j$-th equation by $c_j$ and then add up all the $m$ equations. We obtain
\[(c_1a_{11}+\cdots+c_ma_{m1})x_1+\cdots+(c_1a_{1n}+\cdots+c_ma_{mn})x_n=c_1b_1+\cdots+c_mb_m\]
which we call a \vocab{linear combination} of the equations in \ref{eqn:system-linear-eqn}. Evidently, any soltion of the entire system of equations \ref{eqn:system-linear-eqn} will also be a solution of this new equation. This is the fundamental idea of the elimination process to find the solution(s) of a system of linear equations.

\section{Matrices and Elementary Row Operations}
\section{Row-Reduced Echelon Matrices}
\section{Matrix Multiplication}
\section{Invertible Matrices}