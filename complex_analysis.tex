\part{Complex Analysis}
\chapter{Complex Numbers}
\section{The Algebra of Complex Numbers}
\subsection{Arithmetic Operations}
\vocab{Imaginary unit} $i=\sqrt{-1}$. A \vocab{complex number} $z=\alpha+i\beta$ where $\alpha,\beta\in\RR$; $\alpha$ and $\beta$ are the real and imaginary parts of the complex number respectively. If $\alpha=0$, the number is said to be purely imaginary; if $\beta=0$, it is real. Zero is the only number which is at once real and purely imaginary.

Two complex numbers are equal if and only if they have the same real part and the same imaginary part.

Addition and multiplication do not lead out from the system of complex numbers. Assuming that the ordinary rules of arithmetic apply to complex numbers we find indeed
\[(\alpha+i\beta)+(\gamma+i\delta)=(\alpha+\gamma)+i(\beta+\delta)\]
and
\[(\alpha+i\beta)(\gamma+i\delta)=(\alpha\gamma-\beta\delta)+i(\alpha\delta+\beta\gamma),\]
making use of the relation $i^2=-1$.

It is less obvious that division is also possible. We wish to show that $\dfrac{\alpha+i\beta}{\gamma+i\delta}$ is a complex number, provided that $\gamma+i\delta\neq0$. Denoting the quotient by $x+iy$, we have
\begin{align*}
\alpha+i\beta&=(\gamma+i\delta)(x+iy)\\
&=
\end{align*}

\subsection{Square Roots}
\subsection{Justification}
\subsection{Conjugation, Absolute Value}
\subsection{Inequalities}