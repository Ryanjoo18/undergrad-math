\chapter{Ring Theory}\label{chap:ring-theory}
\section{Introduction to Rings}
\begin{definition}[Ring]
A \vocab{ring} is a set $R$ together with two binary operations, addition and multiplication, denoted $+$ and $\times$, satisfying the following axioms:
\begin{enumerate}[label=(\roman*)]
\item $(R,+)$ is an abelian group;
\item $\times$ is associative: $a\times(b\times c)=(a\times b)\times c$ for all $a,b,c\in R$;
\item Distributive laws: $a\times(b+c)=(a\times b)+(a\times c)$ and $(a+b)\times c=(a\times c)+(b\times c)$ for all $a,b,c\in R$.
\end{enumerate}

$R$ is \vocab{commutative} if multiplication is commutative.

$R$ is said to have an \vocab{identity} if there exists $1\in R$ with $1\times a=a\times1=a$ for all $a\in R$.
\end{definition}

\begin{notation}
We simply write $ab$ rather than $a\times b$ for $a,b\in R$.
\end{notation}

\begin{notation}
The additive identity of $R$ is denoted by $0$; the additive inverse of $a\in R$ is denoted by $-a$.
\end{notation}

\begin{definition}
A ring $R$ with identity $1$, where $1\neq0$, is called a \vocab{division ring} if every $a\in R$, $a\neq0$ has a multiplicative inverse, i.e. there exists $b\in R$ such that $ab=ba=1$.

A commutative division ring is called a \vocab{field}.
\end{definition}

\begin{example}
$\ZZ$ under usual addition and multiplication is a commutative ring with identity $1$.

$\QQ$, $\RR$, $\CC$ are field.

$\ZZ/n\ZZ$ is a commutative ring with identity $\bar{1}$ under addition and multiplication of residue classes.
\end{example}

\begin{proposition}
Let $R$ be a ring. Then
\begin{enumerate}[label=(\arabic*)]
\item $0a=a0=0$ for all $a\in R$.
\item $(-a)b=a(-b)=-(ab)$ for all $a,b\in R$.
\item $(-a)(-b)=ab$ for all $a,b\in R$.
\item if $R$ has identity $1$, then the identity is unique and $-a=(-1)a$.
\end{enumerate}
\end{proposition}

\begin{definition}
Let $R$ be a ring. A non-zero element $a\in R$ is called a \vocab{zero divisor} if there exists $b\in R$, $b\neq0$ such that either $ab=0$ or $ba=0$.

Assume $R$ has an identity $1\neq0$. $u\in R$ is called a \vocab{unit} in $R$ if there exists $v\in R$ such that $uv=vu=1$. The set of units in $R$ is denoted by $R^\times$.
\end{definition}

\begin{proposition}
The units in a ring $R$ form a group under multiplication.
\end{proposition}

We thus call $R^\times$ the \vocab{group of units} of $R$.

\begin{definition}
A commutative ring with identity $1\neq0$ is called an \vocab{integral domain} if it has no zero divisors.
\end{definition}

The absence of zero divisors in integral domains give these rings a cancellation property:

\begin{proposition}
Let $R$ be a ring. $a,b,c\in R$, $a$ is not a zero divisor. If $ab=ac$, then either $a=0$ or $b=c$. In particular, for any $a,b,c$ in an integral domain and $ab=ac$, then either $a=0$ or $b=c$.
\end{proposition}

\begin{corollary}
Any finite integral domain is a field.
\end{corollary}

\begin{definition}[Subring]
$S\subset R$ is a \vocab{subring} of ring $R$ if $S$ is a subgroup of $R$ that is closed under multiplication.
\end{definition}

\begin{example}[Quadratic integer rings]

\end{example}

\begin{example}[Polynomial rings]

\end{example}

\begin{example}[Matrix rings]

\end{example}

\begin{example}[Group rings]

\end{example}

A ring homomorphism is a map from one ring to another that respects the additive and multiplicative structures:

\begin{definition}[Ring homomorphism]
Let $R$ and $S$ be rings. A \vocab{ring homomorphism} is a map $\phi:R\to S$ satisfying
\begin{enumerate}[label=(\roman*)]
\item $\phi(a+b)=\phi(a)+\phi(b)$ for all $a,b\in R$;
\item $\phi(ab)=\phi(a)\phi(b)$ for all $a,b\in R$.
\end{enumerate}
\end{definition}

\begin{definition}
The \vocab{kernel} of the ring homomorphism $\phi$, denoted by $\ker\phi$ is the set of elements of $R$ that map to $0$ in $S$:
\[\ker\phi\coloneqq\{a\in R\mid\phi(a)=0\}\]
\end{definition}

\begin{definition}[Isomorphism]
A bijective ring homomorphism is called an \vocab{isomorphism}, denoted by $R\cong S$.
\end{definition}

%Ideals, quotient rings, isomorphism theorems. Prime and maximal ideals. Field of fractions of an integral domain. Factorization in rings; units, primes and irreducibles. Unique factorization in principal ideal domains, and in polynomial rings. Gauss’ Lemma and Eisenstein’s irreducibility criterion. Rings $\ZZ[\alpha]$ of algebraic integers as subsets of $\CC$ and quotients of $\ZZ[x]$. Examples of Euclidean domains and uniqueness and non-uniqueness of factorization. Factorization in the ring of Gaussian integers; representation of integers as sums of two squares. Ideals in polynomial rings. Hilbert basis theorem