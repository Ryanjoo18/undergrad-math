\chapter{Ring Theory}
\section{Definition}
A ring is just a set where you can add, subtract, and multiply. In some rings you can divide, and in others you can't. There are many familiar examples of rings, the main ones falling into two camps: ``number systems'' and ``functions''.

\begin{definition}
A \vocab{ring} is a set $R$ endowed with two binary operations, addition and multiplication, denoted $+$ and $\times$, with elements $0,1\in R$, which maps $+: R \times R \to R$ and $\times: R \times R \to R$, subject to three axioms:
\begin{enumerate}
\item $(R,+)$ is an abelian group with identity $0$.
\item $(R,\times)$ is a commutative semigroup, i.e. $a \times (b \times c) = (a \times b) \times c$, $a \times 1 = 1 \times a = a$, and $a \times b = b \times a$ for all $a, b, c \in R$.
\item Distributivity: $a \times (b + c) = a \times b + a \times c$ for all $a, b, c \in R$.
\end{enumerate}
\end{definition}

\begin{example}
Examples of rings:
\begin{itemize}
\item $\ZZ$: the integers $\dots,-2,-1,0,1,2,\dots$ with usual addition and multiplication, form a ring. Note that we cannot always divide, since 1/2 is no longer an integer.

\item $2\ZZ$: the even integers $\dots,-4,-2,0,2,4,\dots$

\item $\ZZ[x]$: this is the set of polynomials whose coefficients are integers. 

It is an extension of $\ZZ$, in the sense that we allow all the integers, plus an “extra symbol” $x$, which we are allowed to multiply and add, giving rise to $x^2$, $x^3$, etc., as well as $2x$, $3x$, etc. Adding up various combinations of these gives all the possible integer polynomials.

\item $\ZZ[x,y,z]$: polynomials in three variables with integer coefficients. 

This is an extension of the previous ring. In fact you can continue adding variables to get larger and larger rings.

\item $\ZZ/n\ZZ$: integers mod $n$. 

These are equivalence classes of the integers under the equivalence relation “congruence mod n”. If we just think about addition (and subtraction), this is exactly the cyclic group of order $n$. However, when we call it a ring, it means we are also using the operation of multiplication.

\item $\QQ$, $\RR$, $\CC$
\end{itemize}
\end{example}

Ideals, homomorphisms, quotient rings, isomorphism theorems. Prime and maximal ideals. Fields. The characteristic of a field. Field of fractions of an
integral domain.
Factorization in rings; units, primes and irreducibles. Unique factorization in principal ideal domains, and
in polynomial rings. Gauss’ Lemma and Eisenstein’s irreducibility criterion.
Rings $\ZZ[\alpha]$ of algebraic integers as subsets of $\CC$ and quotients of $\ZZ[x]$. Examples of Euclidean domains and
uniqueness and non-uniqueness of factorization. Factorization in the ring of Gaussian integers; representation of integers as sums of two squares.
Ideals in polynomial rings. Hilbert basis theorem