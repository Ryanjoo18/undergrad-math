\chapter{Group Theory}\label{chap:group-theory}
\section{Introduction to Groups}
\begin{definition}[Binary operation]
A \vocab{binary operation} $\ast$ on a set $G$ is a function $\ast:G\times G\to G$. For any $a,b\in G$, we write $a \ast b$ for the image of $(a,b)$ under $\ast$.

$\ast$ is \vocab{associative} on $G$ if $(a\ast b)\ast c=a\ast(b\ast c)$ for all $a,b,c\in G$.

$\ast$ is \vocab{commutative} on $G$ if $a\ast b=b\ast a$ for all $a,b\in G$.
\end{definition}

\begin{definition}[Group]
A \vocab{group} is a pair $(G,\ast)$, where $G$ is a set and $\ast$ is a binary operation on $G$ satisfying the following group axioms:
\begin{enumerate}[label=(\roman*)]
\item Associativity: $a\ast(b\ast c)=(a\ast b)\ast c$ for all $a,b,c\in G$.
\item Identity: there exists identity element $e\in G$ such that $a\ast e=e\ast a=a$ for all $a\in G$.
\item Invertibility: for all $a\in G$, there exists inverse $c\in G$ such that $a\ast c=c\ast a=e$.
\end{enumerate}

$G$ is \vocab{abelian} if the operation is commutative; it is \vocab{non-abelian} if otherwise.
\end{definition}

\begin{remark}
The ``closure axiom'' (for all $a,b,c \in G$, $a\ast b\in G$) is implicitly implied, as a binary operation has to be closed under the set.
\end{remark}

\begin{notation}
We simply denote a group $(G,\ast)$ by $G$ if the operation is clear.
\end{notation}

\begin{notation}
We abbreviate $a\ast b$ to just $ab$ if the operation is clear.
\end{notation}

\begin{notation}
Since the operation $\ast$ is associative, we can omit unnecessary parentheses and write $(ab)c=a(bc)=abc$.
\end{notation}

\begin{notation}
For any $a\in G$, $n\in\ZZ^+$ we abbreviate $a^n=\underbrace{a\cdots a}_\text{$n$ times}$.
\end{notation}

\begin{notation}
We write $(\ZZ,+)$, $(\QQ,+)$, $(\RR,+)$, $(\CC,+)$ as simply $\ZZ$, $\QQ$, $\RR$, $\CC$.
\end{notation}

\begin{example}
The following are some examples of groups.
\begin{itemize}
\item $\ZZ$, $\QQ$, $\RR$, $\CC$ are groups, with identity $0$ and (additive) inverse $-a$ for all $a$.
\item $\QQ\setminus\{0\}$, $\RR\setminus\{0\}$, $\CC\setminus\{0\}$, $\QQ^+$, $\RR^+$ are groups under $\times$, with identity $1$ and (multiplicative) inverse $\frac{1}{a}$ for all $a$; $\ZZ\setminus\{0\}$ is not a group under $\times$, because all elements except for $\pm1$ do not have an inverse in $\ZZ\setminus\{0\}$.
\item For $n\in\ZZ^+$, $\ZZ/n\ZZ$ is an abelian group under $+$.
\item For $n\in\ZZ^+$, $(\ZZ/n\ZZ)^\times$ is an abelian group under multiplication.
\end{itemize}
\end{example}

\begin{definition}[Product group]
Let $(G,\ast_G)$ and $(H,\ast_H)$ be groups. Then the operation $\ast$ is defined on $G\times H$ by
\[(g_1,h_1)\ast(g_2,h_2)=(g_1\ast_G g_2,h_1\ast_H h_2)\]
for all $g_1,g_2\in G$, $h_1,h_2\in H$. $(G\times H, \ast)$ is called the \vocab{product group} of $G$ and $H$.
\end{definition}

\begin{proposition}
The product group is a group.
\end{proposition}
    
\begin{proof} \
\begin{enumerate}[label=(\arabic*)]
\item Since $\ast_G$ and $\ast_H$ are both associative binary operations, it follows that $\ast$ is also an associative binary operation on $G \times H$.
\item We also note
\[e_{G\times H}=(e_G,e_H),\quad(g,h)^{-1}=(g^{-1},h^{-1})\]
as for any $g \in G$, $h \in H$,
\[(e_G,e_H)\ast(g,h)=(g,h)=(g,h)\ast(e_G,e_H).\]
\item As for identity,
\[(g^{-1},h^{-1})\ast(g,h)=(e_G,e_H)=(g,h)\ast(g^{-1},h^{-1}).\]
\end{enumerate}
\end{proof}

\begin{proposition}
Let $G$ be a group. Then
\begin{enumerate}[label=(\arabic*)]
\item the identity of $G$ is unique,
\item for each $a\in G$, $a^{-1}$ is unique,
\item $(a^{-1})^{-1}=a$ for all $a\in G$,
\item $(ab)^{-1}=b^{-1}a^{-1}$,
\item for any $a_1,\dots,a_n\in G$, $a_1\cdots a_n$ is independent of how we arrange the parantheses (generalised associative law).
\end{enumerate}
\end{proposition}

\begin{proof} \
\begin{enumerate}[label=(\arabic*)]
\item Suppose otherwise, then $e$ and $e^\prime$ are identites of $G$. We have
\[e=ee^\prime=e^\prime\]
where the first equality holds as $e^\prime$ is an identity, and the second equality holds as $e$ is an identity. Since $e=e^\prime$, the identity is unique.
\item Suppose otherwise, then $b$ and $c$ are both inverses of $a$. Let $e$ be the identity of $G$. Then $ab=e$, $ca=e$. Thus
\[c=ce=c(ab)=(ca)b=eb=b.\]
Hence the inverse is unique.
\item To show $(a^{-1})^{-1}=a$ is exactly the problem of showing that $a$ is the inverse of $a^{-1}$, which is by definition of the inverse (with the roles of $a$ and $a^{-1}$ interchanged).
\item Let $c=(ab)^{-1}$. Then $(ab)c=e$, or $a(bc)=e$ by associativity, which gives $bc=a^{-1}$ and thus $c=b^{-1}a^{-1}$ by multiplying $b^{-1}$ on both sides.
\item The result is trivial for $n=1,2,3$. For all $k<n$ assume that any $a_1\cdots a_k$ is independent of parantheses. Then
\[(a_1\cdots a_n)=(a_1\cdots a_k)(a_{k+1}\cdots a_n).\]
Then by assumption both are independent of parentheses since $k,n-k<n$ so by induction we are done.
\end{enumerate}
\end{proof}

\begin{notation}
Since the inverse is unique, we denote the inverse of $a\in G$ by $a^{-1}$.
\end{notation}

\begin{proposition}[Cancellation law]
Let $a,b\in G$. Then the equations $ax=b$ and $ya=b$ have unique solutions for $x,y\in G$. In particular, we can cancel on the left and right.
\end{proposition}

\begin{proof}
That $x=a^{-1}b$ is unique follows from the uniqueness of $a^{-1}$ and the same for $y=ba^{-1}$.
\end{proof}

\begin{definition}[Order]
For a group $G$ and $x\in G$, the \vocab{order} of $x$, denoted by $|x|$, is defined to be the smallest positive integer $n$ such that $x^n=1$; in this case $x$ is said to be of order $n$.

If no positive power of $x$ is the identity, the order of $x$ is defined to be infinity, and $x$ is said to be of infinite order.
\end{definition}

\begin{example}
Some examples to illustrate the above concept.
\begin{itemize}
\item An element of a group has order 1 if and only if it is the identity.
\item In the additive groups $\ZZ$, $\QQ$, $\RR$, $\CC$, every non-zero (i.e. non-identity) element has infinite order.
\item In the multiplicative groups $\RR\setminus\{0\}$ or $\QQ\setminus\{0\}$, the element $-1$ has order 2 and all other non-identity elements have infinite order.
\item In $\ZZ/9\ZZ$, the element $\overline{6}$ has order 3. (Recall that in an additive group, the powers of an element are integer multiples of the element.)
\item In $(\ZZ/7\ZZ)^\times$, the powers of the element $\overline{2}$ are $\overline{2},\overline{4},\overline{8}=\overline{1}$, the identity in this group, so 2 has order 3. Similarly, the element $\overline{3}$ has order 6, since $3^6$ is the smallest positive power of 3 that is congruent to 1 mod 7.
\end{itemize}
\end{example}

\begin{example}[Dihedral groups]
An important family of groups is the dihedral groups. For $n\in\ZZ^+$, $n\ge3$, let $D_{2n}$ be the set of symmetries\footnote{a symmetry is any rigid motion of the $n$-gon which can be effected by taking a copy of the $n$-gon, moving this copy in any fashion in $3$-space and then placing the copy back on the original $n$-gon so it exactly covers it. A symmetry can be a reflection or a rotation.} of a regular $n$-gon.

\begin{remark}
Here ``D'' stands for ``dihedral'', meaning two-sided.
\end{remark}

To visualise this, we first choose a labelling of the $n$ vertices. Then each symmetry $S$ can be described uniquely by the corresponding permutation $\sigma$ of $\{1,2,\dots,n\}$ where if the symmetry $s$ puts vertex $i$ in the place where vertex $j$ was originally, then $\sigma$ is the permutation sending $i$ to $j$.

We now make $D_{2n}$ into a group. For $S,T\in D_{2n}$, define the binary operation $ST$ to be the symmetry obtained by first applying $T$ then $S$ to the $n$-gon (this is analagous to function composition). If $S$ and $T$ effect the permutations $\sigma$ and $\tau$ respectively on the vertices, then $ST$ effects $\sigma\circ\tau$.

\begin{enumerate}[label=(\roman*)]
\item The binary operation on $D_{2n}$ is associative since the composition of functions is associative.
\item The identity of $D_{2n}$ is the identity symmetry, which leaves all vertices fixed, denoted by $1$.
\item The inverse of $S\in D_{2n}$ is the symmetry which reverses all rigid motions of $S$ (so if $S$ effects permutation $\sigma$ on the vertices, $S^{-1}$ effects $\sigma^{-1}$).
\end{enumerate}

Let $r$ be the rotation clockwise about the origin by $\frac{2\pi}{n}$ radians, let $s$ be the reflection about the line of symmetry through the first labelled vertex and the origin.

\begin{proposition} \
\begin{enumerate}[label=(\arabic*)]
\item $|r|=n$
\item $|s|=2$
\item $s\neq r^i$ for all $i$
\item $sr^i\neq sr^j$ for all $i\neq j$ ($0\le i,j\le n-1$), so
\[D_{2n}=\{1,r,\dots,r^{n-1},s,sr,\dots,sr^{n-1}\}\]
and thus $|D_{2n}|=2n$.
\item $rs=sr^{-1}$
\item $r^is=sr^{-i}$
\end{enumerate}
\end{proposition}

\begin{proof} \
\begin{enumerate}[label=(\arabic*)]
\item It is obvious that $1,r,r^2,\dots,r^{n-1}$ are all distinct and $r^n=1$, so $|r|=n$.
\item This is fairly obvious: either reflect or do not reflect.
\item This is also obvious: the effect of any reflection cannot be obtained from any form of rotation.
\item Just cancel on the left and use the fact that $|r|=n$. We assume that $i\not\equiv j\pmod n$.
\item Omitted.
\item By (5), this is true for $i=1$. Assume it holds for $k<n$. Then $r^{k+1}s=r(r^ks)=rsr^{-k}$. Then $rs=sr^{-1}$ so $rsr^{-k}=sr^{-1}r^{-k}=sr^{-k-1}$ so we are done.
\end{enumerate}
\end{proof}

A presentation for the dihedral group $D_{2n}$ using generators and relations is
\[D_{2n}=\langle r,s\mid r^n=s^2=1,rs=sr^{-1}\rangle.\]
\end{example}

\begin{example}[Symmetric groups]
Let $\Omega$ be any non-empty set, let $S_\Omega$ be the set of all bijections from $\Omega$ to itself (i.e., the set of all permutations of $\Omega$).

$S_\Omega$ is a group under function composition $\circ$. We show that the group axioms hold for $(S_\Omega,\circ)$:
\begin{enumerate}[label=(\roman*)]
\item $\circ$ is a binary operation on $S_\Omega$ since if $\sigma:\Omega\to\Omega$ and $\tau:\Omega\to\Omega$ are both bijections, then $\sigma\circ\tau$ is also a bijection from $\Omega$ to $\Omega$.
\item Since function composition is associative in general, $\circ$ is associative.
\item The identity of $S_\Omega$ is $1$, defined by $1(a)=a$ for all $a\in\Omega$.
\item For every permutation $\sigma$, there is a (2-sided) inverse function $\sigma^{-1}:\Omega\to\Omega$ satisfying $\sigma\circ\sigma^{-1}=\sigma^{-1}\circ\sigma=1$.
\end{enumerate}

$(S_\Omega,\circ)$ is called the \vocab{symmetric group} on $\Omega$. In the special case where $\Omega=\{1,2,\dots,n\}$, the symmetric group on $\Omega$ is denoted $S_n$, the symmetric group of degree $n$.

\begin{proposition}
The order of $S_n$ is $n!$.
\end{proposition}

\begin{proof}
Obvious, since there are $n!$ permutations of $\{1,2,\dots,n\}$.
\end{proof}
\end{example}

\begin{example}[Matrix groups]

A field is denoted by $\FF$; $\FF^\times=\FF\setminus\{0\}$.

For $n\in\ZZ^+$, let $GL_n(\FF)$ be the set of all $n\times n$ invertible matrices whose entries are in $\FF$:
\[GL_n(\FF)=\{A\mid A\in M_{n\times n}(\FF),\det(A)\neq0\}.\]

We show that $GL_n(\FF)$ is a group under matrix multiplication:
\begin{enumerate}[label=(\roman*)]
\item Since $\det(AB)=\det(A)\cdot\det(B)$, it follows that if $\det(A)\neq0$ and $\det(B)\neq0$, then $\det(AB)\neq0$, so $GL_n(\FF)$ is closed under matrix multiplication.
\item Matrix multiplication is associative.
\item $\det(A)\neq0$ if and only if $A$ has an inverse matrix, so each $A\in GL_n(\FF)$ has an inverse $A^{-1}\in GL_n(\FF)$ such that
\[AA^{-1}=A^{-1}A=I\]
where $I$ is the $n\times n$ identity matrix.
\end{enumerate}

We call $GL_n(\FF)$ the \vocab{general linear group} of degree $n$.
\end{example}

\begin{example}[Quaternion group]
The \vocab{Quaternion group} $Q_8$ is defined by
\[Q_8=\{1,-1,i,-i,j,-j,k,-k\}\]
with product $\cdot$ computed as follows:
\begin{itemize}
\item $1\cdot a=a\cdot 1=a$ for all $a\in Q_8$
\item $(-1)\cdot(-1)=1$
\item $(-1)\cdot a=a\cdot(-1)=-a$ for all $a\in Q_8$
\item $i\cdot i=j\cdot j=k\cdot k=-1$
\item $i\cdot j=k$, $j\cdot i=-k$, $j\cdot k=i$, $k\cdot j=-i$, $k\cdot i=j$, $i\cdot k=-j$
\end{itemize}
Note that $Q_8$ is a non-abelian group of order $8$.
\end{example}

\begin{definition}[Homomorphism]
Let $(G,\ast)$ and $(H,\diamond)$ be groups. A map $\phi:G\to H$ such that
\[\phi(x\ast y)=\phi(x)\diamond\phi(y)\quad(\forall x,y\in G)\]
is called a \vocab{homomorphism}.
\end{definition}

\begin{definition}[Isomorphism]
The map $\phi:G\to H$ is called an \vocab{isomorphism}, $G$ and $H$ are said to be \vocab{isomorphic}, denoted by $G\cong H$, if
\begin{enumerate}[label=(\roman*)]
\item $\phi$ is a homomorphism, and
\item $\phi$ is a bijection.
\end{enumerate}
\end{definition}

\begin{example}
$\ZZ\cong10\ZZ$ as the map $\phi:\ZZ\to10\ZZ \text{ by } x \mapsto 10x$ is a homomorphism and a bijection.

In other words, $\phi$ is a way of re-assigning names of the elements without changing the structure of the group.
\end{example}

\begin{proposition}
If $\phi:G\to H$ is an isomorphism, then
\begin{enumerate}[label=(\arabic*)]
\item $|G|=|H|$;
\item $G$ is abelian if and only if $H$ is abelian;
\item $|x|=|\phi(x)|$ for all $x\in G$.
\end{enumerate}
\end{proposition}

\begin{definition}[Group action]
A \vocab{group action} on a group $G$ on a set $A$ is a map from $G\times A$ to $A$ (written as $g\cdot a$, for all $g\in G$ and $a\in A$) satisfying the following properties:
\begin{enumerate}[label=(\roman*)]
\item $g_1\cdot(g_2\cdot a)=(g_1g_2)\cdot a$, for all $g_1,g_2\in G$, $a\in A$;
\item $1\cdot a=a$, for all $a\in A$.
\end{enumerate}
\end{definition}

We shall be less formal and say $G$ is a group acting on a set $A$.

\begin{notation}
$g\cdot a$ will usually be written simply as $ga$ where there is no danger of confusing this map with, say, the group operation (remember, $\cdot$ is not a binary operation and $ga$ is always a member of $A$).
\end{notation}

Let the group $G$ act on the set $A$. For each fixed $g\in G$ we get a map $\sigma_g:A\to A$ defined by $\sigma_g(a)=g\cdot a$.

\begin{proposition} \
\begin{enumerate}[label=(\arabic*)]
\item For each fixed $g\in G$, $\sigma_g$ is a permutation of $A$.
\item The map from $G$ to $S_A$ defined by $g\mapsto\sigma_g$ is a homomorphism.
\end{enumerate}
\end{proposition}

\section{Subgroups}
\begin{definition}[Subgroup]
Let $G$ be a group. $H\subseteq G$, $H\neq\emptyset$ is a \vocab{subgroup} of $G$, denoted $H\le G$, if the group operation $\ast$ restricts to make a group of $H$, i.e.
\begin{enumerate}[label=(\roman*)]
\item $e\in H$;
\item $xy\in H$ for all $x,y\in H$;
\item $x^{-1}\in H$ for all $x\in H$.
\end{enumerate}
\end{definition}

\begin{remark}
Observe that if $\ast$ is an associative (respectively, commutative) binary operation on $G$ and $\ast$ is restricted to some $H\subseteq G$ is a binary operation on $H$, then $\ast$ is automatically associative (respectively, commutative) on $H$ as well.
\end{remark}

\begin{lemma}[Subgroup criterion]
Let $G$ be a group. $H\subseteq G$, $H\neq\emptyset$ is a subgroup of $G$ if and only if $xy^{-1}\in H$ for all $x,y\in H$. 

Furthermore, if $H$ is finite, then it suffices to check that $H$ is non-empty and closed under multiplication.
\end{lemma}

\begin{proof}
If $H$ is a subgroup of $G$, then we are done, by definition of subgroup.

Conversely, we want to prove that for $H\neq\emptyset$, if $xy^{-1}\in H$ for all $x,y\in H$, then $H\le G$:
\begin{enumerate}[label=(\roman*)]
\item Since $H\neq\emptyset$, take $x\in H$, let $y=x$, then $1=xx^{-1}\in H$, so $H$ contains the identity of $G$.
\item Since $1\in H$, $x\in H$, then $x^{-1}\in H$ so $H$ is closed under taking inverses.
\item For any $x,y\in H$, $x,y^{-1}\in H$, so by (ii), $x(y^{-1})^{-1}=xy\in H$, so $H$ is closed under multiplication.
\end{enumerate}
Hence $H$ is a subgroup of $G$.

For the last part, suppose that $H$ is finite and closed under multiplication. Take $x\in H$. Then there are only finitely many distinct elements among $x,x^2,x^3,\dots$ and so $x^a=x^b$ for $a,b\in\ZZ$ with $a<b$. If $n=b-a$, then $x^n=1$ so in particular every element $x\in H$ is of finite order. Then $x^{n-1}\in x^{-1}\in H$, so $H$ is closed under inverses.
\end{proof}

We now introduce some important families of subgroups of an arbitrary group $G$. Let $A\subseteq G$, $A\neq\emptyset$.

\begin{definition}[Centraliser]
The \vocab{centraliser} of $A$ in $G$ is defined by
\[C_G(A)\coloneqq\{g\in G\mid\forall a\in A,gag^{-1}=a\}.\]
Since $gag^{-1}=a$ if and only if $ga=ag$, $C_G(A)$ is the set of elements of $G$ which commute with every element of $A$.
\end{definition}

\begin{proposition}
$C_G(A)$ is a subgroup of $G$.
\end{proposition}

\begin{notation}
In the special case when $A=\{a\}$ we simply write $C_G(a)$ instead of $C_G(\{a\})$. In this case $a^n\in C_G(a)$ for all $n\in\ZZ$.
\end{notation}

\begin{definition}[Center]
The \vocab{center} of $G$ is the set of elements commuting with all the elements of $G$:
\[Z(G)\coloneqq\{g\in G\mid\forall x\in G,gx=xg\}.\]
\end{definition}

\begin{proposition}
$Z(G)$ is a subgroup of $G$.
\end{proposition}

\begin{proof}
Note that $Z(G)=C_G(G)$, so the argument above proves $Z(G)\le G$ as a special case.
\end{proof}

\begin{definition}[Normaliser]
Define $gAg^{-1}=\{gag^{-1}\mid a\in A\}$. The \vocab{normaliser} of $A$ in $G$ is
\[N_G(A)\coloneqq\{g\in G\mid gAg^{-1}=A\}.\]
\end{definition}

\begin{proposition}
$N_G(A)$ is a subgroup of $G$.
\end{proposition}

\begin{proof}
Notice that if $g\in C_G(A)$, then $gag^{-1}=a\in A$ for all $a\in A$ so $C_G(A)\le N_G(A)$.
\end{proof}

The fact that the normaliser of $A$ in $G$, the centraliser of $A$ in $G$, and the center of $G$ are all subgroups are special cases of results on group actions.

\begin{definition}[Stabiliser]
If $G$ is a group acting on a set $S$, $s\in S$, then the \vocab{stabiliser} of $s$ in $G$ is
\[G_s=\{g\in G\mid g\cdot s=s\}.\]
\end{definition}

\begin{proposition}
$G_s$ is a subgroup of $G$.
\end{proposition}

Let $G$ be any group, $x\in G$. One way of forming a subgroup $H$ of $G$ is by letting $H$ be the set of all integer powers of $x$. We now study groups which are generated by one element.

\begin{definition}[Cyclic group]
A group $H$ is \vocab{cyclic} if $H$ can be generated by a single element, i.e. there exists $x\in G$ such that $H=\{x^n\mid n\in\ZZ\}$; $x$ is called a \vocab{generator} of $H$.

In additive notation $H$ is cyclic if $H=\{nx\mid n\in\ZZ\}$.
\end{definition}

\begin{notation}
If $H$ is generated by $x$, we write $H=\langle x\rangle$.
\end{notation}

\begin{remark}
A cyclic group may have more than one generator. For example, if $H=\langle x\rangle$, then also $H=\langle x^{-1}\rangle$ because $(x^{-1})^n=x^{-n}\in H$ for $n\in\ZZ$ so does $-n$, so that
\[\{x^n\mid n\in\ZZ\}=\{(x^{-1})^n\mid n\in\ZZ\}.\] 
\end{remark}

\begin{example}
$\ZZ$ is a cyclic group with generators $1$ and $-1$.
\end{example}

\begin{proposition}
Cyclic groups are abelian.
\end{proposition}

\begin{proof}
By the laws of exponents, $x^i x^j=x^{i+j}=x^j x^i$.
\end{proof}

\begin{proposition}
If $H=\langle x\rangle$, then $|H|=|x|$ (where if one side of this equality is infinite, so is the other):
\begin{enumerate}[label=(\arabic*)]
\item if $|H|=n<\infty$, then $x^n=1$ and $1,x,x^2,\dots,x^{n-1}$ are all the distinct elements of $H$;
\item if $|H|=\infty$, then $x^n\neq1$ for all $n\neq0$, and $x^a\neq x^b$ for all $a,b\in\ZZ$, $a\neq b$.
\end{enumerate}
\end{proposition}

\begin{proposition}
Let $G$ be an arbitrary group, $x\in G$ and let $m,n\in\ZZ$. If $x^n=1$ and $x^m=1$, then $x^d=1$ where $d=\gcd(m,n)$. In particular, if $x^m=1$ for some $m\in\ZZ$, then $|x|$ divides $m$.
\end{proposition}

\begin{theorem}
Any two cyclic groups of the same order are isomorphic:
\begin{enumerate}[label=(\arabic*)]
\item if $n\in\ZZ^+$ and $\langle x\rangle$ and $\langle y\rangle$ are both cyclic groups of order $n$, then the map $\phi:\langle x\rangle\to\langle y\rangle$ which maps $x^k\mapsto y^k$ is well-defined and is an isomorphism.
\item if $\langle x\rangle$ is an infinite cyclic group, the map $\phi:\ZZ\to\langle x\rangle$ which maps $k\mapsto x^k$ is well-defined and is an isomorphism.
\end{enumerate}
\end{theorem}

\begin{notation}
For each $n\in\ZZ^+$, $C_n$ denotes the cyclic group of order $n$.
\end{notation}






%Permutation Groups
%Fermat--Euler theorem
from the group-theoretic point of view.

\begin{definition}[Coset]
Let $H$ be a subgroup of $G$.

Then the \vocab{left cosets} of $H$ (or left $H$-cosets) are the sets
\[ gH=\{gh\mid h\in H\}. \]
The \vocab{right cosets} of $H$ (or right $H$-cosets) are the sets
\[ Hg=\{hg\mid h\in H\}. \]
\end{definition}

Two (left) cosets $aH$ and $bH$ are either disjoint or equal. 

Since multiplication is injective, the cosets of $H$ are the same size as $H$, and thus $H$ partitions $G$ into equal-sized parts.

\begin{notation}
We write $G/H$ for the set of (left) cosets of $H$ in $G$. The cardinality of $G/H$ is called the \vocab{index} of $H$ in $G$.
\end{notation}

An important result relating the order of a group with the orders of its subgroups is Lagrange's theorem.

\begin{theorem}[Lagrange's theorem]
If $G$ is a finite group and $H$ is a subgroup of $G$, then $|H|$ divides $|G|$.
\end{theorem}

\begin{theorem}[Fermat's Little Theorem]
For every finite group $G$, for all $a \in G$, $a^{|G|}=e$.
\end{theorem}

\begin{proof}
Consider the subgroup $H$ generated by $a$: $H = \{a^i \mid i \in \ZZ\}$. Since $G$ is finite, the infinite sequence $a^0=e, a^1, a^2, a^3, \dots$ must repeat, say $a^i = a^j, i < j$. Let $k=j-i$. Multiplying both sides by $a^{-i} = (a^{-1})^i$, we get $a^{j-i} = a^k = e$. Suppose $k$ is the least positive integer for which this holds. Then $H = \{a_0, a_1, a_2, \dots, a^{k-1}\}$, and thus $|H| = k$. By Lagrange’s Theorem, $k$ divides $|G|$, so $a^{|G|} = (a^k)^\frac{|G|}{k} = e$.
\end{proof}

\begin{theorem}[Fermat--Euler Theorem (or Euler's totient theorem)]
If $a$ and $N$ are coprime, then $a^{\phi(N)}\equiv1\pmod N$, where $\phi$ is Euler's totient function.
\end{theorem}

\section{Quotient Groups and Homomorphisms}

\section{Group Actions}

\section{Direct and Semidirect Products and Abelian Groups}

\section{Further Topics}