\chapter{Vector Functions}
\section{Vector Functions and Space Curves}
A \textbf{vector-valued function} is a function whose domain is a set of real numbers and whose range is a set of vectors.

In the case of vector functions $\vb{r}$ whose values are three-dimensional vectors, we write
\[\vb{r}(t)=\langle f(t),g(t),h(t)\rangle=f(t)\vb{i}+g(t)\vb{j}+h(t)\vb{k}\]
where real-valued functions $f,g,h$ are called the \textbf{component functions} of $\vb{r}$.

\begin{remark}
We usually use the letter $t$ to denote the independent variable because it represents time in most applications of vector functions.
\end{remark}

The \textbf{limit} of a vector function $\vb{r}$ is defined by taking the limits of its component functions as follows.

\begin{definition}[Limit]
If $\vb{r}(t)=\langle f(t),g(t),h(t)\rangle$, then
\[\lim_{t\to a}\vb{r}(t)=\langle\lim_{t\to a}f(t),\lim_{t\to a}g(t),\lim_{t\to a}h(t)\rangle\]
provided the limits of the component functions exist.
\end{definition}

Equivalently, we could have used an epsilon--delta definition (see Exercise 45). Limits of vector functions obey the same rules as limits of real-valued functions (see Exercise 43).

A vector function $\vb{r}$ is \textbf{continuous} at $a$ if
\[\lim_{t\to a}\vb{r}(t)=\vb{r}(a).\]
In view of the above definition, we see that $\vb{r}$ is continuous at $a$ if and only if its component functions $f,g,h$ and are continuous at $a$.

There is a close connection between continuous vector functions and space curves. Suppose that $f,g,h$ and are continuous real-valued functions on an interval $I$. Then the set of all points in space, where
\[x=f(t)\quad y=g(t)\quad z=h(t)\]
and $t$ varies throughout the interval $I$, is called a \textbf{space curve}. The equation above is called the \textbf{parametric equations} of $C$, and $t$ is called a \textbf{parameter}.

\begin{remark}
We can think of as $C$ being
traced out by a moving particle whose position at time $t$ is $\brac{f(t),g(t),h(t)}$. If we now consider the vector function $\vb{r}(t)=\langle f(t),g(t),h(t)\rangle$, then $\vb{r}(t)$ is the position vector of the point $P\brac{f(t),g(t),h(t)}$ on $C$. Thus any continuous vector function $\vb{r}$ defines a space curve $C$ that is traced out by the tip of the moving vector $\vb{r}(t)$.
\end{remark}

\section{Derivatives and Integrals of Vector Functions}
The \textbf{derivative} $\vb{r}^\prime$ of a vector function $\vb{r}$ is defined in much the same way as for real-valued functions:
\begin{equation}
\dv{\vb{r}}{t}=\vb{r}^\prime(t)=\lim_{h\to0}\frac{\vb{r}(t+h)-\vb{r}(t)}{h}
\end{equation}
if this limit exists. The vector $\vb{r}^\prime(t)$ is also called the \textbf{tangent vector} to the curve $C$ defined by $\vb{r}$ at a point $P$, provided that $\vb{r}^\prime(t)$ exists and $\vb{r}^\prime(t)\neq\vb{0}$. The \textbf{tangent line} to $C$ at $P$ is defined to be the line through $P$ parallel to the tangent vector $\vb{r}^\prime(t)$. We will also have occasion to consider the \textbf{unit tangent vector}, which is
\[\vb{T}(t)=\frac{\vb{r}^\prime(t)}{|\vb{r}^\prime(t)|}.\]

The following theorem gives us a convenient method for computing the derivative of a vector function $\vb{r}$: just differentiate each component of $\vb{r}$.

\begin{theorem}
If $\vb{r}(t)=\langle f(t),g(t),h(t)\rangle=f(t)\vb{i}+g(t)\vb{j}+h(t)\vb{k}$, where $f,g,h$ are differentiable functions, then
\[\vb{r}^\prime(t)=\langle f^\prime(t),g^\prime(t),h^\prime(t)\rangle=f^\prime(t)\vb{i}+g^\prime(t)\vb{j}+h^\prime(t)\vb{k}.\]
\end{theorem}

\begin{proof}
\begin{align*}
\vb{r}^\prime(t)&=\lim_{\Delta t\to0}\frac{1}{\Delta t}[\vb{r}(t+\Delta t)-\vb{r}(t)]\\
&=\lim_{\Delta t\to0}\frac{1}{\Delta t}[\langle f(t+\Delta t),g(t+\Delta t),h(t+\Delta t)\rangle-\langle f(t),g(t),h(t)\rangle]\\
&=\lim_{\Delta t\to0}\left\langle\frac{f(t+\Delta t)-f(t)}{\Delta t},\frac{g(t+\Delta t)-g(t)}{\Delta t},\frac{h(t+\Delta t)-h(t)}{\Delta t}\right\rangle\\
&=\left\langle\lim_{\Delta t\to0}\frac{f(t+\Delta t)-f(t)}{\Delta t},\lim_{\Delta t\to0}\frac{g(t+\Delta t)-g(t)}{\Delta t},\lim_{\Delta t\to0}\frac{h(t+\Delta t)-h(t)}{\Delta t}\right\rangle\\
&=\langle f^\prime(t),g^\prime(t),h^\prime(t)\rangle.
\end{align*}
\end{proof}

Just as for real-valued functions, the second derivative of a vector function $\vb{r}$ is the derivative of $\vb{r}^\prime$, that is, $(\vb{r}^\prime)^\prime$.

The next theorem shows that the differentiation rules for real-valued functions have their counterparts for vector-valued functions.

\begin{theorem}[Differentiation rules]
Suppose $\vb{u}$ and $\vb{v}$ are differentiable vector functions, $c$ is a scalar, and $f$ is a real-valued function. Then
\begin{enumerate}[label=(\arabic*)]
\item (addition) $\displaystyle\dv{t}[\vb{u}(t)+\vb{v}(t)]=\vb{u}^\prime(t)+\vb{v}^\prime(t)$
\item (scalar multiplication) $\displaystyle\dv{t}[c\vb{u}(t)]=c\vb{u}^\prime(t)$
\item (product rule) $\displaystyle\dv{t}[f(t)\vb{u}(t)]=f^\prime(t)\vb{u}(t)+f(t)\vb{u}^\prime(t)$
\item (dot product rule) $\displaystyle\dv{t}[\vb{u}(t)\cdot\vb{v}(t)]=\vb{u}^\prime(t)\cdot\vb{v}(t)+\vb{u}(t)\cdot\vb{v}^\prime(t)$
\item (cross product rule) $\displaystyle\dv{t}[\vb{u}(t)\times\vb{v}(t)]=\vb{u}^\prime(t)\times\vb{v}(t)+\vb{u}(t)\times\vb{v}^\prime(t)$
\item (chain rule) $\displaystyle\dv{t}[\vb{u}(f(t))]=f^\prime(t)\vb{u}^\prime(f(t))$
\end{enumerate}
\end{theorem}

The proof of (4) follows; the remaining proofs are left as exercises.

\begin{proof}

\end{proof}

\begin{lemma}
If $|\vb{r}(t)|=c$ is a constant, then $\vb{r}^\prime(t)$ is orthogonal to $\vb{r}(t)$ for all $t$.
\end{lemma}

\begin{proof}

\end{proof}