\part{Linear Algebra}
\chapter{Linear Equations}
\section{Systems of Linear Equations}
Suppose $F$ is a field. We consider tbe problem of finding $n$ scalars (elements of $F$) $x_1,\dots,x_n$ which satisfy the conditions
\begin{equation}\label{eqn:system-linear-eqn}
\begin{split}
A_{11}x_1+A_{12}x_2+\cdots+A_{1n}x_n&=y_1\\
A_{21}x_1+A_{22}x_2+\cdots+A_{2n}x_n&=y_2\\
\vdots&=\vdots\\
A_{m1}x_1+A_{m2}x_2+\cdots+A_{mn}x_n&=y_m
\end{split}
\end{equation}
where $y_1,\dots,y_m$ and $A_{ij},1\le i\le m,1\le j\le n$ are given elements of $F$. We call \ref{eqn:system-linear-eqn} a \vocab{system of $m$ linear equations in $n$ unknowns}.

Any $n$-tuple $(x_1,\dots,x_n)$ of elements of $F$ which satisfies each of the equations in \ref{eqn:system-linear-eqn} is called a \vocab{solution} of the system.

If $y_1=\cdots=y_m=0$, we say that the system is \vocab{homogeneous}, or that each of the equations is homogeneous.

For the general system \ref{eqn:system-linear-eqn}, suppose we select $m$ scalars $c_1,\dots,c_m$, multiply the $j$-th equation by $c_j$ and then add up all the $m$ equations. We obtain
\[(c_1A_{11}+\cdots+c_mA_{m1})x_1+\cdots+(c_1A_{1n}+\cdots+c_mA_{mn})x_n=c_1y_1+\cdots+c_my_m\]
which we call a \vocab{linear combination} of the equations in \ref{eqn:system-linear-eqn}. Evidently, any soltion of the entire system of equations \ref{eqn:system-linear-eqn} will also be a solution of this new equation. This is the fundamental idea of the elimination process to find the solution(s) of a system of linear equations.

If we have another system of linear equations
\begin{equation}\label{eqn:system-linear-eqn2}
\begin{split}
B_{11}x_1+B_{12}x_2+\cdots+B_{1n}x_n&=z_1\\
\vdots&=\vdots\\
B_{k1}x_1+B_{k2}x_2+\cdots+B_{kn}x_n&=z_k
\end{split}
\end{equation}
in which each of the $k$ equations is a linear combination of the equations in \ref{eqn:system-linear-eqn}, then every solution of \ref{eqn:system-linear-eqn} is a solution of this new system. Two systems of linear equations are \vocab{equivalent} if each equation in each system is a linear combination of the equations in the other system.

\begin{theorem}
Equivalent systems of linear equations have exactly the same equations.
\end{theorem}

\section{Matrices and Elementary Row Operations}
Notive that there is no need to write the unknowns $x_1,\dots,x_n$ since one actually computes only with the coefficients $A_{ij}$ and scalars $y_i$. We abbreviate the system \ref{eqn:system-linear-eqn} as
\[AX=Y\]
where
\[A=\begin{bmatrix}
A_{11}&\cdots&A_{1n}\\
\vdots&&\vdots\\
A_{m1}&\cdots&A_{mn}
\end{bmatrix},\quad X=\begin{bmatrix}
x_1\\\vdots\\x_n
\end{bmatrix},\quad Y=\begin{bmatrix}
y_1\\\vdots\\y_m
\end{bmatrix}.\]
We call $A$ the \vocab{matrix of coefficients} of the system. The \vocab{entries} of the matrix $A$ are the scalars $A_{ij}$.

We wish now to consider operations on the rows of the matrix $A$ which correspond to forming linear combinations of the equations in the system $AX=Y$. We restrict our attention to three \vocab{elementary row operations} on an $m\times n$ matrix $A$ over the field $F$:
\begin{enumerate}
\item 
\end{enumerate}

\section{Row-Reduced Echelon Matrices}
\section{Matrix Multiplication}
\section{Invertible Matrices}