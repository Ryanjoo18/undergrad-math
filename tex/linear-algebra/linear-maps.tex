\chapter{Linear Maps}\label{chap:linear-maps}
\section{Vector Space of Linear Maps}
\begin{definition}[Linear map]
A \vocab{linear map} from $V$ to $W$ is a function $T:V\to W$ with the following properties:
\begin{enumerate}[label=(\roman*)]
\item Additivity: $T(v+w)=Tv+Tw$ for all $v,w\in V$
\item Homogeneity: $T(\lambda v)=\lambda T(v)$ for all $\lambda\in\FF$, $v\in V$
\end{enumerate}
\end{definition}

\begin{notation}
The set of all linear maps from $V$ to $W$ is denoted $\mathcal{L}(V,W)$; the set of linear transformations on $V$ is denoted $\mathcal{L}(V)$.
\end{notation}

\begin{proposition}[Linear map lemma]
Suppose $\{v_1,\dots,v_n\}$ is a basis of $V$ and $w_1,\dots,w_n\in W$. Then there exists a unique linear map $T:V\to W$ such that
\[Tv_i=w_i\quad(i=1,\dots,n)\]
\end{proposition}

\begin{proof}
First we show the existence of a linear map $T$ with the desired property. Define $T:V\to W$ by
\[T(c_1v_1+\cdots+c_nv_n)=c_1w_1+\cdots+c_nw_n,\]
for some $c_i\in\FF$. Since $\{v_1,\dots,v_n\}$ is a basis of $V$, by \cref{lemma:basis-criterion}, each $v\in V$ can be uniquely expressed as a linear combination of $v_1,\dots,v_n$, thus the equation above does indeed define a function $T:V\to W$. For $i=1,\dots,n$, take $c_i=1$ and the other $c$'s equal to $0$ to show that $Tv_i=w_i$.

We now show that $T:V\to W$ is a linear map:
\begin{enumerate}[label=(\roman*)]
\item If $u,v\in V$ with $u=a_1v_1+\cdots+a_nv_n$ and $c_1v_1+\cdots+c_nv_n$, then
\begin{align*}
T(u+v)&=T\brac{(a_1+c_1)v_1+\cdots+(a_n+c_n)v_n}\\
&=(a_1+c_1)w_1+\cdots+(a_n+c_n)w_n\\
&=(a_1w_1+\cdots+a_nw_n)+(c_1w_1+\cdots+c_nw_n)\\
&=Tu+Tv
\end{align*}
so $T$ satisfies additivity.

\item If $\lambda\in\FF$ and $v=c_1v_1+\cdots+c_nv_n$, then
\begin{align*}
T(\lambda v)&=T(\lambda c_1v_1+\cdots+\lambda c_nv_n)\\
&=\lambda c_1w_1+\cdots+\lambda c_nw_n\\
&=\lambda(c_1w_1+\cdots+c_nw_n)\\
&=\lambda Tv
\end{align*}
so $T$ satisfies homogeneity.
\end{enumerate}

To prove uniqueness, now suppose that $T\in\mathcal{L}(V,W)$ and $Tv_i=w_i$ for $i=1,\dots,n$. Let $c_i\in\FF$. The homogeneity of $T$ implies that $T(c_iv_i)=c_iw_i$. The additivity of $T$ now implies that 
\[T(c_1v_1+\cdots+c_nv_n)=c_1w_1+\cdots+c_nw_n.\]
Thus T is uniquely determined on $\spn\{v_1,\dots,v_n\}$. Since $\{v_1,\dots,v_n\}$ is a basis of $V$, this implies that $T$ is uniquely determined on $V$.
\end{proof}

\begin{proposition}
$\mathcal{L}(V,W)$ is a vector space, with the operations addition and scalar multiplication defined as follows: suppose $S,T\in\mathcal{L}(V,W)$, $\lambda\in\FF$,
\begin{enumerate}[label=(\roman*)]
\item $(S+T)(v)=Sv+Tv$
\item $(\lambda T)(v)=\lambda(Tv)$
\end{enumerate}
for all $v\in V$.
\end{proposition}

\begin{definition}[Product of linear maps]
$T\in\mathcal{L}(U,V)$, $S\in\mathcal{L}(V,W)$, then the \vocab{product} $ST\in\mathcal{L}(U,W)$ is defined by
\[(ST)(u)=S(Tu)\quad(\forall u\in U)\]
\end{definition}

\begin{remark}
In other words, $ST$ is just the usual composition $S\circ T$ of two functions.
\end{remark}

\begin{remark}
$ST$ is defined only when $T$ maps into the domain of $S$.
\end{remark}

\begin{proposition}[Algebraic properties of products of linear maps] \
\begin{enumerate}[label=(\roman*)]
\item Associativity: $(T_1T_2)T_3=T_1(T_2T_3)$ for all linear maps $T_1,T_2,T_3$ such that the products make sense (meaning that $T_3$ maps into the domain of $T_2$, $T_2$ maps into the domain of $T_1$)
\item Iidentity: $TI=IT=T$ for all $T\in\mathcal{L}(V,W)$ (the first $I$ is the identity map on $V$, and the second $I$ is the identity map on $W$)
\item Distributive: $(S_1+S_2)T=S_1T+S_2T$ and $S(T_1+T_2)=ST_1+ST_2$ for all $T,T_1,T_2\in\mathcal{L}(U,V)$ and $S,S_1,S_2\in\mathcal{L}(V,W)$
\end{enumerate}
\end{proposition}

\begin{proposition}\label{prop:linear-map-0-0}
$T\in\mathcal{L}(V,W)$. Then $T(0)=0$.
\end{proposition}

\begin{proof}
By additivity, we have
\[T(0)=T(0+0)=T(0)+T(0).\]
Add the additive inverse of $T(0)$ to each side of the equation above to conclude that $T(0)=0$.
\end{proof}

\section{Kernel and Image}
\begin{definition}[Kernel]
For $T\in\mathcal{L}(V,W)$, the \vocab{kernel} of $T$ is the subset of $V$ consisting of those vectors that $T$ maps to $0$:
\[\ker T\coloneqq\{v\in V\mid Tv=0\}.\]
\end{definition}

\begin{proposition}
$T\in\mathcal{L}(V,W)$, $\ker T$ is a subspace of $V$.
\end{proposition}

\begin{proof}
By \cref{lemma:subspace-conditions}, we check the conditions of a subspace:
\begin{enumerate}[label=(\roman*)]
\item $T(0)=0$ by \cref{prop:linear-map-0-0}, so $0\in\ker T$.
\item For all $v,w\in\ker T$, 
\[T(v+w)=Tv+Tw=0\implies v+w\in\ker T\]
so $\ker T$ is closed under addition.
\item For all $v\in\ker T$, $\lambda\in\FF$,
\[T(\lambda v)=\lambda Tv=0\implies\lambda v\in\ker T\]
so $\ker T$ is closed under scalar multiplication.
\end{enumerate}
\end{proof}

\begin{definition}[Injectivity]
$T:V\to W$ is \vocab{injective} if
\[Tu=Tv\implies u=v.\]
\end{definition}

\begin{proposition}
$T\in\mathcal{L}(V,W)$, $T$ is injective if and only if $\ker T=0$.
\end{proposition}

\begin{proof}

\end{proof}

\begin{definition}[Image]
For $T:V\to W$, the \vocab{image} of $T$ is the subset of $W$ consisting of those vectors that are of the form $Tv$ for some $v\in V$:
\[\im T\coloneqq\{Tv\mid v\in V\}.\]
\end{definition}

\begin{proposition}
$T\in\mathcal{L}(V,W)$, $\im T$ is a subspace of $W$.
\end{proposition}

\begin{proof}

\end{proof}

\begin{definition}[Surjectivity]
$T:V\to W$ is \vocab{surjective} if $\im T=W$.
\end{definition}

\begin{theorem}[Fundamental Theorem of Linear Maps]
$T\in\mathcal{L}(V,W)$, then $\im T$ is finite-dimensional and
\[\dim V=\dim\ker T+\dim\im T.\]
\end{theorem}

\section{Matrices}
\begin{definition}[Matrix]
A $m\times n$ \vocab{matrix} $A$ is a rectangular array with $m$ rows and $n$ columns:
\[A=\begin{pmatrix}
a_{11} & \cdots & a_{1n}\\
\vdots & & \vdots\\
a_{m1} & \cdots & a_{mn}
\end{pmatrix}\]
where $a_{ij}\in\FF$.
\end{definition}

\begin{definition}[Matrix of a linear map]
$T\in\mathcal{L}(V,W)$, $\{v_1,\dots,v_n\}$ is a basis of $V$, $\{w_1,\dots,w_m\}$ is a basis of $W$. The \vocab{matrix} of $T$ with respect to these bases of the $m\times n$ matrix $M(T)$ whose entries $a_{ij}$ are defined by

\end{definition}

\section{Invertibility and Isomorphism}
\begin{definition}[Invertibility]
$T\in\mathcal{L}(V,W)$ is \vocab{invertible} if there exists $S\in\mathcal{L}(W,V)$ such that $ST=I_V$, $TS=I_W$, where $I_V$ and $I_W$ are the \vocab{identity maps} on $V$ and $W$ respectively; $S$ is known as the \vocab{inverse} of $T$.
\end{definition}

\begin{proposition}[Uniqueness of inverse]
An invertible linear map has a unique inverse.
\end{proposition}

\begin{proof}
Suppose $T\in\mathcal{L}(V,W)$ is invertible. $S_1$ and $S_2$ are inverses of $T$. Then
\[S_1=S_1I=S_1(TS_2)=(S_1T)S_2=IS_2=S_2.\]
Thus $S_1=S_2$.
\end{proof}

Now that we know that the inverse is unique, we can give it a notation.

