\chapter{Eigenvalues and Eigenvectors}
\section{Invariant Subspaces}
\subsection{Eigenvalues}
\begin{definition}[Operator]
A linear map from a vector space to itself is called an \vocab{operator}\index{operator}.
\end{definition}

\begin{definition}[Invariant subspace]
Suppose $T\in\mathcal{L}(v)$. $U\le V$ is called \vocab{invariant}\index{invariant subspace} under $T$ if $Tu\in U$ for all $u\in U$.
\end{definition}

\begin{definition}[Eigenvalue]
Suppose $T\in\mathcal{L}(V)$. $\lambda\in\FF$ is called an \vocab{eigenvalue}\index{eigenvalue} of $T$ if there exists $v\in V$, $v\neq\vb{0}$ such that $Tv=\lambda v$.
\end{definition}

\begin{lemma}[Equivalent conditions to be an eigenvalue]
Suppose $V$ is finite-dimensional, $T\in\mathcal{L}(V)$, $\lambda\in\FF$. Then the following are equivalent:
\begin{enumerate}[label=(\arabic*)]
\item $\lambda$ is an eigenvalue of $T$.
\item $T-\lambda I$ is not injective.
\item $T-\lambda I$ is not surjective.
\item $T-\lambda I$ is not invertible.
\end{enumerate}
\end{lemma}

\begin{definition}[Eigenvector]
Suppose $T\in\mathcal{L}(V)$, $\lambda\in\FF$ is an eigenvalue of $T$. A vector $v\in V$, $v\neq\vb{0}$ is called an \vocab{eigenvector}\index{eigenvector} of $T$ corresponding to $\lambda$ if $Tv=\lambda v$.
\end{definition}

\begin{proposition}[Linearly independent eigenvectors]
Suppose $T\in\mathcal{L}(V)$. Then every list of eigenvectors of $T$ corresponding to distinct eigenvalues of $T$ is linearly independent.
\end{proposition}

\begin{proposition}
Suppose $V$ is finite-dimensional. Then each operator on $V$ has at most $\dim V$ distinct eigenvalues.
\end{proposition}

\subsection{Polynomials Applied to Operators}
\begin{notation}
Suppose $T\in\mathcal{L}(V)$, $n\in\ZZ^+$. $T^n\in\mathcal{L}(V)$ is defined by $T^n=\underbrace{T\cdots T}_\text{$m$ times}$. $T^0$ is defined to be the identity operator $I$ on $V$. If $T$ is invertible with inverse $T^{-1}$, then $T^{-n}\in\mathcal{L}(V)$ is defined by $T^{-n}=\brac{T^{-1}}^n$.
\end{notation}



\section{The Minimal Polynomial}
\section{Upper-Triangular Matrices}
\section{Diagonalisable Operators}
\section{Commuting Operators}