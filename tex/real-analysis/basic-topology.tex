\chapter{Basic Topology}\label{chap:basic-topology}
\section{Metric Space}
\begin{definition}
A set $X$, whose elements we shall call \vocab{points}, is a \vocab{metric space} if for any two points $p,q\in X$ there is associated a real value function (called distance function or \vocab{metric}) $d:X\times X\to\RR$ which satisfies the following properties:
\begin{enumerate}[label=(\roman*)]
\item (\textbf{positive definitiveness}) $d(p,q)\ge0$, where equality holds if and only if $x=y$;
\item (\textbf{symmetry}) $d(p,q)=d(q,p)$;
\item (\textbf{triangle inequality}) $d(p,q)\le d(p,r)+d(r,q)$ for any $r\in X$.
\end{enumerate}
\end{definition}

\begin{example}\label{exmp:r-metrics}
Take $X=\RR^n$. Then each of the following functions define
metrics on $X$.
\begin{align*}
d_1(x,y)&=\sum_{i=1}^{n}|x_i-y_i|;\\
d_2(x,y)&=\sqrt{\sum_{i=1}^{n}(x_i-y_i)}\\
d_\infty(x,y)&=\max_{i\in\{1,2,\dots,n\}}|x_i-y_i|.
\end{align*}
These are called the $\ell^1$-(``ell one''), $\ell^2$- (or Euclidean) and $\ell^\infty$-distances respectively. Of course, the Euclidean distance is the most familiar one.
\end{example}

The proof that each of $d_1$, $d_2$, $d_\infty$ is a metric is mostly very routine, with the exception of proving that $d_2$, the Euclidean distance, satisfies the triangle inequality. To establish this, recall that the Euclidean norm $\norm{x}_2$ of a vector $x=(x_1,\dots,x_n)\in\RR^n$ is
\[\norm{x}_2\coloneqq\brac{\sum_{i=1}^n{x_i}^2}^\frac{1}{2}=\langle x,x\rangle^\frac{1}{2},\]
where the inner product is given by
\[\langle x,y\rangle\coloneqq\sum_{i=1}^{n}x_i y_i.\]
Then $d_2(x,y)=\norm{x-y}_2$, and so the triangle inequality is the statement that
\[\norm{w-y}_2\le\norm{w-x}_2+\norm{x-y}_2.\]
This follows immediately by taking $u=w-x$ and $v=x-y$ in the following lemma.

\begin{lemma}
If $u,v\in\RR^n$ then $\norm{u+v}_2\le\norm{u}_2+\norm{v}_2$.
\end{lemma}

\begin{proof}
Since $\norm{u}_2\ge0$ for all $u\in\RR^n$, the desired inequality is equivalent to
\[\norm{u+v}_2^2\le\norm{u}_2^2+2\norm{u}_2\norm{v}_2+\norm{v}_2^2.\]
But since $\norm{u+v}_2^2=\langle u+v,u+v\rangle=\norm{u}_2^2+2\langle u,v\rangle+\norm{v}_2^2$, this inequality is immediate from the Cauchy--Schwarz inequality, that is to say the inequality $|\langle u,v\rangle|\le\norm{u}_2\norm{v}_2$.
\end{proof}

\begin{example}[Discrete metric]
Let $X$ be an arbitrary set. The \textbf{discrete metric} on a set $X$ is defined as follows:
\[d(x,y)=\begin{cases}
1&\text{if }x\neq y,\\
0&\text{if }x=y.
\end{cases}\]
\end{example}

The axioms for a metric are easy to check.

Now we turn to some metrics which come up very naturally in diverse areas of mathematics. Our first example is critical in number theory, and also serves to show that metrics need not conform to one's most na\"{i}ve understand of ``distance''.

\begin{example}[2-adic metric]
Let $X=\ZZ$, and define $d(x,y)$ to be $2^{-m}$, where $2^m$ is the largest power of two dividing $x-y$. The triangle inequality holds in the following stronger form, known as the ultrametric property:
\[d(x,z)\le\max\{d(x,y),d(y,z)\}.\]
Indeed, this is just a rephrasing of the statement that if $2^m$ divides both $x-y$ and $y-z$, then $2^m$ divides $x-z$.

This metric is very unlike the usual distance. For example, $d(999,1000) = 1$, whilst $d(0,1000)=\frac{1}{8}$.

The role of $2$ can be replaced by any other prime $p$, and the metric may also be
extended in a natural way to the rationals $\QQ$.
\end{example}

Metrics are also ubiquitous in graph theory:

\begin{example}[Path metric]
Let $G$ be a graph, that is to say a finite set of vertices $V$ joined by edges. Suppose that $G$ is connected, that is to say that there is a path joining any pair of distinct vertices. Define a distance $d$ as follows: $d(v,v)=0$, and $d(v,w)$ is the length of the shortest path from $v$ to $w$. Then $d$ is a metric on $V$, as can be easily checked.
\end{example}

They also come up in group theory:

\begin{example}[Word metric]
Let $G$ be a group, and suppose that it is generated by elements $a$, $b$ and their inverses. Define a distance on $G$ as follows: $d(v,w)$ is the minimal $k$ such that $v=wg_1\cdots g_k$, where $g_i\in\{a,b,a^{-1},b^{-1}\}$ for all $i$.

When $G$ is finite, the word metric is a special case of the path metric -- you may
wish to think about why.
\end{example}

There are many metrics with a prominent position in computer science, for instance:

\begin{example}[Hamming distance]
Let $X=\{0,1\}^n$ (the boolean cube), the set of all strings of $n$ zeroes and ones. Define $d(x,y)$ to be the number of coordinates
in which $x$ and $y$ differ.
\end{example}

It hardly need be said that metrics are ubiquitous in geometry.

\begin{example}[Projective space]
Consider the set $\PP(\RR^n)$ of one-dimensional subspaces of $\RR^n$, that is to say lines through the origin. One way to define a distance on this set is to take, for lines $L_1$, $L_2$, the distance between $L_1$ and $L_2$ to be
\[d(L_1,L_2)=\sqrt{1-\frac{|\langle v,w\rangle|^2}{\norm{v}^2\norm{w}^2}}\]
where $v$ and $w$ are any non-zero vectors in $L_1$ and $L_2$ respectively. It is easy to see this is independent of the choice of vectors $v$ and $w$. The Cauchy--Schwarz inequality ensures that $d$ is well-defined, and moreover the criterion for equality in that inequality ensures positivity. The symmetry property is evident, while the triangle inequality is left as an exercise.

It is useful to think of the case when $n=2$ here, that is, the case of lines through the origin in the plane $\RR^2$. The distance between two such lines given by the above formula is then $\sin\theta$ where $\theta$ is the angle between the two lines (another exercise).
\end{example}

\subsection{Norms}
\begin{definition}[Norms]
Let $V$ be any vector space (over the reals). A function $\norm{\cdot}:V\to[0,\infty)$ is called a norm if the following are all true:
\begin{enumerate}[label=(\arabic*)]
\item $\norm{x}=0$ if and only if $x=0$;
\item $\norm{\lambda x}=|\lambda|\norm{x}$ for all $\lambda\in\RR$, $x\in V$;
\item $\norm{x+y}\le\norm{x}+\norm{y}$ for all $x,y\in V$.
\end{enumerate}
\end{definition}

Given a norm, it is very easy to check that $d(x,y)\coloneqq\norm{x-y}$ defines a metric on $V$. Indeed, we have already seen that when $V=\RR^n$, $\norm{\cdot}_2$ is a norm (and so the name ``Euclidean norm'' is appropriate) and we defined $d_2(x,y)=\norm{x-y}_2$.

As we mentioned, the other metrics in Example \ref{exmp:r-metrics} also come from norms. Indeed, $d_1$ comes from the $\ell^1$-norm
\[\norm{x}_1\coloneqq\sum_{i=1}^n|x_i|,\]
whilst $d_\infty$ comes from the $\ell^\infty$-norm
\[\norm{x}_\infty\coloneqq\brac{\sum_{i=1}^n|x_i|p}^\frac{1}{p}.\]


The principle of turning norms into metrics is important enough that we state it as a lemma in its own right

\begin{lemma}
Let $V$ be a vector space over the reals, and let $\norm{\cdot}$ be a norm on it. Define $d:V\times V\to[0,\infty)$ by $d(x,y)\coloneqq\norm{x-y}$. Then $(V,d)$ is a metric space
\end{lemma}

\begin{remark}
The converse is very far from true. For instance, the discrete metric does not arise from a norm.
\end{remark}

All metrics arising from a norm
have the \vocab{translation invariance property} $d(x+z,y+z)=d(x,y)$, as well as the \vocab{scalar invariance} $d(\lambda x,\lambda y)=|\lambda|d(x,y)$, neither of which are properties of arbitrary metrics. Conversely one can show that a metric with these two additional properties does come from a norm, an exercise we leave to the reader (Hint: the norm must arise as $\norm{v}=d(v,0)$)


We call a vector space endowed with a norm $\norm{\cdot}$ a \vocab{normed space}. Whenever we talk about normed spaces it is understood that we are also thinking of them as metric spaces, with the metric being defined by $d(v,w)=\norm{v-w}$.

Note that we do not assume that the underlying vector space $V$ is finitedimensional. Here are some examples which are not finite-dimensional (whilst we do not prove that they are not finite-dimensional here, it is not hard to do so and we suggest this as an exercise).

\begin{example}[$\ell^p$ spaces]
Let
\begin{align*}
\ell_1&=\crbrac{(x_n)_{n=1}^\infty\mid\sum_{n\ge1}|x_n|<\infty},\\
\ell_2&=\crbrac{(x_n)_{n=1}^\infty\mid\sum_{n\ge1}x_n^2<\infty},\\
\ell_\infty=&\crbrac{(x_n)_{n=1}^\infty\mid\sup_{n\in\NN}|x_n|<\infty}.
\end{align*}
The sets $\ell_1$, $\ell_2$, $\ell_\infty$ are all real vector spaces, and moreover $\norm{(x_n)}_1=\sum_{n\ge1}|x_n|$, $\norm{(x_n)}_2=\brac{\sum_{n\ge1}x_n^2}^\frac{1}{2}$, $\norm{(x_n)}_\infty=\sup_{n\in\NN}|x_n|$ define norms on $\ell_1$, $\ell_2$ and $\ell_\infty$ respectively. Note that $\ell_2$ is in fact an inner product space where
\[\langle(x_n),(y_n)\rangle=\sum_{n\ge1}x_ny_n,\]
(the fact that the right-hand side converges if $(x_n)$ and $(y_n)$ are in $\ell_2$ follows from the Cauchy--Schwarz inequality).

The space $\ell^2$ is known as \vocab{Hilbert space} and it is of great importance in mathematics.
\end{example}

\subsection{New metric spaces from old one}
A metric space $(X,d)$ naturally induces a metric on any of its subsets.

\begin{definition}[Subspace]
Suppose that $(X, d)$ is a metric space and let $Y$ be a subset of $X$. Then the restriction of $d$ to $Y\times Y$ gives $Y$ a metric so that $(Y,d_{Y\times Y})$ is a metric space. We call $Y$ equipped with this metric a \vocab{subspace}.
\end{definition}

\begin{example}
If $X=\RR$, we could take $Y=[0,1]$, for instance, or $Y=\QQ$, or $Y=\ZZ$.
\end{example}

\begin{definition}[Product space]
If $(X,d_X)$ and $(Y,d_Y)$ are metric spaces, then it is natural to try to make $X\times Y$ into a metric space. One method is as follows: if $x_1,x_2\in X$ and $y_1,y_2\in Y$ then we set
\[d_{X\times Y}\brac{(x_1,y_1),(x_2,y_2)}=\sqrt{d_X(x_1,x_2)^2+d_Y(y_1,y_2)^2.}\]
\end{definition}

The use of the square mean on the right, rather than the max or the sum, is appealing since then the product $\RR\times\RR$ becomes the space $\RR^2$ with the Euclidean metric. However, either of those alternative definitions results in a metric which is equivalent.

\begin{proposition}
With notation as above, $d_{X\times Y}$ gives a metric on $X\times Y$.
\end{proposition}

\begin{proof}
Reflexivity and symmetry are obvious. Less clear is the triangle inequality. We need to prove that
\begin{equation*}\tag{1}
\begin{split}
&\sqrt{d_X(x_1,x_3)^2+d_Y(y_1,y_3)^2}+\sqrt{d_X(x_3,x_2)^2+d_Y(y_3,y_2)^2}\\
&\ge\sqrt{d_X(x_1,x_2)^2+d_Y(y_1,y_2)^2}
\end{split}
\end{equation*}
Write $a_1=d_X(x_2,x_3)$, $a_2=d_X(x_1,x_3)$, $a_3=d_X(x_1,x_2)$ and similarly $b_1=d_Y(y_2,y_3)$, $b_2=d_Y(y_1,y_3)$ and $b_3=d_Y(y_1,y_2)$. Thus we want to show
\begin{equation*}\tag{2}
\sqrt{a_2^2+b_2^2}+\sqrt{a_1^2+b_1^2}\ge\sqrt{a_3^2+b_3^2}.
\end{equation*}
To prove this, note that from the triangle inequality we have $a_1+a_2>a_3$, $b_1+b_2>b_3$. Squaring and adding gives
\[a_1^2+b_1^2+a_2^2+b_2^2+2(a_1a_2+b_1b_2)\ge a_3^2+b_3^2.\]
By Cauchy--Schwarz,
\[a_1a_2+b_1b_2\le\sqrt{a_1^2+b_1^2}\sqrt{a_2^2+b_2^2}.\]
Substituting this into the previous line gives precisely the square of (2), and (1) follows.
\end{proof}

\subsection{Balls and boundedness}
\begin{definition}[Balls]
Let $X$ be a metric space. IF $x\in X$ and $r>0$, we define the \vocab{open ball} centred at $x$ with radius $r$ to be the set
\[B_r(x)\coloneqq\{y\in X\mid d(x,y)<r\}.\]
Similarly we define the \vocab{closed ball} centred at $x$ with radius $r$ to be the set
\[\bar{B}_r(x)\coloneqq\{y\in X\mid d(x,y)\le r\}.\]
The \vocab{punctured ball} is defined as
\[B_r(x)\setminus\{x\}=\{y\in X\mid 0<d(x,y)<r\}.\]
\end{definition}

\begin{definition}[Bounded]
Let $X$ be a metric space, and let $Y\subseteq X$. Then we say that $Y$ is \vocab{bounded} if $Y$ is contained in some open ball.
\end{definition}

\begin{proposition}
Let $X$ be a metric space and let $Y\subseteq X$. Then the following are equivalent:\
\begin{enumerate}[label=(\roman*)]
\item $Y$ is bounded;
\item $Y$ is contained in some closed ball;
\item The set $\{d(y_1,y_2)\mid y_1,y_2\in Y\}$ is a bounded subset of $\RR$.
\end{enumerate}
\end{proposition}

\begin{proof}
That (i) implies (ii) is totally obvious. That (ii) implies (iii) follows immediately from the triangle inequality. Finally, suppose $Y$ satisfies (iii). Then there is some $K$ such that $d(y_1,y_2)\le K$ whenever $y_1,y_2\in Y$. If $Y$ is empty, it is certainly bounded. Otherwise, let $a\in Y$ be an arbitrary point. Then Y is contained in $B_r(a)$ where $r=K+1$.
\end{proof}

\begin{example}
An open (closed) ball in $\RR$ is equivalent to a finite open (closed) interval, i.e. $(a,b)$ ($[a,b]$), $a,b\in\RR$.
\end{example}

\begin{definition}[Neighbourhood]
A set $N\subset X$ is called a \vocab{neighbourhood} of $x\in X$ if $\exists r>0\suchthat B_r(x)\subset N$.
\end{definition}

\subsection{Open and closed sets}
\begin{definition}[Open set]
If $X$ is a metric space, we say $E\subseteq X$ is \vocab{open} (in $X$) if it is a neighbourhood of each of its elements, i.e.,  $\forall x\in E\exists r>0\suchthat B_r(x)\subset E$.
\end{definition}

\begin{proposition}
Any open ball is open.
\end{proposition}

\begin{proof}
Assume $B_r(x)$ is an open ball in a metric space $(X,d)$. Then for any point $y\in B_r(x)$, there is
\[ d(y,x)<r. \]
Now we define $r^\prime\coloneqq r-d(y,x)$, which is positive.

Consider the ball $B_{r^\prime}(y)$. We shall show it lives in $B_r(x)$. For this, take any point $z\in B_{r^\prime}(y)$. Using the triangle inequality of a metric, we have
\begin{align*}
d(z,x)&\le d(z,y)+d(y,x)\\
&<r^\prime+d(y,x)\\
&=r.
\end{align*}
Hence $z\in B_r(x)$, and $B_{r^\prime}(y)\subset B_r(x)$.
\end{proof}

\begin{proposition}
\begin{enumerate}[label=(\arabic*)]
\item Both $\emptyset$ and $X$ are open.
\item If $E_1$, $E_2$ are open, then $E_1\cap E_2$ is open.
\item If $E_i$ is open for $i\in I$, then $\bigcup_{i\in I}E_i$ is open.
\end{enumerate}
\end{proposition}

An arbitrary union of open sets is open; a finite intersection of open sets is open.

\begin{proof} \
\begin{enumerate}[label=(\arabic*)]
\item Obvious by definition.
\item Take a point $x\in E_1\cap E_2$, we need to find an open ball with radius $r>0$ such that $x\in B_r(x)\subset E_1\cap E_2$.

To find such $r>0$, notice that since both $E_1$ and $E_2$ are open, there are open balls
\begin{align*}
&x\in B_{r_1}(x)\subset E_1\\
&x\in B_{r_2}(x)\subset E_2
\end{align*}
Take $r\coloneqq\min\{r_1,r_2\}$. Then $B_r(x)\subset B_{r_1}(x)\subset E_1$ and $B_r(x)\subset B_{r_2}(x)\subset E_2$, and hence $B_r(x)\subset E_1\cap E_2$.

\item Take a point $x\in\bigcup_{i\in I}E_\alpha$, then we can assume $x$ lives in some $E_k$, $k\in I$. Since $E_k$ is open, take an open ball 
\[B_r(x)\subset E_k.\]
It follows
\[B_r(x)\subset E_k\subset\bigcup_{i\in I}E_i.\]
Hence $\bigcup_{i\in I}E_i$ is open.
\end{enumerate}
\end{proof}

\begin{example}
We know $I_n\coloneqq\brac{-\frac{1}{n},\frac{1}{n}}\subset\RR$ is open for any $n\in\ZZ^+$. However, $\bigcap_{n\in\ZZ^+}I_n=\{0\}$ is not open.
\end{example}

The complement of an open set is a closed set.

\begin{definition}[Closed set]
$E$ is \vocab{closed} if its complement $E^c$ is open.
\end{definition}

\begin{example}
The closed interval $[a,b]$, $a\le b$ is closed in $\RR$.
\end{example}

\begin{proposition}
Any closed ball is closed.
\end{proposition}

\begin{proof}
To prove that $\bar{B}_r(x)=\{y\in X\mid d(x,y)\le r\}$ is closed, we need to show that its complement $\bar{B}_r(x)^c=\{y\in X\mid d(x,y)>r\}$ is open.

Let $z\in\bar{B}_r(x)^c$. Choose $r^\prime>0$ such that $r+r^\prime<d(x,z)$; that is, $r^\prime<d(x,z)-r$.

We claim that $B_{r^\prime}\subseteq\bar{B}_r(x)^c$. Pick $y\in B_{r^\prime}(z)$. Then $d(y,z)<r^\prime$. But $r+d(y,z)<d(x,z)$ so $r<d(x,z)-d(y,z)\le d(x,y)$ by triangle inequality. Hence we have $r<d(x,y)$, thus $y\in\bar{B}_r(x)^c$. Therefore $\bar{B}_r(x)^c$ is open, so $\bar{B}_r(x)$ is closed.
\end{proof}

\begin{proposition}
\begin{enumerate}[label=(\arabic*)]
\item Both $\emptyset$ and $X$ are closed.
\item If $E_1$ and $E_2$ are closed, then $E_1\cup E_2$ is closed.
\item If $E_i$ is closed for $i\in I$, then $\bigcap_{i\in I}E_i$ is closed.
\end{enumerate}
\end{proposition}

An arbitrary intersection of closed sets is closed; a finite union of closed sets is closed.

\begin{proof} \
\begin{enumerate}[label=(\arabic*)]
\item It follows immediately from $\emptyset=X^c$ and $X=\emptyset^c$.
\item It follows from above that
\[ (E_1\cup E_2)^c={E_1}^c\cap {E_2}^c \]
is open (de Morgan's law applied), and hence $E_1\cup E_2$ is closed.
\item It follows from above that
\[ \brac{\bigcap_{i\in I}E_i}^c=\bigcup_{i\in I}{E_i}^c \]
is open (de Morgan's law applied), and hence $\bigcap_{i\in I}E_i$ is closed.
\end{enumerate}
\end{proof}

\begin{example}
Consider a sequence of closed sets $\sqbrac{-1+\frac{1}{n},1-\frac{1}{n}}$, $n\in\ZZ^+$, of $\RR$. Take their union
\[\bigcup_{n\in\ZZ^+}\sqbrac{-1+\frac{1}{n},1-\frac{1}{n}}=(-1,1)\]
which is open, not closed.
\end{example}

\begin{definition}[Limit point]
$p$ is a \vocab{limit point} of $E$ if every neighborhood of $p$ contains $q\neq p$ such that $q\in E$:
\[ \forall r>0,\exists q\in E, q\neq p\suchthat q\in B_r(p). \]
The \vocab{induced set} of $E$, denoted by $E^\prime$, is the set of all limit points of $E$ in $X$.

The \vocab{closure} of $E$, denoted by $\bar{E}$, is the union set $E\cup E^\prime$.
\end{definition}

\begin{example} \
\begin{itemize}
\item Consider the metric space $\RR$, $a$ and $b$ are limit points $(a,b]$. The limit point set of $(a,b]$ is $[a,b]$, which is also the closure $(a,b]$.
\item Consider the metric space $\RR^2$. The limit point set of any open ball $B_r(x)$ is the closed ball $\bar{B}_r(x)$, which is also the closure of $B_r(x)$.
\item Consider $\QQ\subset\RR$. $\QQ^\prime=\bar{\QQ}=\RR$.
\end{itemize}
\end{example}

\begin{proposition}
If $p$ is a limit point of $E$, then every neighbourhood of $p$ contains infinitely many points of $E$.
\end{proposition}

\begin{proof}
Prove by contradiction. Suppose there is a neighborhood $B_r(p)$ which contains only a finite number of points of $E$: $q_1,\dots,q_n$, which are distinct from $p$. Define
\[ r=\min_{1\le m\le n} d(p,q_m). \]
The minimum of a finite set of positive numbers is clearly positive, so that $r>0$.

The neighborhood $B_r(p)$ contains no point $q\in E,q\neq p$ so that $p$ is not a limit point of $E$, a contradiction.
\end{proof}

\begin{corollary}
A finite point set has no limit points.
\end{corollary}

\begin{definition}
$E$ is called \vocab{dense} if $\bar{E}=X$.
\end{definition}

\begin{proposition}
\begin{enumerate}[label=(\arabic*)]
\item $A$ is a dense set in $X$ if and only if $A$ intersects with all open sets in $X$.
\item If $A$ is dense in $X$ and $B$ is dense in $A$, then $B$ is dense in $X$.
\item If $A$ and $B$ are dense in $X$ where $A$ is open, then $A\cap B$ is dense in $X$.
\end{enumerate}
\end{proposition}

%%%%%%%%%%%%%%%%%%%%%%%%%%%%%%%%%%%%%%%%%%



\begin{definition}
\begin{enumerate}[label=(\arabic*)]
\item $p$ is a \vocab{limit point} of $E$ if every neighborhood of $p$ contains $q\neq p$ such that $q\in E$:
\[ \forall r>0,\exists q\in E, q\neq p\suchthat q\in B_r(p). \]
The \vocab{induced set} of $E$, denoted by $E^\prime$, is the set of all limit points of $E$ in $X$.
\item $p$ is an \vocab{isolated point} of $E$ if it not a limit point of $E$.
\item $E$ is \vocab{closed} if every limit point of $E$ is a point of $E$, i.e $\bar{E}=E$.

The \vocab{closure} of $E$, denoted by $\bar{E}$, is the union set $E\cup E^\prime$.

\item $p$ is an \vocab{interior point} of $E$ if there is a neighborhood $N$ of $p$ such that $N\subset E$:
\[ \exists r>0 \suchthat B_r(p)\subset E. \]
The \vocab{interior} of $E$, denoted by $E^\circ$, is the set of all interior points in $E$:
\[E^\circ\coloneqq\{p\in X\mid \exists r>0 \suchthat B_r(p)\subset E\}\]
A point $x$ is an \vocab{exterior point} of $A$ if it is an interior point of $A^c$.
\item $E$ is \vocab{open} if every point of $E$ is an interior point of $E$, i.e. $E^\circ=E$.
\item $E$ is \vocab{perfect} if $E$ is closed and if every point of E is a limit point of $E$.
\item The \vocab{boundary} of $E$, denoted by $\partial E$, is the set difference $\bar{E}\setminus E^\circ$.

$p$ is a \vocab{boundary point} of $E$ if $p\in\partial E$.

$E$ is compact if it is a bounded closed set.
\item $E$ is \vocab{dense} in $X$ if every point of $X$ is a limit point of $E$, or a point of $E$ (or both). 

A subset $B\subset A$ is a dense subset of $A$ if $\bar{B}=A$.

$E$ is \vocab{nowhere dense} its closure has no interior, i.e. $(\bar{E})^\circ=\emptyset$.
\end{enumerate}
\end{definition}




%%%%%%%%%%%%%%%%%%%%%%%%%%%%

\begin{proposition}
\begin{enumerate}[label=(\arabic*)]
\item $\bar{E}$ is closed;
\item $E=\bar{E}$ if and only if $E$ is closed;
\item $\bar{E}\subset F$ for every closed set $F\subset X$ such that $E\subset F$.
\end{enumerate}
By (1) and (3), $\bar{E}$ is the \vocab{smallest} closed subset of $X$ that contains $E$.
\end{proposition}

\begin{proof} \
\begin{enumerate}[label=(\arabic*)]
\item 
\item 
\item 
\end{enumerate}
\end{proof}

\begin{proposition}
\begin{enumerate}[label=(\arabic*)]
\item $E^\circ$ is open.
\item $E$ is open if and only if $E=E^\circ$.
\item If $G\subset E$ and $G$ is open, then $G\subset E^\circ$.
\end{enumerate}
\end{proposition}

\begin{proof} \
\begin{enumerate}[label=(\arabic*)]
\item If $p\in E^\circ$ then $B_r(p)\subset E$ for some $r>0$ and if $q\in B_r(p)$ then by triangle inequality, $B_{r-d(p, q)}(q)\subset E$ so $B_r(p)\subset E^\circ$ and
hence $E^\circ$ is open.

\item Certainly if $E$ is open then $E=E^\circ$ since for each $p\in E$ there exists $r>0$ such that $B_r(p)\subset E$.

Conversely if $E^\circ=E$ then this holds for each $p\in E$ so $E$ is open.

\item If $G\subset E$ is open then for each $p\in G$ there exists $r>0$ such that
$B_r(p)\subset G$, hence $B_r(p)\subset E$ so $p\in E^\circ$ and it follows that $G\subset E^\circ$.
\end{enumerate}
\end{proof}

\begin{proposition}
The set of exterior points, $(A^c)^\circ$ is the same as $(\bar{A})^c$.
\end{proposition}

\begin{proof}
\begin{align*}
x \in (A^c)^\circ 
&\iff \exists \epsilon>0 \text{ such that } B(x,\epsilon) \subset A^c \\
&\iff B(x,\epsilon) \cap A = \emptyset \\
&\iff x \notin A \text{ and } B_0(x,\epsilon) \cap A=\emptyset \\
&\iff x \notin A \cup A^\prime = \bar A \\
&\iff x \in (\bar A^c)
\end{align*}
\end{proof}

\begin{proposition}
\begin{enumerate}[label=(\arabic*)]
\item $A^\prime$ is closed.
\item $\bar{A}$ is closed, i.e. $\bar{\bar{A}}=\bar{A}$
\end{enumerate}
\end{proposition}

\begin{proof} \
\begin{enumerate}[label=(\arabic*)]
\item In order to show that $A^\prime$ is closed, we need to show that if $x$ is a limit point of $A^\prime$, then $x\in A^\prime$, i.e. $x$ is a limit point of $A$.

So we need to show that limit points of $A^\prime$ are always limit points of $A$: 
Let $x$ be a limit point of $A^\prime$, then for all $\epsilon>0$, $B_0(x,\epsilon/2)$ intersects with $A^\prime$ and we may pick $y \in B_0(x,\epsilon/2)\cap A^\prime$

Now here's the tricky part
Since $y \in A^\prime$, y is a limit point of $A$, hence $B_0(y,|y-x|)$ intersects with $A$ and thus we may pick $z \in B_0(y,|y-x|)\cap A$.

We show that $z \in B_0(x,\epsilon)$:
\[ |z-x|\le|z-y|+|y-x|<2|y-x|<\epsilon, \]
hence $z \in B(x,\epsilon)$.
\[ |z-y|<|x-y|, \]
hence $z \neq x$

$\therefore\:z \in B_0(x,\epsilon)$

\item 
\end{enumerate}
\end{proof}


%%%%%%%%%%%%%%%%%%%%%%%%

\begin{theorem}[Cantor's Intersection Theorem]
Given a decreasing sequence of compact sets $A_1\supset A_2 \supset \cdots$, there exists a point $x\in\RR^n$ such that $x$ belongs to all $A_i$. In other words, $\bigcap_{i=1}^\infty A_i\neq\emptyset$. Moreover, if for all $i\in\NN$ we have $\diam A_{i+1}\le c\cdot\diam A_k$ for some constant $c<1$, then such a point must be unique, i.e. $\bigcap_{i=1}^\infty A_k=\{x\}$ for some $x\in\RR^n$.
\end{theorem}

\begin{theorem}[Heine--Borel Theorem]
A set $A\subset\RR^n$ is compact if and only if every open covering has a finite subcover, i.e. for any family of open sets $\mathscr{U}=\{U_i\}_{i\in I}$ satisfying $A\subset\bigcup_{i\in I}U_i$, there exists $\{U_1,\dots,U_n\}\subset\mathscr{U}$ such that $A\subset\bigcup_{i=1}^n U_i$.
\end{theorem}

\begin{theorem}[Bolzano--Weierstrass Theorem]
Infinite bounded sets in $\RR^n$ must contain limit points.
\end{theorem}

We will follow a very specific sequence of steps to prove these three theorems:
\begin{enumerate}[label=(\alph*)]
\item Cantor Intersection for $n=1$
\item Bolzano--Weierstrass for $n=1$
\item Bolzano--Weierstrass for general $n$
\item Cantor Intersection for general $n$
\item Heine--Borel for general $n$
\end{enumerate}

\begin{proof} \
\begin{enumerate}[label=(\alph*)]
\item Suppose that there is a decreasing sequence of compact sets $A_1, A_2, \dots$ in the real numbers

Since $A_k$ are bounded, we may let $a_k=\inf A_k$
Also since $A_k$ are closed, $a_k \in A_k$

Note that since $A_k$ is a decreasing sequence of sets we have $a_1\le a_2\le\dots$

Also, whenever we have $n>k$, we have $a_n \in A_n$, but $A_n \subset A_k$ and thus $a_n \in A_k$.

Let $b_1=\sup A_1$, then $a_k \in A_1$ and thus $a_k\le b_1$ for all $k$.

This tells us that the sequence $\{a_k\}$ is bounded above, and thus we may let $a=\sup a_k$.

Our goal is to show that the number $a$ appears in all $A_k$, thus showing that the entire intersection $\bigcap A_k$ contains $a$ and thus must be non-empty.

Now we split this in two cases, which asks whether a is simply made from isolated points, or if it is actually some nontrivial point obtained from the boundaries of $A_k$

\textbf{Case 1:} $a_k=a$ for some $k$
In this case we see that $a_k\le a_n\le a$ for all $n>k$ and thus $a_n=a$ in this case, therefore a is an element in $A_n$ for all $n$

In this case you can imagine that there is a possibility where a is an isolated minimum point of $A_n$ which stays there forever in the decreasing sequence of sets

\textbf{Case 2:} $a_k<a$ for all $k$; in this case we see that $a$ is the limit point of the increasing sequence $\{a_k\}$

Exercise 1: Show that $a$ is a limit point of each $A_k$.

Note that $a_n$ is in $A_k$ for each $n>k$, and since $a=\sup\{a_k\}$ where $a_k$ is increasing, we can actually show that a is a limit point of $\{a_n \mid n \le k\}$:
For every $\epsilon>0$, we pick $n_0$ such that $0 < a-a_{n_0} < \epsilon$
Pick $n\prime > \max\{k,n_0\}$, then $a_{n^\prime} \ge a_{n_0}$ and so
\[ 0<a-a_n\prime \le a_{n_0} < \epsilon \]
This shows that there exists $a_n^\prime$ in $B_0(a,\epsilon) \cap \{a_n \mid n>k\}$ for all $\epsilon$, and so $a$ is a limit point of $\{a_n \mid n>k\}$.

Now since $\{a_n|n \ge k\}$ is a subset of $A_k$ we also see that a is a limit point of $A_k$
Finally, since $A_k$ is closed, we conclude that $a$ is in $A_k$ for all $k$, and we are done

Wait hold on, I forgot about the second part

Now we consider a decreasing sequence of compact sets $A_1, A_2, \dots$ such that $\diam A_{k+1} \le c \diam A_k$ for $c<1$.

Suppose otherwise that there exists $x, y$ in $\bigcap A_k$

You can imagine that this will form a fixed distance between two points, and thus there is a constant positive lower bound for the diameters:
\[ \diam A_k \ge |x-y| > 0 \forall k \]

But this cannot be true because $\diam A_{k+1} \le c \diam A_k$ and so the diameter is controlled by a decreasing geometric sequence:
\[ \diam A_{k+1} \le c^k \diam A_1 \]

So we can simply pick a natural number $k$ such that
\[ k > \log_c \frac{|x-y|}{\diam A_1} \]

\item We consider an infinite bounded set $A$ in the real numbers. Since $A$ is bounded, we can pick a closed interval $[a_1,b_1]$ containing $A$.

We then perform a series of binary cuts: Consider the two halves of $[a_1,b_1]$. We know that at least one of these two must contain infinitely many elements in $A$, otherwise $A$ cannot be infinite. We pick this half of the interval and denote it by $[a_2,b_2]$. We continue this to pick a decreasing sequence of closed intervals $[a_n,b_n]$.

Now $\diam [a_{n+1},b_{n+1}] = \frac{1}{2} \diam [a_n,b_n]$, so by the Cantor Intersection Theorem, there exists a unique real number $c$ in the intersection $\bigcap[a_n,b_n]$.

We show that this $c$ is in fact a limit point of $A$.

For any $\epsilon>0$, we need to show that $B_0(c,\epsilon) \cap A \neq \emptyset$, i.e. we need to find an element $x \neq c$ in $A$ that is less than $\epsilon$ apart from $c$.

We then realize that we can simply exploit the decreasing sequence $[a_n,b_n]$
Since $\diam [a_n,b_n]$ is controlled by a decreasing sequence:
\[ \diam [a_{n+1},b_{n+1}] \le 1/2^n \diam [a_1,b_1] \]
We take a sufficiently large n so that $b_n-a_n<\epsilon$
Since $c$ is in $[a_n,b_n]$, for all $x$ in $[a_n,b_n]$ we have $|x-c|\le b_n-a_n<\epsilon$ and therefore $[a_n,b_n]$ is within $B(c,\epsilon)$.

Here's the funny part: $[a_n,b_n]$ contains infinitely many elements of $A$, so it must contain at least one element in A that is not $c$.

Therefore this element $x \neq c$ is in $B_0(c,\epsilon)$.

\item Now we have an infinte bounded set $A$ in $\RR^n$

The idea here is to consecutively come up with better and better sequences of points in $A$. We denote $x_i$ to be the $i$-th coordinate in $\RR^n$.

Our first wish is to pick some elements in $A$ so that they sort of converge at $x_1$.

Because such considerations of 'restricting to a single coordinate' is important here, we define the projection map to the $i$-th coordinate by
\[ f_i(x_1,\dots,x_n)=x_i \]

So, we look at $f_i(A)$ and try to apply BW for the case where $n=1$.

However, the problem is that $f_i(A)$ need not be infinite. For example, the set $\{(0,0),(0,1),(0,2),\dots\}$ projected onto the first coordinate is simply $\{0\}$.

This forces us to consider two cases

Exercise 2: Show that $f_i(A)$ is bounded
This is simple
1. $f_1(A)$ is infinite, then we can apply BW(n=1) to find a real number $c_1$ which is a limit point in $f_1(A)$

Here we can construct a sequence of points 
\[ \{x^{(1),1},x^{(1),2},...\} \]
so that their first coordinates satisfy
\[ |x^{(1),n}_1-c_1| < 1/n \]
for all natural number n
(I know this notation is cumbersome but the problem is that we need multiple sequences for this proof)

2. $f_1(A)$ is finite, then by the Pigeonhole Principle there exists a real number $c_1$ such that its preimage $f_1^{-1}(c_1)$ in $A$ is infinite

In this case we can randomly pick a sequence $\{x^{(1),1},x^{(1),2},\dots\}$ in $A$ so that their first coordinate is equal to $c_1$

I forgot to mention something that is implied, but we actually do have the need to vocabasize that the sequence $\{x^{(1),1},x^{(1),2},\dots\}$ can be chosen to contain mutually distinct entries

Now that we have a sequence that behaves nice on the first coordinate, we may then move on to the second coordinate

Let $A_1=\{x^{(1),1},x^{(1),2},\dots\}$
We again consider $f_2(A_1)$ in two cases, infinite or finite

In any case, we are able to find a subsequence $\{x^{(2),1},x^{(2),2},\dots\}$, where
$x^{(2),k}=x^{(1),n_k}$ for some strictly increasing sequence of natural numbers $n_k$

So that, for the limit point/point with infinite preimage $c_2$, this sequence satisfies
\[ |f_2(x^{(2),n})-c_2| < \frac{1}{n} \]
Note that the property we have for the second case (we in fact have $f_2(x^{(2),n})=c_2$) is just a better version of this.

Now, take note that picking this subsequence does no harm whatsoever towards the first coordinate (if anything it would turn out to be better) since
\[ |f_1(x^{(2),k})-c_1| = |f_1(x^{(1),n_k}-c_1| < \frac{1}{n_k} \le \frac{1}{k} \]
($n_1<\dots<n_k$ is a strictly increasing sequence of natural numbers so $n_k \ge k$)

This continues on until we obtain a sequence of points $\{x^{(n),1},x^{(n),2},\dots\}$ in $A$ so that
\[ |f_i(x^{(n),k}-c_i|<\frac{1}{k} \quad \forall i,k \]

As we can see, the point $c=(c_1,\dots,c_n)$ is in fact a limit point of $A$ as we can always choose a big enough $k$ so that $x^{(n),k}$ is in $B(c,\epsilon) \cap A$.

Since $\{x^{(n),k}\}$ was always chosen to be a sequence of distinct entries, there is no danger for this sequence to always be c, and so c must be a limit point of $A$.

\item We may now return to the general case of Cantor.

Suppose that there is a sequence of decreasing compact sets $A_1,A_2,\dots$ in $\RR^n$. 
Note that every point is contained in $A_1$, so boundedness will never be an issue here.

Since $A_k$ are all nonempty, we can simply pick any element $a_k$ from $A_k$.

For the uncannily specific case that there are only finitely many $\{a_k\}$ chosen, we simply note that, again by Pigeonhole Principle, one of the $a_k$ appears infinitely often; thus for each $A_n$ we simply pick $n_k>n$ so that $A_{n_k}$ contains $a_k$, then $a_k$ is in $A_{n_k}$ which is a subset of $A_n$.

Otherwise, we can then note that $\{a_k\}$ is an infinite bounded set of points, so there must exist a limit point a of $\{a_k\}$.

We can now see that $a$ is always an element of $A_k$:
Using the same technique as Exercise 1, we see that a is a limit point of $\{a_n \mid n>k\}$ and so is a limit point of $A_k$, therefore a is in $A_k$ as $A_k$ is closed.

This proves the first part of the statement
The second part is completely identical to the second part of the $n=1$ case so we don't need to waste our time there either

\item We now consider a compact set A with some open covering $\mathscr{U}$.

This theorem is proved by contradiction: 
Suppose otherwise that set $A$ cannot be covered by any finite collection of open sets in $\mathscr{U}$

Since $A$ is compact, we may enclose it in a closed cube $Q_1$ (whose edges are parallel to the axes)

Now, for each step, we partition $Q$ into $2^n$ cubes by cutting it in half from each direction.

Then, starting from $Q_1$, there must exist one of these smaller cubes, denoted by $Q_2$, such that $A \cap Q_2$ cannot be covered by a finite collection of open sets in $\mathscr{U}$. 
Otherwise, if each $A \cap Q$ has a finite cover, then we simply collect all of these open sets together to form a finite cover of $A$, which violates our assumption.

We continue on to partition $Q_n$ and pick $Q_{n+1}$ so that $A_{n+1}$ has no finite cover (denote $A_n = A \cap Q_n$).

Note that $A$ and $Q_n$ are both compact, so $A_n$ is compact
Also we see that there is a decreasing sequence $A_1,A_2,\dots$
(we can't exactly obtain a relation between $\diam A_n$ and $\diam A_{n+1}$ here)

By Cantor Intersection Theorem we can always find a point $x$ in $A$ located in the intersection $\bigcap A_k$.

Now, since $\mathscr{U}$ is an open covering of $A$, there exists an open set $U$ in $\mathscr{U}$ such that $x\in U$.

The final key step is to exploit the sequence of decreasing cubes $Q_n$. So even though there isn't a clear cut way to control the sizes of $\diam A_n$, we do in fact have the property that $\diam Q_{n+1} = \frac{1}{2^n} \diam Q_1$.

Therefore, by picking a sufficiently large $n$, we can obtain $Q_n$ that is contained in $U$.

But this is a contradiction. 
This is because we've specifically chosen the sequence $A_n$ to be sets that do not possess any finite cover $\{U_1,...,U_n\}$ in $\mathscr{U}$. But here $A_n$ simply would have a one-element cover $\{U\}$.

This completes our proof.
\end{enumerate}
\end{proof}
%https://www.maths.usyd.edu.au/u/bobh/UoS/MATH3901/00met21.pdf

\begin{proposition}
Suppose $Y\subset X$. A subset $E$ of $Y$ is open relative to $Y$ if and only if $E=Y\cap G$ for some open subset $G$ of $X$.
\end{proposition}

\begin{proof} \

($\implies$) Suppose $E$ is open relative to $Y$. Thus for each $p\in E$ there exists $r_p>0$ such that $d(p,q)<r_p$, $q\in Y$ imply $q\in E$.

Let $V_p$ be the set of all $q\in X$ such that $d(p,q)<r_p$, and define
\[G\coloneqq\bigcup_{p\in E}V_p.\]
Then $G$ is an open subset of $X$, by 

($\impliedby$)
\end{proof}

\subsection{Interiors, closures, limit points}
\begin{definition}
Let $X$ be a metric space, and let $E\subset X$. The interior $int(E)$ of $E$ is defined to be the union of all open subsets of $X$ contained in $E$.

The \vocab{closure} $\bar{E}$ is defined to be the intersection of all closed subsets of $X$ containing $E$.

The set $\bar{E}\setminus int(E)$ is known as the boundary of $E$ and denoted $\partial E$. 

A set $E\subseteq X$ is said to be \vocab{dense} if $\bar{E}=X$.
\end{definition}

\begin{definition}
If $X$ is a metric space and $E\subseteq X$ is any subset, then we say a point $x\in X$ is a limit point of $E$ if any open ball about a contains a point of $E$ other than $x$ itself.
\end{definition}

\begin{notation}
We will write $L(E)$ for the set of limit points of $E$.
\end{notation}

\begin{definition}
$x$ is a \vocab{isolated point} of $E$ if $\exists r>0\suchthat B_r(x)\cap S=\{x\}$.
\end{definition}
\pagebreak

\section{Compactness}
\begin{definition}
By an \vocab{open cover} of a set $E$ in a metric space $X$ we mean a collection $\{G_i\mid i\in I\}$ of open subsets of $X$ such that
\[ E\subset\bigcup_{i\in I}G_i. \]
For $I^\prime\subset I$, if the subcollection $\{G_i\mid i\in I^\prime\}$ is also an open cover of $S$; that is,
\[ E\subset\bigcup_{i\in I^\prime}G_\alpha, \]
then $\{G_i\mid i\in I^\prime\}$ is called a \vocab{subcover}. If moreover, $I^\prime$ is finite, then it is called a \vocab{finite subcover}.
\end{definition}

\begin{definition}[Compactness]
A subset $K$ of metric space $X$ is said to be \vocab{compact} if every open cover of $K$ contains a finite subcover.
\end{definition}

\begin{proposition}
Suppose $K\subset Y\subset X$. Then $K$ is compact relative to $X$ if and only if $K$ is compact relative to $Y$.
\end{proposition}

\begin{proof} \

($\implies$) Suppose $K$ is compact relative to $X$. Let $\{V_i\mid i\in I\}$ be a collection of sets open relative to $Y$, such that $K\subset\bigcup_{i\in I}V_i$. By 
\end{proof}

sequential compactness
A set $K$ is compact if and only if every sequence of points in $K$ has a subsequence that converges to a point in $K$.

Any continuous function defined on a compact set is bounded.

extreme value theorem

\section{Perfect Sets}


\section{Connectedness}
\begin{definition}
Two subsets $A$ and $B$ of a metric space $X$ are said to be \vocab{separated} if both $A\cap\bar{B}$ and $\bar{A}\cap B$ are empty, i.e. no point of $A$ lies in the closure of $B$ and no point of $B$ lies in the closure of $A$.

A set $E\subset X$ is said to be \vocab{connected} if $E$ is not a union of two non-empty separated sets. 
\end{definition}

\begin{remark}
Separated sets are of course disjoint, but disjoint sets need not be separated. For example, the interval $[0,1]$ and the segment $(1,2)$ are not separated, since $1$ is a limit point of $(1,2)$. However, the segments $(0,1)$ and $(1,2)$ are separated.
\end{remark}

The connected subsets of the line have a particularly simple structure: 

\begin{proposition}
A subset $E\subset\RR^1$ is connected if and only if it has the following property: if $x,y\in E$ and $x<z<y$, then $z\in E$.
\end{proposition}

\begin{proof}
($\impliedby$) If there exists $x,y\in E$ and some $z\in(x,y)$ such that $z\notin E$, then $E=A_z\cup B_z$ where
\[ A_z=E\cap(-\infty,z), \quad B_z=E\cap(z,\infty). \]
Since $x\in A_z$ and $y\in B_z$, $A$ and $B$ are non-empty. Since $A_z\subset(-\infty,z)$ and $B_z\subset(z,\infty)$, they are separated. Hence $E$ is not connected.

($\implies$) Suppose $E$ is not connecetd. Then there are non-empty separated sets $A$ and $B$ such that $A\cup B=E$. Pick $x\in A$, $y\in B$, and WLOG assume that $x<y$. Define
\[z\coloneqq\sup(A\cap[x,y].)\]
By 
\end{proof}



%%%%

\begin{definition}
We say that a metric space is \vocab{disconnected} if we can write it as the disjoint union of two nonempty open sets. We say that a space is \vocab{connected} if it is not disconnected.
\end{definition}

If $X$ is written as a disjoint union of two nonempty open sets $U$ and $V$ then we say that these sets \vocab{disconnect} $X$.

\begin{example}
If $X=[0,1]\cup[2,3]\subset\RR$ then we have seen that both $[0,1]$ and $[2,3]$ are open in $X$. Since $X$ is their disjoint union, $X$ is disconnected.
\end{example}

The following lemma gives some equivalent ways to formulate the concept of connected space.

\begin{lemma}
Let $X$ be a metric space. Then the following are equivalent:
\begin{enumerate}[label=(\arabic*)]
\item $X$ is connected.
\item If $f:X\to\{0,1\}$ is a continuous function then $f$ is constant.
\item The only subsets of $X$ which are both open and closed are $X$ and $\emptyset$.
\end{enumerate}
(Here the set $\{0,1\}$ is viewed as a metric space via its embedding in $\RR$, or equivalently with the discrete metric.)
\end{lemma}

\begin{proof}

\end{proof}

Frequently one has a metric space $X$ and a subset $Y$ of it whose connectedness or otherwise one wishes to ascertain. To this end, it is useful to record the following lemma.

\begin{lemma}
Let $X$ be a metric space, and let $Y\subseteq X$ be a subset, considered as a metric space with the metric induced from $X$. Then $Y$ is connected if and only if the following is true: if $U$, $V$ are open subsets of $X$, and $U\cap V\cap Y=\emptyset$, then whenever $Y\subseteq U\cup V$, either $Y\subseteq U$ or $Y\subseteq V$.
\end{lemma}

\begin{proof}

\end{proof}

We now turn to some basic properties of the notion of connectedness. These broadly conform with one's intuition about how connected sets should behave.

\begin{lemma}[Sunflower lemma]
Let $X$ be a metric space. Let $\{Ai\mid i\in I\}$ be a collection of connected subsets of $X$ such that $\bigcap_{i\in I}A_i\neq\emptyset$. Then $\bigcup_{i\in I}A_i$ is connected.
\end{lemma}

\begin{proof}

\end{proof}

