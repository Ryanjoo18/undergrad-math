\chapter{Number Systems}\label{chap:number-systems}
\section{Natural Numbers}
In Peano's development, it is assumed that there is a set $\NN$ (the natural numbers) of undefined objects with a distinguished element $1$ such that
\begin{enumerate}[label=(\roman*)]
\item $1$ is a natural number; that is $1\in\NN$;
\item every $n\in\NN$ has a successor $S(n)\in\NN$;
\item for every $n$, $S(n)\neq1$ (there is no number with 1 as successor)
\item if $S(n)=S(m)$, then $n=m$;
\item if $A$ is a set of natural numbers such that $1\in A$ and $n\in A\implies S(n)\in A$, then $A$ contains all natural numbers.
\end{enumerate}
These are known as \vocab{Peano's axioms}.

\begin{theorem}[Archimedean property of $\NN$]
$\NN$ is not bounded above.
\end{theorem}

\begin{proof}
Suppose, for a contradiction, that $\NN$ is bounded above. Then $\NN$ is non-empty and bounded above, so by completeness (of $\RR$) $\NN$ has a supremum.

By the Approximation property with $\epsilon=\frac{1}{2}$, there is a natural number $n\in\NN$ such that $\sup\NN-\frac{1}{2}<n\le\sup\NN$.

Now $n+1\in\NN$ and $n+1>\sup\NN$. This is a contradiction.
\end{proof}
\pagebreak

\section{Integers}
\begin{definition}
For $(a,b),(c,d)\in\NN\times\NN$, we define a relation
\[(a,b)\sim(c,d)\iff a+d=b+c.\]
\end{definition}

\begin{proposition}
$\sim$ is an equivalence relation on $\NN\times\NN$.
\end{proposition}

\begin{proof}
Suppose $(a,b),(c,d),(e,f)\in\NN\times\NN$.
\begin{enumerate}[label=(\roman*)]
\item $\sim$ is reflexive: $(a,b)\sim(a,b)$ because $a+b=b+a$ in $\NN$, by commutativity in $\NN$.
\item $\sim$ is symmetric: If $(a,b)\sim(c,d)$, then $(c,d)\sim(a,b)$ because if $a+d=b+c$, then $c+b=d+a$ in $\NN$.
\item $\sim$ is transitive: 
\end{enumerate}
\end{proof}

%https://www.math.wustl.edu/~freiwald/310integers.pdf
\pagebreak

\section{Rational Numbers}
\begin{notation}
$\ZZ^\prime=\ZZ\setminus\{0\}$.
\end{notation}

\begin{definition}
Let $\sim$ be the binary relation defined on $\ZZ\times\ZZ^\prime$ by
\[ (a,b)\sim(c,d) \iff ad=bc. \]
\end{definition}

\begin{proposition}
$\sim$ is an equivalence on $\ZZ\times\ZZ^\prime$.
\end{proposition}

\begin{proof}
We just check that $\sim$ is transitive. So suppose that $(a,b)\sim(c,d)$ and $(c,d)\sim(e,f)$. Then
\begin{equation*}\tag{1}
ad=bc
\end{equation*}
\begin{equation*}\tag{2}
cf=de
\end{equation*}
Multiplying (1) by $f$ and (2) by $b$, we obtain
\begin{equation*}\tag{3}
adf=bcf
\end{equation*}
\begin{equation*}\tag{4}
bcf=bde
\end{equation*}
Hence $adf=bde$. Since $d\neq0$, the Cancellation Law implies that $af=bc$. Hence $(a,b)\sim(e,f)$.
\end{proof}

\begin{definition}
The set of \vocab{rational numbers} is defined by
\[\QQ\coloneqq\ZZ\times\ZZ^\prime/\sim\]
i.e. $\QQ$ is the set of $\sim$ equivalence classes.
\end{definition}

\begin{notation}
For each $(a,b)\in\ZZ\times\ZZ^\prime$, the corresponding equivalence class is denoted by $[(a,b)]$.
\end{notation}

We define addition $+_\QQ$ and multiplication $\cdot_\QQ$ on $\QQ$ as follows:
\[[(a,b)]+_\QQ[(c,d)]=[(ad+bc,bd)].\]
\[[(a,b)]\cdot_\QQ[(c,d)]=[(ac,bd)].\]

\begin{proposition}
$+_\QQ$ and $\cdot_\QQ$ are well-defined.
\end{proposition}

\begin{lemma}
$\QQ$ is a field, with addition and multiplication as defined above.
\end{lemma}

\begin{proof}
We check the field axioms.
\begin{enumerate}[label=(\roman*)]
\item commutativity of addition
\item associativity of addition
\item Let $0_\QQ=[(0,1)]$. We now show that $0_\QQ$ is an additive identity.

Let $q=[(a,b)]$. Then
\begin{align*}
q+_\QQ 0_\QQ&=[(a,b)]+_\QQ[(0,1)]\\
&=[(a\cdot1+0\cdot b,b\cdot1)]\\
&=[(a,b)]\\
&=q.
\end{align*}
Since for any $q\in\QQ$, $q+_\QQ0_\QQ=q$, thus $0_\QQ$ is an additive identity. Hence an additive identity exists.

\item Consider $r=[(-a,b)]$. Then
\begin{align*}
q+_\QQ r&=[(a,b)]+_\QQ[(-a,b)]\\
&=[(ab+(-a)b,b^2)]\\
&=[(0,b^2)]
\end{align*}
Since $0\cdot1=0\cdot b^2$, we have $(0,b^2)=(0,1)$. Hence 
\begin{align*}
q+_\QQ r&=[(0,b^2)]\\
&=[(0,1)]\\
&=0_\QQ
\end{align*}
Since for any $q\in\QQ$, there exists a unique $r\in\QQ$ such that $q+_\QQ r=0_\QQ$, hence the additive inverse exists.

\item commutativity of multiplication

We want to show that for all $q,r\in\QQ$, $q\cdot_\QQ r=r\cdot_\QQ q$.

\item associativity of multiplication

We want to show that for all $q,r\in\QQ$, $(q\cdot_\QQ r)\cdot_\QQ s=q\cdot_\QQ(r\cdot_\QQ s)$.

\item distributivity

We want to show that for all $q,r,s\in\QQ$, $q\cdot_\QQ(r+_\QQ s)=(q\cdot_\QQ r)+_\QQ(q\cdot_\QQ s)$.

\item Let $1_\QQ=[(1,1)]$. We now show that $1_\QQ$ is a multiplicative identity.

Let $q=[(a,b)]$. Then
\begin{align*}
q\cdot_\QQ1_\QQ
&=[(a,b)]\cdot_\QQ[(1,1)]\\
&=[(a\cdot1,b\cdot1)]\\
&=[(a,b)]\\\
&=1_\QQ
\end{align*}

Since for all $q\in\QQ$, $q\cdot_\QQ1_\QQ=q$, $1_\QQ$ is a multiplicative identity. Hence a multiplicative identity exists.

\item Suppose that $q=[(a,b)]\neq[(0,1)]$. Then $a\neq0$ and so $(b,a)\in\ZZ\times\ZZ^\prime$. Let $r=[(b,a)]$. Then
\begin{align*}
q\cdot_\QQ r
&=[(a,b)]\cdot_\QQ[(b,a)]\\
&=[(ab,ba)]\\
&=[(1,1)]\\
&=1_\QQ.
\end{align*}
\end{enumerate}
Since for every $0_\QQ\neq q\in\QQ$, there exists a unique $r\in\QQ$ such that $q\cdot_\QQ r=1_\QQ$, thus $r$ is a multiplicative inverse. Hence a multiplicative inverse exists.
\end{proof}

Since $\QQ$ is a field, we have the following results:
\begin{enumerate}[label=(\arabic*)]
\item The additive identity in $\QQ$ is unique.
\item The additive inverse of an element of $\QQ$ is unique.
\item The multiplicative identity of $\QQ$ is unique.
\item The multiplicative inverse of a nonzero element of $\QQ$ is unique.
\end{enumerate}

\begin{notation}
Since the additive inverse is unique, we denote the additive inverse of $q\in\QQ$ by $-q$; we define the binary operation $-_\QQ$ on $\QQ$ by
\[q-_\QQ r=q+_\QQ(-r).\]
\end{notation}

\begin{notation}
Since the multiplicative inverse is unique, we denote the additive inverse of $q\in\QQ$ by $q^{-1}$.
\end{notation}

Finally we want to define an order relation on $\QQ$.
\begin{definition}[Order on $\QQ$]
Suppose that $r,s\in\QQ$ and that $r=[(a,b)]$ and $s=[(c,d)]$, where $b,d>0$. Then
\[r\le_\QQ s\iff ad<bc.\]
\end{definition}

\begin{proposition}
$<_\QQ$ is well-defined.
\end{proposition}

\begin{definition}
If $q\in\QQ$, then
\begin{itemize}
\item $q$ is \vocab{positive} if and only if $0_\QQ<_\QQ q$,
\item $q$ is \vocab{negative} if and only if $q<_\QQ0_\QQ$.
\end{itemize}
\end{definition}

\begin{definition}
If $q\in\QQ$, then the \vocab{absolute value} of $q$ is
\[|q|=\begin{cases}
-q&\text{if $q$ is negative,}\\
q&\text{if otherwise.}
\end{cases}\]
\end{definition}
\pagebreak

\section{Real Numbers}
One method to construct $\RR$ from $\QQ$ is Dedekind cuts.

\begin{definition}[Dedekind cut]
A \vocab{Dedekind cut} $\alpha\subset\QQ$ satisfies the following properties:
\begin{enumerate}[label=(\roman*)]
\item $\alpha\neq\emptyset$, $\alpha\neq\QQ$;
\item if $p\in\alpha$, $q\in\QQ$ and $q<p$, then $q\in\alpha$;
\item if $p\in\alpha$, then $p<r$ for some $r\in\alpha$.
\end{enumerate}
\end{definition}

Note that (iii) simply says that $\alpha$ has no largest member; (ii) implies two facts which will be used freely:
\begin{itemize}
\item If $p\in\alpha$ and $q\notin\alpha$ then $p<q$.
\item If $r\notin\alpha$ and $r<s$ then $s\notin\alpha$.
\end{itemize}

\begin{example}
Let $r\in\QQ$ and define
\[ \alpha_r\coloneqq\{p\in\QQ\mid p<r\}. \]
We now check that this is indeed a Dedekind cut.
\begin{enumerate}[label=(\arabic*)]
\item $p=1+r\notin\alpha_r$ thus $\alpha_r\neq\QQ$. $p=r-1\in\alpha_r$ thus $\alpha_r\neq\emptyset$.

\item Suppose that $q\in\alpha_r$ and $q^\prime<q$. Then $q^\prime<q<r$ which implies that $q^\prime<r$ thus $q^\prime\in\alpha_r$.

\item Suppose that $q\in\alpha_r$. Consider $\dfrac{q+r}{2}\in\QQ$ and $q<\dfrac{q+r}{2}<r$. Thus $\dfrac{q+r}{2}\in\alpha_r$.
\end{enumerate}
\end{example}

This example shows that every rational $r$ corresponds to a Dedekind cut $\alpha_r$.

\begin{example}
$\sqrt[3]{2}$ is not rational, but it is real. $\sqrt[3]{2}$ corresponds to the cut
\[ \alpha=\{p\in\QQ\mid p^3<2\}. \]
\begin{enumerate}[label=(\arabic*)]
\item Trivial.
\item If $q<p$, by the monotonicity of the cubic function, this implies that $q^3<p^3<2$ thus $q\in\alpha$.
\item If $p\in\alpha$, consider $\brac{p+\frac{1}{n}}^3<2$.
\end{enumerate}
\end{example}

\begin{definition}
The set of real numbers, denoted by $\RR$, is the set of all Dedekind cuts.
\[ \RR\coloneqq\{\alpha\mid\alpha\text{ is a Dedekind cut}\} \]
\end{definition}

\begin{proposition}
$\RR$ has an order.
\end{proposition}

\begin{proof}
We define $\alpha<\beta$ to mean that $\alpha\subset\beta$. Let us check if this is an order (check for transitivity and trichotomy).
\begin{enumerate}[label=(\arabic*)]
\item For $\alpha,\beta,\gamma\in\RR$, if $\alpha<\beta$ and $\beta<\gamma$ it is clear that $\alpha<\gamma$. (A proper subset of a proper subset is a proper subset.)

\item It is clear that at most one of the three relations
\[ \alpha<\beta, \quad \alpha=\beta, \quad \beta<\alpha \]
can hold for any pair $\alpha,\beta$. 

To show that at least one holds, assume that the first two fail. Then $\alpha$ is not a subset of $\beta$. Hence there exists some $p\in\alpha$ with $p\in\beta$.

If $q\in\beta$, it follows that $q<p$ (since $p\notin\beta$), hence $q\in\alpha$, by (ii). Thus $\beta\subset\alpha$. Since $\beta\neq\alpha$, we conclude that $\beta<\alpha$.
\end{enumerate}
Thus $\RR$ is an ordered set.
\end{proof}

\begin{proposition}
The ordered set $\RR$ has the least-upper-bound property.
\end{proposition}

\begin{proof}
Let $A\neq\emptyset$, $A\subset\RR$. Assume that $\beta\in\RR$ is an upper bound of $A$.

Define $\beta$ to be the union of all $\alpha\in A$; in other words, $p\in\gamma$ if and only if $p\in\alpha$ for some $\alpha\in A$. We shall prove that $\gamma\in\RR$ by checking the definition of Dedekind cuts:
\begin{enumerate}[label=(\arabic*)]
\item Since $A$ is not empty, there exists an $\alpha_0\in A$. This $\alpha_0$ is not empty. Since $\alpha_0\subset\gamma$, $\gamma$ is not empty.

Next, $\gamma\subset\beta$ (since $\alpha\subset\beta$ for every $\alpha\in A$), and therefore $\gamma\neq\QQ$.

\item Pick $p\in\gamma$. Then $p\in\alpha_1$ for some $\alpha_1\in A$. If $q<p$, then $q\in\alpha_1$, hence $q\in\gamma$.

\item If $r\in\alpha_1$ is so chosen that $r>p$, we see that $r\in\gamma$ (since $\alpha_1\subset\gamma$).
\end{enumerate}

Next we prove that $\gamma=\sup A$.
\begin{enumerate}[label=(\arabic*)]
\item It is clear that $\alpha\le\gamma$ for every $\alpha\in A$.
\item Suppose $\delta<\gamma$. Then there is an $s\in\gamma$ and that $s\notin\delta$. Since $s\in\gamma$, $s\in\alpha$ for some $\alpha\in A$. Hence $\delta<\alpha$, and $\delta$ is not an upper bound of $A$.
\end{enumerate}
\end{proof}

\begin{definition}
Given $\alpha,\beta\in\RR$. Define 
\[\alpha+\beta\coloneqq\{r\in\QQ\mid r=a+b,a\in\alpha,b\in\beta\}.\]
\end{definition}

\begin{proposition}[Addition on $\RR$ is closed]
$\alpha+\beta\in\RR$.
\end{proposition}

\begin{proof} \
\begin{enumerate}[label=(\arabic*)]
\item $\alpha\neq\emptyset$ and $\beta\neq\emptyset$ implies there exists $a\in\alpha$ and $b\in\beta$. Hence $r=a+b\in\alpha+\beta$ so $\alpha+\beta\neq\emptyset$.

Since $\alpha\neq\QQ$ and $\beta\neq\QQ$, there is $c\neq\alpha$ and $d\neq\beta$. $r^\prime=c+d>a+b$ for any $a\in\alpha,b\in\beta$, so $r^\prime\notin\alpha+\beta$. Hence $\alpha+\beta\neq\QQ$.

\item Suppose that $r\in\alpha+\beta$ and $r^\prime<r$. We want to show that $r^\prime\in\alpha+\beta$.

$r=a+b$ for some $a\in\alpha,b\in\beta$. $r^\prime-a<b$. Since $\beta\in\RR$, $r^\prime-a\in\beta$ so $r^\prime-a=b_1$ for some $b_1\in\beta$. Hence $r^\prime=a+b_1\in\alpha+\beta$.

\item Suppose $r\in\alpha+\beta$, so $r=a+b$ for some $a\in\alpha,b\in\beta$. There exists $a^\prime\in\alpha,b^\prime\in\beta$ with $a<a^\prime$ and $b<b^\prime$. Then $r=a+b<a^\prime+b^\prime\in\alpha+\beta$. We define $r^\prime=a^\prime+b^\prime\in\alpha+\beta$ with $r<r^\prime$.
\end{enumerate}
\end{proof}

We now prove that the set of real numbers satisfies the commutative, associative, and identity field axioms with respect to addition.

\begin{proposition}[Addition on $\RR$ is commutative]
$\forall\alpha,\beta\in\RR$, $\alpha+\beta=\beta+\alpha$.
\end{proposition}

\begin{proof}
We need to show that $\alpha+\beta\subseteq\beta+\alpha$ and $\beta+\alpha\subseteq\alpha+\beta$.

Let $r\in\alpha+\beta$. Then $r=a+b$ for $a\in\alpha$ and $b\in\beta$. Thus $r=b+a$ since $+$ is commutative on $\QQ$. Hence $r\in\beta+\alpha$. Therefore $\alpha+\beta\subseteq\beta+\alpha$.

Similarly, $\beta+\alpha\subseteq\alpha+\beta$.

Therefore $\alpha+\beta=\beta+\alpha$.
\end{proof}

\begin{proposition}[Addition on $\RR$ is associative]
$\forall\alpha,\beta,\gamma\in\RR$, $\alpha+(\beta+\gamma)=(\alpha+\beta)+\gamma$.
\end{proposition}

\begin{proof}
Let $r\in\alpha+(\beta+\gamma)$. Then $r=a+(b+c)$ where $a\in\alpha,b\in\beta,c\in\gamma$. Thus $r=(a+b)+c$ by associativity of $+$ on $\QQ$. Therefore $r\in(\alpha+\beta)+\gamma$, hence $\alpha+(\beta+\gamma)\subseteq(\alpha+\beta)+\gamma$.

Similarly, $(\alpha+\beta)+\gamma\subseteq\alpha+(\beta+\gamma)$.
\end{proof}

\begin{proposition}
Define $0^*\coloneqq\{p\in\QQ\mid p<0\}$. Then $\alpha+0^*=\alpha$.
\end{proposition}

\begin{proof}
Let $r\in\alpha+0^*$. Then $r=a+p$ for some $a\in\alpha,p\in0^*$. Thus $r=a+p<a+0=a$ by ordering on $\QQ$ and identity on $\QQ$. Hence $\alpha+0^*\subseteq\alpha$.

Let $r\in\alpha$. Then there exists $r^\prime>p$ where $r^\prime\in\alpha$. Thus $r-r^\prime<0$, so $r-r^\prime\in0^*$. We see that
\[ r=\underbrace{r^\prime}_{\in\alpha}+\underbrace{(r-r^\prime)}_{\in0^*}. \]
Hence $\alpha\subseteq\alpha+0*$.
\end{proof}

%%%%%%%%%%%%%%%%

\begin{exercise}
Express $-\alpha$ in terms of $\alpha$; show
\[ \alpha+(-\alpha)=0=(-\alpha)+\alpha \]
\end{exercise}

\begin{proof}
We split this into two cases.

\textbf{Case 1}: $\alpha$ is a rational number, then $\alpha=(A,B)$ where $A = \{x \mid x < \alpha\}$, $B = \{x \mid x \ge \alpha\}$.

Let $-\alpha=(A^\prime,B^\prime)$, where $A^\prime = \{x \mid x < -\alpha\}$, $B^\prime = \{x \mid x\ge -\alpha\}$. 
We see that $\alpha+(-\alpha) \le 0$ is obvious.

On the other hand, since $0=(O,O^\prime)$, for any $\epsilon<0$ we have
\[ \epsilon = \brac{\alpha+\frac{\epsilon}{2}} + \brac{-\alpha+\frac{\epsilon}{2}} \in A+A^\prime \]
Hence $\alpha+(-\alpha)=0$.

\

\textbf{Case 2}: $\alpha$ is irrational, let $\alpha = (A,B)$ where $B$ does not have a lowest value. 
Then $-B = \{-x \mid x \in B\}$ does not have a highest value.

We wish to define $-\alpha=(-B,-A)$, but first we need to show that this is well-defined by checking through all the conditions.

\begin{itemize}
\item Property 1: This is trivial.

\item Property 2: Prove that $- A$ and $B$ are disjoint.

Note that $\forall x \in \RR$, if $x=-y$, then exactly one out of $y \in A$ and $y \in B$ is true $\implies$ exactly one out of $x \in -B$ and $x \in -A$ is true.

\item Property 3: Prove $-B$ is closed downwards.

Suppose otherwise, that $x<y, y \in -B$ but $x \notin -B$. Then $-y \in B$, $-x \notin B$. Since $A$ is the complement of $B$, $-y \notin A$, $-x \in A$. But $-y<-x$, which is a contradiction.

\item Property 4 is already guaranteed by the irrationality of $\alpha$.
\end{itemize}

All of these properties imply that the real numbers form a commutative group by addition.
\end{proof}

\subsubsection{Negation}
Given any set $X \subset \RR$, let $-X$ denote the set of the negatives of those rational numbers. That is $x \in X$ if and only if $-x \in -X$. 

If $(A,B)$ is a Dedekind cut, then $-(A,B)$ is defined to be
$(-B,-A)$.

This is pretty clearly a Dedekind cut. - proof

\subsubsection{Signs}
A Dedekind cut $(A,B)$ is \textbf{positive} if $0 \in A$ and \textbf{negative} if $0 \in B$. If $(A,B)$ is neither positive nor negative, then $(A,B)$ is the cut representing 0.

If $(A,B)$ is positive, then $-(A,B)$ is negative. Likewise, if $(A,B)$ is negative, then $-(A,B)$ is positive. The cut $(A,B)$ is non-negative if it is either positive or 0.




\begin{definition}[Multiplication on $\RR$]
Given $\alpha,\beta\in\RR$, $\alpha,\beta>0^*$. Define
\[\alpha\cdot\beta\coloneqq\{p\in\QQ\mid p\le ab,a\in\alpha,b\in\beta,a,b>0\}.\]
\end{definition}

\begin{proposition}[Multiplication on $\RR$ is closed]
$\alpha\cdot\beta\in\RR$.
\end{proposition}

\begin{proof} \
\begin{enumerate}[label=(\arabic*)]
\item $\alpha\neq\emptyset$ means there exists $a\in\alpha,a>0$. Similarly, $\beta\neq\emptyset$ means there exists $b\in\beta,b>0$. Then $a\cdot b\in\QQ$ and $ab\le ab$, so $ab\in\alpha\cdot\beta\neq\emptyset$.

$\alpha\neq\QQ$ means there exists $a^\prime\notin\alpha,a^\prime>a$ for all $a\in\alpha$. $\beta\neq\QQ$ means there exists $b^\prime\in\beta,b^\prime>b$ for all $b\in\beta$. Then $a^\prime b^\prime>ab$ for all $a\in\alpha,b\in\beta$, so $a^\prime b^\prime\notin\alpha\cdot\beta$, thus $\alpha\cdot\beta\neq\QQ$.

\item $p<\alpha\cdot\beta$ means $p\le a\cdot b$ for some $a\in\alpha,b\in\beta,a,b>0$.

For $q<p$, $q<p\le a\cdot b$ so $q\in\alpha\cdot\beta$.

\item $p\in\alpha\cdot\beta$ means $p\le a\cdot b$ for some $a\in\alpha,b\in\beta,a,b>0$. Pick $a^\prime\in\alpha$ and $b^\prime\in\beta$ with $a^\prime>a$ and $b^\prime>b$. Form $a^\prime b^\prime>ab\ge p$, $a^\prime b^\prime\le a^\prime b^\prime$ means $a^\prime b^\prime\in\alpha\cdot\beta$.
\end{enumerate}
Hence $\alpha\cdot\beta$ is a Dedekind cut.
\end{proof}
\pagebreak

\subsection{Properties}
\begin{theorem}[$\RR$ is archimedian]\label{thrm:r-archimedian}
For any $x\in\RR^+$ and $y\in\RR^+$, there exists some $n\in\ZZ^+$ so that
\[n\cdot x>y.\]
\end{theorem}

\begin{proof}
In particular, if we take $x=1$ from this theorem, we immediately get the following statement.

\begin{proposition}\label{prop:r-archimedian}
For any $y\in\RR$, there exists some positive integer $n$ so that $n>y$.
\end{proposition}

We now give a proof of Proposition \ref{prop:r-archimedian} directly without using Theorem \ref{thrm:r-archimedian}, and then we prove Theorem \ref{thrm:r-archimedian} from Proposition \ref{prop:r-archimedian}. This shows that these two statements are in fact equivalent, though Proposition \ref{prop:r-archimedian} looks much simpler.

\begin{proof}
Assume $n\in\ZZ^+$ does not exist; that is to say that the set of positive integers $\ZZ^+$ has an upper bound $y$. Then using the l.u.b. property of $\RR$, $\sup\ZZ^+$ exists, which we denote by $x_0\in\RR$.

Now we look at $x_0-1$. This is not an upper bound by definition of $x_0$, which means there exists some $N\in\ZZ^+$ such that
\[x_0-1<N.\]
Then it follows that $x_0<N+1$. Notice that $N+1\in\ZZ^+$. So this contradicts the assumption that $x_0$ is an upper bound.

Hence our original assumption cannot be true, and thus there exists $n\in\ZZ^+$ with $n>y$.
\end{proof}

For any $x\in\RR^+$ and $y\in\RR$, consider $y\cdot x^{-1}\in\RR$. From Proposition \ref{prop:r-archimedian}, there exists some $n\in\ZZ^+$ such that
\[n>y\cdot x^{-1}.\]
Then this is equivalent to $n-yx^{-1}>0$. Since $x>0$, and $\RR$ is an ordered field, we have
\[(n-y\cdot x^{-1})\cdot x>0.\]
This is equivalent to $n\cdot x>y$.
\end{proof}

\begin{remark}
The archimedian property guarantees that we can use decimals to represent real numbers. %See Rudin 1.22
\end{remark}

% https://mth32015.files.wordpress.com/2015/01/jan-26-30.pdf

\begin{theorem}[$\QQ$ is dense in $\RR$]
For any $a,b\in\RR$ with $a<b$, there exists some $x\in\QQ$ such that $a<x<b$.
\end{theorem}

\begin{proof}
This means one can find some $m\in\ZZ$ and $n\in\ZZ^+$ so that
\[a<\frac{m}{n}<b,\]
which is further equivalent to finding $m\in\ZZ$ and $n\in\ZZ^+$ so that
\[an<m<bn.\]
Notice that $b-a>0$, so by the archimedian property, there exists $n\in\ZZ^+$ so that
\[bn-an=(b-a)n>1.\]
We now argue that there exists some integer between two real numbers, whenever their difference is larger than 1.

\begin{lemma}
For any $\alpha,\beta\in\RR$ with $\beta-\alpha>1$, there exists some integer $m$ so that $\alpha<m<\beta$.
\end{lemma}

\begin{proof}
We prove this lemma by finding such $m$. First, using archimedian property of $\RR$, we can find some integer $N>0$ so that
\[-N<\alpha<\beta<N.\]
Then consider the integers which are smaller than $N$ and greater than $\alpha$, i.e., the set
\[A\coloneqq\{k\in\ZZ\mid a<k\le N\}.\]
It is not empty since $N\in A$. Since this a subset of $\{-N+1,-N+2,\dots,N-2,N-1,N\}$ which is a finite set, it contains only finite elements. We can pick the smallest one from it and denote it by $m$, i.e., $m\coloneqq\min A$. We claim this $m$ is just the one we are looking for.

First since $m\in A$, $m>\alpha$. Then we only need to check $m<\beta$. If this is not true, i.e., $m\ge\beta$, then we consider $m-1$. It follows
\[m-1\ge\beta-1\ge\alpha.\]
This contradicts the fact that $m$ is the smallest integer which is greater than $\alpha$.

Above all, we are done with the lemma.
\end{proof}

At last, apply the lemma to $\alpha=an$ and $\beta=bn$, we are done.
\end{proof}

\begin{theorem}[$\RR$ is closed under taking roots]
For every $y\in\RR^+$ and every $n\in\ZZ^+$, there exists a unique $x\in\RR^+$ so that $x^n=y$.
\end{theorem}

\begin{proof}
We first claim that such $x\in\RR^+$, if exists, must be unique. Otherwise, assume that both $x_1,x_2\in\RR^+$ are solutions of the equation
\[x^n=y,\quad y\in\RR^+,n\in\ZZ^+.\]
Assume now $x_1<x_2$, then from the fact that $\RR$ is an ordered field, we have $x_1^n<x_2^n$ (why?), a contradiction. Similarly, $x_1>x_2$ also leads to a contradiction, and so $x_1=x_2$.

Now we look for a solution for the equation. Consider a subset of $\RR$ as
\[S\coloneqq\{a\in\RR^+\mid a^n<y\}.\]
Try to check that
\begin{enumerate}[label=(\arabic*)]
\item $S\neq\emptyset$;
\item $S$ has an upper bound.
\end{enumerate}
Then using the fact that $\RR$ has the l.u.b. property, $\sup S$ exists. Define it as $x$, clearly $x\in\RR^+$. We show that $x$ solves the equation. (The idea of the proof is similar to the proof of $\sup_\QQ\{x\in\QQ\mid x^2\le2\}$ does not exist.)

First, we show that if $x^n<y$, then we can construct some $x_0\in S$ which is greater than $x$, which says $x$ is not an upper bound of $S$. So $x^n\ge y$.

Second, we show that if $x^n>y$, then we can find an upper bound of $S$ which is smaller than $x$, which says that $x$ is not the least upper bound. So $x^n\le y$.

Above all, we must have $x^n=y$.

From now on, we use $y^\frac{1}{n}$ to denote the unique solution for the equation
\[x^n=y,\quad y\in\RR^+,n\in\ZZ^+,\]
and call it the $n$-th real root of $y$. The property
\[(ab)^\frac{1}{n}=a^\frac{1}{n}\cdot b^\frac{1}{n}\]
immediately follows from the uniqueness of $n$-th real root.
\end{proof}

\begin{theorem}[Completeness axiom for $\RR$]
If non-empty $E\subset\RR$ is bounded above, then $E$ has a supremum.
\end{theorem}

Any set in the reals bounded from above/below must have a supremum/infimum.

\begin{proof}
We prove this using Dedekind cuts.

Let $S$ be a real number set. 
We consider the rational number set $A = \{x \in \QQ \mid \exists y \in S\}$. Set $B$ is defined to be the complement of $A$ in $\QQ$.

We go through the definitions to check that $(A|B)$ is a Dedekind cut.
\begin{enumerate}
\item Since $S \neq \emptyset$, pick $y \in S$, then $[y]-1$ is a real number smaller than some element in $S$, hence $[y]-1 \in A$ and thus $A \neq \emptyset$.

Since we're given that $S$ is bounded, $\exists M>0$ as the upper bound for $S$, thus $B \neq \emptyset$.

(Note that an upper bound is simply a number that is bigger than anything from the set, and is not the supremum

\item We defined $B$ to be the complement of $A$ in $\QQ$, so this condition is trivial.

\item For any $x,y \in A$, if $x<y$ and $y\in A$, then $\exists z \in S$ such that $y<z \implies x<z \implies x \in A$.

\item Suppose otherwise that $x \in A$ is the largest element in A, then $\exists y \in S$ such that $x<y$
We then pick a rational number $z$ between $x$ and $y$. 
Since we still have $z<y$, we have $z \in A$ but $z>x$, contradictory to $z$ being the largest.

Now there's actually an issue with the proof for property 4 here
How exactly are we finding z?

First $x \in \QQ$. 
Then $y \in \RR$ so we rewrite it as $y=(C|D)$ via definition.

$x<y$ translates to the fact that $x \in C$.

Since $y$ is real, by definition we know that $C$ must not have a largest element.

In particular, $x$ is not largest and we can pick $z \in C$ such that $z>x$. 
This is in fact the $z$ that we need
\end{enumerate}

Now that all the properties of a real number are validated, we may finally conclude that $\alpha=(A|B)$ is indeed a real number.

Now we need to show that $\alpha = \sup S$.

Let $x \in S$. 
If $x$ is not the maximum value of $S$, i.e. $\exists y \in S,x<y$, then $x \in A$ and thus $x<\alpha$.

If $x$ is the maximum value of $S$, then for any rational number $y<x$ we have $y \in A$, and for any rational number $y \ge x$ we have $y \in B$.
Thus $x=(A|B)=\alpha$.

In conclusion, $x \le \alpha$ for all $x \in S$.

For any upper bound $x$ of $S$, since $\forall y \in S, x \ge y$ we have $x \in B$ and thus $x \ge \alpha$.

$\therefore$ $\alpha$ is the smallest upper bound of $S$ and thus $\sup S = \alpha$ exists.
\end{proof}

\subsection{Extended real number system}
\begin{definition}
We add $\pm\infty$ to $\RR$, and call the union $\RR\cup\{\pm\infty\}$ the \vocab{extended real number system}. Now any non-empty set $E\subset\RR$ has a supremum and infimum, since we can define
\[\sup E=+\infty,\quad\text{if $E$ has no upper bound in $\RR$}\]
and
\[\inf E=-\infty,\quad\text{if $E$ has no lower bound in $\RR$.}\]
\end{definition}

The extended real number system does not form a field, but it is customary to make the following conventions:
\begin{enumerate}[label=(\arabic*)]
\item If $x$ is real then
\[ x+\infty=+\infty, \quad x-\infty=-\infty, \quad \frac{x}{+\infty}=\frac{x}{-\infty}=0. \]
\item If $x>0$ then $x\cdot(+\infty)=+\infty$, $x\cdot(-\infty)=-\infty$.
\item If $x<0$ then $x\cdot(+\infty)=-\infty$, $x\cdot(-\infty)=+\infty$.
\end{enumerate}
When it is desired to make the distinction between real numbers on the one hand and the symbols $+\infty$ and $-\infty$ on the other quite explicit, the former are called \vocab{finite}.
\pagebreak

\section{Complex Numbers}
We consider the Cartesian product of $\RR$ with $\RR$; that is,
\[ \RR^2\coloneqq\RR\times\RR\coloneqq\{(x_1,x_2)\mid x_1,x_2\in\RR\}. \]
Over $\RR^2$, we can define operations
\begin{itemize}
\item Addition $+$: $(x_1,x_2)+(y_1,y_2)=(x_1+y_1,x_2+y_2)$;
\item Scalar multiplication $\RR\times\RR^2\to\RR^2$: $c\cdot(x_1,x_2)=(c\cdot x_1,c\cdot x_2)$.
\end{itemize}

This two operations make $\RR^2$ a 2-dimensional vector space (linear space) over the real field $\RR$. We also say $\RR^2$ is a $\RR$-linear space of real dimension 2. For example, $\{(1,0),(0,1)\}$ form a basis of $\RR^2$.

Moreover, over the linear space $\RR^2$, one can define an inner product as
\[ \langle(x_1,x_2),(y_1,y_2)\rangle=x_1y_1+x_2y_2. \]
The inner product induces a norm
\[ |(x_1,x_2)|=\sqrt{\langle(x_1,x_2),(x_1,x_2)\rangle}=\sqrt{x_1^2+x_2^2}. \]
From now on, we use $\vec{x}$ to denote $(x_1,x_2)$.
\begin{proposition} \
\begin{itemize}
\item $|\vec{x}|\ge0$, where equality holds if and only if $\vec{x}=\vec{0}$.
\item $|c\cdot\vec{x}|=|c||\vec{x}|$
\item $|\vec{x}+\vec{y}|\le|\vec{x}|+|\vec{y}|$
\item $|\langle\vec{x},\vec{y}\rangle|\le|\vec{x}||\vec{y}|$
\end{itemize}
\end{proposition}

All constructions here can be easily generalised to any $\RR^n$ with $n\in\ZZ^+$.

Over $\RR^2$, we can define a multiplication $\cdot$ as
\[ (a,b)\cdot(c,d)=(ac-bd,ad+bc). \]
If we identity $\RR^2$ with
\[ \CC\coloneqq\{x+yi\mid x,y\in\RR\} \]
via $(x,y)\mapsto x+yi$, then all structures defined above are induced to $\CC$. In particular, the multiplication is induced to $\CC$ via requiring $i^2=-1$. A nontrivial fact is that $(\CC,+,\cdot)$ is a field. A element in $\CC$ is called a complex number. Usually, people prefer to use $z=x+yi$, $x,y\in\RR$, to denote a complex number. Here $x$ is called the real part of $z$ and $y$ is called the imaginary part of $z$. We use $|z|$ to denote its norm.
\pagebreak

\section{Euclidean Spaces}
For each positive integer $n$, let $\RR^n$ be the set of all ordered $n$-tuples
\[ \vb{x}=(x_1,x_2,\dots,x_n), \]
where $x_1,\dots,x_n$ are real numbers, called the \vocab{coordinates} of $\vb{x}$. The elements of $\RR^n$ are called points, or vectors, especially when $n>1$. We shall denote vectors by boldfaced letters.

Since $\RR^n$ is a vector space (over $\RR$), $\RR^n$ has the following extra properties
\begin{itemize}
\item For any two vectors $\vb{x}$ and $\vb{y}$ we may perform addition:
\[ \vb{x}+\vb{y}=(x_1+y_1,\dots,x_n+y_n) \]
Properties of addition:
\begin{enumerate}
\item $\vb{x}+\vb{y}=\vb{y}+\vb{x}$
\item $(\vb{x}+\vb{y})+\vb{z}=\vb{x}+(\vb{y}+\vb{z})$
\item Zero vector $\vb{0}=(0,\dots,0)$ satisfies $\vb{x}+\vb{0}=\vb{0}+\vb{x}=\vb{x}$
\item For any vector $\vb{x}$, its negative $-\vb{x}$ satisfies $\vb{x}+(-\vb{x})=(-\vb{x})+\vb{x}=\vb{0}$
\end{enumerate}
\item For any vector $\vb{x}$ and scalar $k\in\RR$ we may perform scalar multiplication:
\[ k\vb{x}=(kx_1,\dots,kx_n) \]
Properties of scalar multiplication:
\begin{enumerate}
\item $0\cdot\vb{x}=\vb{0},1\cdot\vb{x}=\vb{x}$
\item $(kl)\vb{x}=k(l\vb{x})=l(k\vb{x})$
\item $k(\vb{x}+\vb{y})=k\vb{x}+k\vb{y}$
\item $(k+l)\vb{x}=k\vb{x}+l\vb{x}$
\end{enumerate}
\end{itemize}

We define the \textbf{inner product} (or scalar product) of $\vb{x}$ and $\vb{y}$ by
\[ \vb{x}\cdot\vb{y}\coloneqq\sum_{i=1}^nx_iy_i. \]

The Euclidean space builds upon the vector space $\RR^n$; specifically speaking, it is $\RR^n$ endowed with two extra notions:
\begin{itemize}
\item The \textbf{norm} of the Euclidean space $\norm{\cdot}$ is a real-valued function $\norm{\cdot}:\RR^n\to\RR$. Given a vector $\vb{x}=(x_1,\dots,x_n)$ in $\RR^n$, the norm of $\vb{x}$ is defined as
\[ \norm{\vb{x}}\coloneqq\sqrt{\vb{x}\cdot\vb{x}}=\sqrt{\sum_{i=1}^nx_i^2}=\sqrt{x_1^2+\cdots+x_n^2}. \]
\item The \textbf{metric} $d$ of the Euclidean space is a real-valued function $d:\RR^n\times\RR^n\to\RR$. Given two vectors $\vb{x}=(x_1,\dots,x_n)$ and $\vb{y}=(y_1,\dots,y_n)$, the distance between $\vb{x}$ and $\vb{y}$ is defined as
\[ d(\vb{x},\vb{y})\coloneqq\norm{\vb{x}-\vb{y}}=\sqrt{\sum_{i=1}^n(x_i-y_i)^2}=\sqrt{(x_1-y_1)^2+\cdots+(x_n-y_n)^2}. \]
\end{itemize}

\begin{remark}
The norm is something like the length of the vector itself (distant to the origin); the metric refers to the distance function which measures the length between two points in $\RR^n$ (determined by their positional vectors $\vb{x}$ and $\vb{y}$). Essentially, the metric is a much more general notion than the norm: the norm can only be defined on vector spaces; the metric can literally be defined on any set.
\end{remark}

Norms are required to satisfy the following properties:
\begin{enumerate}[label=(\arabic*)]
\item (\textbf{positive definiteness}) for any vector $\vb{x}$, $\norm{\vb{x}}\ge0$, and equality holds if and only if $\vb{x}=\vb{0}$.
\item (\textbf{absolute homogeneity}) for any vector $\vb{x}$ and scalar $a$, $\norm{a\vb{x}}=|a|\norm{\vb{x}}$.
\item (\textbf{triangle inequality}) for any two vectors $\vb{x}$ and $\vb{y}$, $\norm{\vb{x}+\vb{y}}\le\norm{\vb{x}}+\norm{\vb{y}}$.
\end{enumerate}

Metrics are required to satisfy the following properties:
\begin{enumerate}[label=(\arabic*)]
\item (\textbf{positive definiteness}) for any two elements $\vb{x}$ and $\vb{y}$, $d(\vb{x},\vb{y})\ge0$, equality holds if and only if $\vb{x}=\vb{y}$.
\item (\textbf{symmetry}) for any two elements $\vb{x}$ and $\vb{y}$, $d(\vb{x},\vb{y})=d(\vb{y},\vb{x})$.
\item (\textbf{triangle inequality}) for any three elements $\vb{x}$, $\vb{y}$ and $\vb{z}$, $d(\vb{x},\vb{z})\le d(\vb{x},\vb{y})+d(\vb{y},\vb{z})$.
\end{enumerate}

Generally, if there is a norm $\norm{\cdot}$ on some vector space, then this norm naturally determines a metric $d(x,y)=\norm{x-y}$, which is precisely the case for Euclidean spaces.

\begin{definition}
$E\subset\RR^n$ is \vocab{bounded} if there exists $M>0$ such that $\norm{x}\le M$ for all $x\in E$.
\end{definition}

\begin{exercise}
Given $E$ and $F$ in $\RR^n$ and real number $k$, define
\[ kE=\{kx \mid x\in E\} \]
\[ E+F=\{x+y \mid x\in E,y\in F\} \]
\begin{enumerate}[label=(\alph*)]
\item Show that if $E$ is bounded, then $kE$ is bounded;
\item Show that if $E$ and $F$ are bounded, then $E+F$ is bounded.
\end{enumerate}
\end{exercise}

\begin{definition}
The \vocab{diameter} of $E\subset\RR^n$ is defined as
\[ \diam E\coloneqq\sup_{x,y\in E}d(x,y). \]
\end{definition}

\begin{exercise}
Find the diameter of the open unit ball in $\RR^n$ given by
\[ B=\{x\in\RR^n \mid \norm{x}<1\}. \]
\end{exercise}

\begin{solution}
First note that
\[ d(x,y)=\norm{x-y}\le\norm{x}+\norm{-y}=\norm{x}+\norm{y}<1+1=2. \]
On the other hand, for any $\epsilon>0$, we pick
\[ x=\brac{1-\frac{\epsilon}{4},0,\dots,0}, \quad y=\brac{-\brac{1-\frac{\epsilon}{4}},0,\dots,0}. \]
Then $d(x,y)=2-\dfrac{\epsilon}{2}>2-\epsilon$.

Therefore $\diam B = 2$.
\end{solution}

\begin{exercise}
Given a set $E$ in $\RR^n$, show that $E$ is bounded if and only if $\diam E<+\infty$.
\end{exercise}
\begin{proof} \

($\implies$) If $E$ is bounded, then there exists $M>0$ such that $\norm{x}\le M$ for all $x \in E$.

Thus for any $x,y \in E$,
\[ d(x,y)=\norm{x-y}\le\norm{x}+\norm{y}\le2M. \]
Thus $\diam E = \sup d(x,y) \le 2M<+\infty$.

($\impliedby$) Suppose that $\diam E=r$. Pick a random point $x \in E$, suppose that $\norm{x}=R$.

Then for any other $y \in E$,
\[ \norm{y}=\norm{x+(y-x)}\le\norm{x}+\norm{y-x}\le R+r. \]
Thus, by picking $M=R+r$, we obtain $\norm{y}\le M$ for all $y \in E$, and we are done.

\begin{remark}
Basically you use $x$ to confine $E$ within a ball, which is then confined within an even bigger ball centered at the origin.
\end{remark}
\end{proof}

\begin{definition}
The \vocab{distance between sets} $E\subset\RR^n$ and $F\subset\RR^n$ is defined as
\[ d(E,F)\coloneqq\inf_{x\in E,y\in F}\norm{x-y}. \]
\end{definition}

Obviously $d(E,F)>0$ implies that $E$ and $F$ are disjoint, but $E$ and $F$ may still be disjoint even if $d(E,F)=0$. For example, the closed intervals $E=(-1,0)$ and $F=(0,1)$.

\begin{exercise}
Suppose that $E$ and $F$ are sets in $\RR^n$ where $E$ and $F$ is finite. Prove that $E$ and $F$ are disjoint if and only if $d(E,F)>0$.
\end{exercise}
\pagebreak

\begin{exercise}[Rudin Q1]
If $r\neq0$ is rational and $x$ is irrational, prove that $r+x$ and $rx$ are irrational.
\end{exercise}

\begin{solution}
We prove by contradiction. Suppose $r+x$ is rational, then $r+x=\dfrac{m}{n},m,n\in\ZZ$, and $m,n$ have no common factors. Then $m=n(r+x)$. Let $r=\frac{p}{q},p,q\in\ZZ$, the former equation implies that $m=n\brac{\frac{p}{q}+x}$, i.e., $qm=n(p+qx)$, giving
\[x=\frac{mq-np}{nq},\]
which says that $x$ can be written as the quotient of two integers, so $x$ is rational, a contradiction.

The proof for the case $rx$ is similar.
\end{solution}

\begin{exercise}[Rudin Q2]
Prove that there is no rational number whose square is $12$.
\end{exercise}