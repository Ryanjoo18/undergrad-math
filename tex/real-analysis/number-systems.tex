\chapter{Real and Complex Number Systems}\label{chap:number-systems}

This chapter discusses the construction and properties of the real field $\RR$, the complex field $\CC$, and Euclidean space $\RR^n$.

\begin{comment}
\section{Natural Numbers}
In Peano's development, it is assumed that there is a set $\NN$ (the natural numbers) of undefined objects with a distinguished element $1$ such that
\begin{enumerate}[label=(\roman*)]
\item $1$ is a natural number; that is $1\in\NN$;
\item every $n\in\NN$ has a successor $S(n)\in\NN$;
\item for every $n$, $S(n)\neq1$ (there is no number with 1 as successor)
\item if $S(n)=S(m)$, then $n=m$;
\item if $A$ is a set of natural numbers such that $1\in A$ and $n\in A\implies S(n)\in A$, then $A$ contains all natural numbers.
\end{enumerate}
These are known as \vocab{Peano's axioms}.

\begin{theorem}[Archimedean property of $\NN$]
$\NN$ is not bounded above.
\end{theorem}

\begin{proof}
Suppose, for a contradiction, that $\NN$ is bounded above. Then $\NN$ is non-empty and bounded above, so by completeness (of $\RR$) $\NN$ has a supremum.

By the Approximation property with $\epsilon=\frac{1}{2}$, there is a natural number $n\in\NN$ such that $\sup\NN-\frac{1}{2}<n\le\sup\NN$.

Now $n+1\in\NN$ and $n+1>\sup\NN$. This is a contradiction.
\end{proof}
\pagebreak

\section{Integers}
\begin{definition}
For $(a,b),(c,d)\in\NN\times\NN$, we define a relation
\[(a,b)\sim(c,d)\iff a+d=b+c.\]
\end{definition}

\begin{proposition}
$\sim$ is an equivalence relation on $\NN\times\NN$.
\end{proposition}

\begin{proof}
Suppose $(a,b),(c,d),(e,f)\in\NN\times\NN$.
\begin{enumerate}[label=(\roman*)]
\item $\sim$ is reflexive: $(a,b)\sim(a,b)$ because $a+b=b+a$ in $\NN$, by commutativity in $\NN$.
\item $\sim$ is symmetric: If $(a,b)\sim(c,d)$, then $(c,d)\sim(a,b)$ because if $a+d=b+c$, then $c+b=d+a$ in $\NN$.
\item $\sim$ is transitive: 
\end{enumerate}
\end{proof}

%https://www.math.wustl.edu/~freiwald/310integers.pdf
\pagebreak

\section{Rational Numbers}
\begin{notation}
$\ZZ^\prime=\ZZ\setminus\{0\}$.
\end{notation}

\begin{definition}
Let $\sim$ be the binary relation defined on $\ZZ\times\ZZ^\prime$ by
\[ (a,b)\sim(c,d) \iff ad=bc. \]
\end{definition}

\begin{proposition}
$\sim$ is an equivalence on $\ZZ\times\ZZ^\prime$.
\end{proposition}

\begin{proof}
We just check that $\sim$ is transitive. So suppose that $(a,b)\sim(c,d)$ and $(c,d)\sim(e,f)$. Then
\begin{equation*}\tag{1}
ad=bc
\end{equation*}
\begin{equation*}\tag{2}
cf=de
\end{equation*}
Multiplying (1) by $f$ and (2) by $b$, we obtain
\begin{equation*}\tag{3}
adf=bcf
\end{equation*}
\begin{equation*}\tag{4}
bcf=bde
\end{equation*}
Hence $adf=bde$. Since $d\neq0$, the Cancellation Law implies that $af=bc$. Hence $(a,b)\sim(e,f)$.
\end{proof}

\begin{definition}
The set of \vocab{rational numbers} is defined by
\[\QQ\coloneqq\ZZ\times\ZZ^\prime/\sim\]
i.e. $\QQ$ is the set of $\sim$ equivalence classes.
\end{definition}

\begin{notation}
For each $(a,b)\in\ZZ\times\ZZ^\prime$, the corresponding equivalence class is denoted by $[(a,b)]$.
\end{notation}

We define addition $+_\QQ$ and multiplication $\cdot_\QQ$ on $\QQ$ as follows:
\[[(a,b)]+_\QQ[(c,d)]=[(ad+bc,bd)].\]
\[[(a,b)]\cdot_\QQ[(c,d)]=[(ac,bd)].\]

\begin{proposition}
$+_\QQ$ and $\cdot_\QQ$ are well-defined.
\end{proposition}

\begin{lemma}
$\QQ$ is a field, with addition and multiplication as defined above.
\end{lemma}

\begin{proof}
We check the field axioms.
\begin{enumerate}[label=(\roman*)]
\item commutativity of addition
\item associativity of addition
\item Let $0_\QQ=[(0,1)]$. We now show that $0_\QQ$ is an additive identity.

Let $q=[(a,b)]$. Then
\begin{align*}
q+_\QQ 0_\QQ&=[(a,b)]+_\QQ[(0,1)]\\
&=[(a\cdot1+0\cdot b,b\cdot1)]\\
&=[(a,b)]\\
&=q.
\end{align*}
Since for any $q\in\QQ$, $q+_\QQ0_\QQ=q$, thus $0_\QQ$ is an additive identity. Hence an additive identity exists.

\item Consider $r=[(-a,b)]$. Then
\begin{align*}
q+_\QQ r&=[(a,b)]+_\QQ[(-a,b)]\\
&=[(ab+(-a)b,b^2)]\\
&=[(0,b^2)]
\end{align*}
Since $0\cdot1=0\cdot b^2$, we have $(0,b^2)=(0,1)$. Hence 
\begin{align*}
q+_\QQ r&=[(0,b^2)]\\
&=[(0,1)]\\
&=0_\QQ
\end{align*}
Since for any $q\in\QQ$, there exists a unique $r\in\QQ$ such that $q+_\QQ r=0_\QQ$, hence the additive inverse exists.

\item commutativity of multiplication

We want to show that for all $q,r\in\QQ$, $q\cdot_\QQ r=r\cdot_\QQ q$.

\item associativity of multiplication

We want to show that for all $q,r\in\QQ$, $(q\cdot_\QQ r)\cdot_\QQ s=q\cdot_\QQ(r\cdot_\QQ s)$.

\item distributivity

We want to show that for all $q,r,s\in\QQ$, $q\cdot_\QQ(r+_\QQ s)=(q\cdot_\QQ r)+_\QQ(q\cdot_\QQ s)$.

\item Let $1_\QQ=[(1,1)]$. We now show that $1_\QQ$ is a multiplicative identity.

Let $q=[(a,b)]$. Then
\begin{align*}
q\cdot_\QQ1_\QQ
&=[(a,b)]\cdot_\QQ[(1,1)]\\
&=[(a\cdot1,b\cdot1)]\\
&=[(a,b)]\\\
&=1_\QQ
\end{align*}

Since for all $q\in\QQ$, $q\cdot_\QQ1_\QQ=q$, $1_\QQ$ is a multiplicative identity. Hence a multiplicative identity exists.

\item Suppose that $q=[(a,b)]\neq[(0,1)]$. Then $a\neq0$ and so $(b,a)\in\ZZ\times\ZZ^\prime$. Let $r=[(b,a)]$. Then
\begin{align*}
q\cdot_\QQ r
&=[(a,b)]\cdot_\QQ[(b,a)]\\
&=[(ab,ba)]\\
&=[(1,1)]\\
&=1_\QQ.
\end{align*}
\end{enumerate}
Since for every $0_\QQ\neq q\in\QQ$, there exists a unique $r\in\QQ$ such that $q\cdot_\QQ r=1_\QQ$, thus $r$ is a multiplicative inverse. Hence a multiplicative inverse exists.
\end{proof}

Since $\QQ$ is a field, we have the following results:
\begin{enumerate}[label=(\arabic*)]
\item The additive identity in $\QQ$ is unique.
\item The additive inverse of an element of $\QQ$ is unique.
\item The multiplicative identity of $\QQ$ is unique.
\item The multiplicative inverse of a nonzero element of $\QQ$ is unique.
\end{enumerate}

\begin{notation}
Since the additive inverse is unique, we denote the additive inverse of $q\in\QQ$ by $-q$; we define the binary operation $-_\QQ$ on $\QQ$ by
\[q-_\QQ r=q+_\QQ(-r).\]
\end{notation}

\begin{notation}
Since the multiplicative inverse is unique, we denote the additive inverse of $q\in\QQ$ by $q^{-1}$.
\end{notation}

Finally we want to define an order relation on $\QQ$.
\begin{definition}[Order on $\QQ$]
Suppose that $r,s\in\QQ$ and that $r=[(a,b)]$ and $s=[(c,d)]$, where $b,d>0$. Then
\[r\le_\QQ s\iff ad<bc.\]
\end{definition}

\begin{proposition}
$<_\QQ$ is well-defined.
\end{proposition}

\begin{definition}
If $q\in\QQ$, then
\begin{itemize}
\item $q$ is \vocab{positive} if and only if $0_\QQ<_\QQ q$,
\item $q$ is \vocab{negative} if and only if $q<_\QQ0_\QQ$.
\end{itemize}
\end{definition}

\begin{definition}
If $q\in\QQ$, then the \vocab{absolute value} of $q$ is
\[|q|=\begin{cases}
-q&\text{if $q$ is negative,}\\
q&\text{if otherwise.}
\end{cases}\]
\end{definition}
\pagebreak
\end{comment}

\section{Real Numbers}
$\QQ$ has some problems, the first of which being \emph{algebraic incompleteness}: there exists equations with coefficients in $\QQ$ but do not have solutions in $\QQ$ (in fact $\RR$ has this problem too, but $\CC$ is algebraically complete, by the Fundamental Theorem of Algebra).

\begin{lemma*}
$x^2-2=0$ has no solution in $\QQ$.
\end{lemma*}

\begin{proof}
Suppose, for a contradiction, that $x^2-2=0$ has a solution $x=\frac{p}{q}$, $q\neq0$. We also assume $\frac{p}{q}$ is in lowest terms; that is, $p,q$ are coprime. Squaring both sides gives $\frac{p^2}{q^2}=2$, or $p^2=2q^2$. Observe that $p^2$ is even, so $p$ is even; let $p=2m$ for some integer $m$. Then this implies $4m^2=2q^2$, or $2m^2=q^2$. Similarly, $q^2$ is even so $q$ is even.

Since $p$ and $q$ share a common factor of $2$, we have reached a contradiction.
\end{proof}

The second problem is \emph{analytic incompleteness}: there exists a sequence of rational numbers that approach a point that is not in $\QQ$; for example, the sequence
\[1,1.4,1.41,1.414,1.4142,\dots\]
tends to the the irrational number $\sqrt{2}$.

Continuing from the above lemma,
\begin{lemma*}
Let
\begin{align*}
A&=\{p\in\QQ\mid p>0,p^2<2\},\\
B&=\{p\in\QQ\mid p>0,p^2>2\}.
\end{align*}
Then $A$ contains no largest number, and $B$ contains no smallest number.
\end{lemma*}

\begin{proof}
Prove by construction. We associate with each rational $p>0$ the number
\[q=p-\frac{p^2-2}{p+2}=\frac{2p+2}{p+2}\]
and so
\[q^2-2=\frac{2(p^2-2)}{(p+2)^2}.\]

For any $p\in A$, $q>p$ and $q\in A$ since $q^2<2$, so $A$ has no largest number.

For any $p\in B$, $q<p$ and $q\in B$ since $q^2>2$, so $B$ has no smallest number.
\end{proof}

A direct consequence of this is that $\QQ$ does not have the least-upper-bound property, for $A\subset\QQ$ is bounded above but $A$ has no least upper bound in $\QQ$ [$B$ is the set of all upper bounds of $A$, and $B$ does not have a smallest element].

\subsection{Real Field}
\begin{definition}[Ordered field]
A field $F$ is an \vocab{ordered field} if there eists an order $<$ on $F$ such that for all $x,y,z\in F$,
\begin{enumerate}[label=(\roman*)]
\item if $y<z$ then $x+y<x+z$;
\item if $x>0$ and $y>0$ then $xy>0$.
\end{enumerate}
\end{definition}

\begin{proposition}[Basic properties]
The following statements are true in every ordered field.
\begin{enumerate}[label=(\arabic*)]
\item If $x>0$ then $-x<0$, and vice versa.
\item If $x>0$ and $y<z$ then $xy<xz$.
\item If $x<0$ and $y<z$ then $xy>xz$.
\item If $x\neq0$ then $x^2>0$. In particular, $1>0$.
\item If $0<x<y$ then $0<\frac{1}{y}<\frac{1}{x}$.
\end{enumerate}
\end{proposition}

\begin{theorem}[Existence of real field]
There exists an ordered field $\RR$ that
\begin{enumerate}[label=(\roman*)]
\item contains $\QQ$ as a subfield, and
\item has the least-upper-bound property (also known as the completeness axiom).
\end{enumerate}
\end{theorem}

\begin{proof}
We prove by contruction, as follows. 
\end{proof}

One method to construct $\RR$ from $\QQ$ is Dedekind cuts.

\begin{definition*}[Dedekind cut]
A \vocab{Dedekind cut} $\alpha\subset\QQ$ satisfies the following properties:
\begin{enumerate}[label=(\roman*)]
\item $\alpha\neq\emptyset$, $\alpha\neq\QQ$;
\item if $p\in\alpha$, $q\in\QQ$ and $q<p$, then $q\in\alpha$;
\item if $p\in\alpha$, then $p<r$ for some $r\in\alpha$.
\end{enumerate}
\end{definition*}

Note that (iii) simply says that $\alpha$ has no largest member; (ii) implies two facts which will be used freely:
\begin{itemize}
\item If $p\in\alpha$ and $q\notin\alpha$ then $p<q$.
\item If $r\notin\alpha$ and $r<s$ then $s\notin\alpha$.
\end{itemize}

\begin{example}
Let $r\in\QQ$ and define
\[ \alpha_r\coloneqq\{p\in\QQ\mid p<r\}. \]
We now check that this is indeed a Dedekind cut.
\begin{enumerate}[label=(\roman*)]
\item $p=1+r\notin\alpha_r$ thus $\alpha_r\neq\QQ$. $p=r-1\in\alpha_r$ thus $\alpha_r\neq\emptyset$.

\item Suppose that $q\in\alpha_r$ and $q^\prime<q$. Then $q^\prime<q<r$ which implies that $q^\prime<r$ thus $q^\prime\in\alpha_r$.

\item Suppose that $q\in\alpha_r$. Consider $\dfrac{q+r}{2}\in\QQ$ and $q<\dfrac{q+r}{2}<r$. Thus $\dfrac{q+r}{2}\in\alpha_r$.
\end{enumerate}
\end{example}

This example shows that every rational $r$ corresponds to a Dedekind cut $\alpha_r$.

\begin{example}
$\sqrt[3]{2}$ is not rational, but it is real. $\sqrt[3]{2}$ corresponds to the cut
\[ \alpha=\{p\in\QQ\mid p^3<2\}. \]
\begin{enumerate}[label=(\roman*)]
\item Trivial.
\item If $q<p$, by the monotonicity of the cubic function, this implies that $q^3<p^3<2$ thus $q\in\alpha$.
\item If $p\in\alpha$, consider $\brac{p+\frac{1}{n}}^3<2$.
\end{enumerate}
\end{example}

\begin{definition*}
The set of real numbers, denoted by $\RR$, is the set of all Dedekind cuts:
\[\RR\coloneqq\{\alpha\mid\alpha\text{ is a Dedekind cut}\}.\]
\end{definition*}

\begin{proposition*}
$\RR$ has an order, where $\alpha<\beta$ is defined to mean that $\alpha\subset\beta$.
\end{proposition*}

\begin{proof}
Let us check if this is a valid order (check for transitivity and trichotomy).
\begin{enumerate}[label=(\roman*)]
\item For $\alpha,\beta,\gamma\in\RR$, if $\alpha<\beta$ and $\beta<\gamma$ it is clear that $\alpha<\gamma$. (A proper subset of a proper subset is a proper subset.)

\item It is clear that at most one of the three relations
\[ \alpha<\beta, \quad \alpha=\beta, \quad \beta<\alpha \]
can hold for any pair $\alpha,\beta$. 

To show that at least one holds, assume that the first two fail. Then $\alpha$ is not a subset of $\beta$. Hence there exists some $p\in\alpha$ with $p\in\beta$.

If $q\in\beta$, it follows that $q<p$ (since $p\notin\beta$), hence $q\in\alpha$, by (ii). Thus $\beta\subset\alpha$. Since $\beta\neq\alpha$, we conclude that $\beta<\alpha$.
\end{enumerate}
Thus $\RR$ is an ordered set.
\end{proof}

\begin{proposition*}
The ordered set $\RR$ has the least-upper-bound property.
\end{proposition*}

\begin{proof}
Let $A\neq\emptyset$, $A\subset\RR$. Assume that $\beta\in\RR$ is an upper bound of $A$.

Define $\beta$ to be the union of all $\alpha\in A$; in other words, $p\in\gamma$ if and only if $p\in\alpha$ for some $\alpha\in A$. We shall prove that $\gamma\in\RR$ by checking the definition of Dedekind cuts:
\begin{enumerate}[label=(\roman*)]
\item Since $A$ is not empty, there exists an $\alpha_0\in A$. This $\alpha_0$ is not empty. Since $\alpha_0\subset\gamma$, $\gamma$ is not empty.

Next, $\gamma\subset\beta$ (since $\alpha\subset\beta$ for every $\alpha\in A$), and therefore $\gamma\neq\QQ$.

\item Pick $p\in\gamma$. Then $p\in\alpha_1$ for some $\alpha_1\in A$. If $q<p$, then $q\in\alpha_1$, hence $q\in\gamma$.

\item If $r\in\alpha_1$ is so chosen that $r>p$, we see that $r\in\gamma$ (since $\alpha_1\subset\gamma$).
\end{enumerate}

Next we prove that $\gamma=\sup A$.
\begin{enumerate}[label=(\roman*)]
\item It is clear that $\alpha\le\gamma$ for every $\alpha\in A$.
\item Suppose $\delta<\gamma$. Then there is an $s\in\gamma$ and that $s\notin\delta$. Since $s\in\gamma$, $s\in\alpha$ for some $\alpha\in A$. Hence $\delta<\alpha$, and $\delta$ is not an upper bound of $A$.
\end{enumerate}
\end{proof}

\begin{definition*}[Addition]
Given $\alpha,\beta\in\RR$, addition is defined as
\[\alpha+\beta\coloneqq\{r\in\QQ\mid r=a+b,a\in\alpha,b\in\beta\}.\]
\end{definition*}

\begin{proposition*}[Addition on $\RR$ is closed]
For all $\alpha,\beta\in\RR$, $\alpha+\beta\in\RR$.
\end{proposition*}

\begin{proof}
We check that $\alpha+\beta$ is a Dedekind cut:
\begin{enumerate}[label=(\roman*)]
\item $\alpha\neq\emptyset$ and $\beta\neq\emptyset$ implies there exists $a\in\alpha$ and $b\in\beta$. Hence $r=a+b\in\alpha+\beta$ so $\alpha+\beta\neq\emptyset$.

Since $\alpha\neq\QQ$ and $\beta\neq\QQ$, there is $c\neq\alpha$ and $d\neq\beta$. $r^\prime=c+d>a+b$ for any $a\in\alpha,b\in\beta$, so $r^\prime\notin\alpha+\beta$. Hence $\alpha+\beta\neq\QQ$.

\item Suppose that $r\in\alpha+\beta$ and $r^\prime<r$. We want to show that $r^\prime\in\alpha+\beta$.

$r=a+b$ for some $a\in\alpha,b\in\beta$. $r^\prime-a<b$. Since $\beta\in\RR$, $r^\prime-a\in\beta$ so $r^\prime-a=b_1$ for some $b_1\in\beta$. Hence $r^\prime=a+b_1\in\alpha+\beta$.

\item Suppose $r\in\alpha+\beta$, so $r=a+b$ for some $a\in\alpha,b\in\beta$. There exists $a^\prime\in\alpha,b^\prime\in\beta$ with $a<a^\prime$ and $b<b^\prime$. Then $r=a+b<a^\prime+b^\prime\in\alpha+\beta$. We define $r^\prime=a^\prime+b^\prime\in\alpha+\beta$ with $r<r^\prime$.
\end{enumerate}
\end{proof}

\begin{proposition*} \
\begin{enumerate}[label=(\arabic*)]
\item Addition on $\RR$ is commutative:
$\forall\alpha,\beta\in\RR$, $\alpha+\beta=\beta+\alpha$.
\item Addition on $\RR$ is associative:
$\forall\alpha,\beta,\gamma\in\RR$, $\alpha+(\beta+\gamma)=(\alpha+\beta)+\gamma$.
\item Define $0^*\coloneqq\{p\in\QQ\mid p<0\}$. Then $\alpha+0^*=\alpha$.
\item Fix $\alpha\in\RR$, define $\beta=\{p\in\QQ\mid\exists r>0\suchthat-p-r\notin\alpha\}$. Then $\alpha+\beta=0^*$
\end{enumerate}
\end{proposition*}

\begin{proof} \
\begin{enumerate}[label=(\arabic*)]
\item We need to show that $\alpha+\beta\subseteq\beta+\alpha$ and $\beta+\alpha\subseteq\alpha+\beta$.

Let $r\in\alpha+\beta$. Then $r=a+b$ for $a\in\alpha$ and $b\in\beta$. Thus $r=b+a$ since $+$ is commutative on $\QQ$. Hence $r\in\beta+\alpha$. Therefore $\alpha+\beta\subseteq\beta+\alpha$.

Similarly, $\beta+\alpha\subseteq\alpha+\beta$.

Therefore $\alpha+\beta=\beta+\alpha$.

\item Let $r\in\alpha+(\beta+\gamma)$. Then $r=a+(b+c)$ where $a\in\alpha,b\in\beta,c\in\gamma$. Thus $r=(a+b)+c$ by associativity of $+$ on $\QQ$. Therefore $r\in(\alpha+\beta)+\gamma$, hence $\alpha+(\beta+\gamma)\subseteq(\alpha+\beta)+\gamma$.

Similarly, $(\alpha+\beta)+\gamma\subseteq\alpha+(\beta+\gamma)$.

\item Let $r\in\alpha+0^*$. Then $r=a+p$ for some $a\in\alpha,p\in0^*$. Thus $r=a+p<a+0=a$ by ordering on $\QQ$ and identity on $\QQ$. Hence $\alpha+0^*\subseteq\alpha$.

Let $r\in\alpha$. Then there exists $r^\prime>p$ where $r^\prime\in\alpha$. Thus $r-r^\prime<0$, so $r-r^\prime\in0^*$. We see that
\[ r=\underbrace{r^\prime}_{\in\alpha}+\underbrace{(r-r^\prime)}_{\in0^*}. \]
Hence $\alpha\subseteq\alpha+0^*$.

\item We first need to show that $\beta$ is a Dedekind cut.
\begin{enumerate}[label=(\roman*)]
\item If $s\notin\alpha$ and $p=-s-1$, then $-p-1\notin\alpha$, hence $p\in B$, so $\beta\neq\emptyset$. If $q\in\alpha$, then $-q\notin\beta$ so $\beta\neq\QQ$.
\item Pick $p\in\beta$ and pick $r>0$ such that $-p-r\notin\alpha$. If $q<p$, then $-q-r>-p-r$, hence $-q-r\notin\alpha$. Thus $q\in\beta$.
\item Put $t=p+\frac{r}{2}$. Then $t>p$, and $-t-\frac{r}{2}=-p-r\notin\alpha$, so $t\in\beta$.
\end{enumerate}
Hence $\beta\in\RR$.

If $r\in\alpha$ and $s\in\beta$, then $-s\notin\alpha$, hence $r<-s$ so $r+s<0$. Thus $\alpha+\beta\subset0^*$.

To prove the opposite inclusion, pick $v\in0^*$, put $w=-\frac{v}{2}$. Then $w>0$, and there exists $n\in\NN$ such that $nw\in\alpha$ but $(n+1)w\notin\alpha$, by the Archimedean property on $\QQ$. Put $p=-(n+2)w$. Then $p\in\beta$, since $-p-w\notin\alpha$, and
\[v=nw+p\in\alpha+\beta.\]
Thus $0*\subset\alpha+\beta$. We conclude that $\alpha+\beta=0^*$.

This $\beta$ will of course be denoted by $-\alpha$.
\end{enumerate}
\end{proof}

We say that a Dedekind cut $\alpha$ is \emph{positive} if $0\in\alpha$ and negative if $0\notin\alpha$. If $\alpha$ is neither positive nor negative, then $\alpha=0^*$.

Multiplication is a little more bothersome than addition in the present context, since products of negative rationals are positive. For this reason we confine ourselves first to $\RR^+$, the set of all $\alpha\in\RR$ with $\alpha>0*$.

For all $\alpha,\beta\in\RR^+$, we define multiplication as
\[\alpha\cdot\beta\coloneqq\{p\in\QQ\mid p\le ab,a\in\alpha,b\in\beta,a,b>0\}.\]

We define $1^*\coloneqq\{q\in\QQ\mid q<1\}$.



\begin{proposition*}[Multiplication on $\RR$ is closed]
For all $\alpha,\beta\in\RR$, $\alpha\cdot\beta\in\RR$.
\end{proposition*}

\begin{proof} \
\begin{enumerate}[label=(\roman*)]
\item $\alpha\neq\emptyset$ means there exists $a\in\alpha,a>0$. Similarly, $\beta\neq\emptyset$ means there exists $b\in\beta,b>0$. Then $a\cdot b\in\QQ$ and $ab\le ab$, so $ab\in\alpha\cdot\beta\neq\emptyset$.

$\alpha\neq\QQ$ means there exists $a^\prime\notin\alpha,a^\prime>a$ for all $a\in\alpha$. $\beta\neq\QQ$ means there exists $b^\prime\in\beta,b^\prime>b$ for all $b\in\beta$. Then $a^\prime b^\prime>ab$ for all $a\in\alpha,b\in\beta$, so $a^\prime b^\prime\notin\alpha\cdot\beta$, thus $\alpha\cdot\beta\neq\QQ$.

\item $p<\alpha\cdot\beta$ means $p\le a\cdot b$ for some $a\in\alpha,b\in\beta,a,b>0$.

For $q<p$, $q<p\le a\cdot b$ so $q\in\alpha\cdot\beta$.

\item $p\in\alpha\cdot\beta$ means $p\le a\cdot b$ for some $a\in\alpha,b\in\beta,a,b>0$. Pick $a^\prime\in\alpha$ and $b^\prime\in\beta$ with $a^\prime>a$ and $b^\prime>b$. Form $a^\prime b^\prime>ab\ge p$, $a^\prime b^\prime\le a^\prime b^\prime$ means $a^\prime b^\prime\in\alpha\cdot\beta$.
\end{enumerate}
Hence $\alpha\cdot\beta$ is a Dedekind cut.
\end{proof}


We complete the definition of multiplication by setting $\alpha0^*=0^*=0^*\alpha$, and by setting
\[\alpha\cdot\beta=\begin{cases}
(-\alpha)(-\beta)&a<0^*,\beta<0^*,\\
-[(-\alpha)\beta]&a<0^*,\beta>0^*,\\
-[\alpha\cdot(-\beta)]&\alpha>0^*,\beta<0^*.
\end{cases}\]
\pagebreak

We now discuss properties of $\RR$.

\begin{theorem}[$\RR$ is archimedian]\label{thrm:r-archimedian}
For any $x\in\RR^+$ and $y\in\RR$, there exists $n\in\NN$ such that $nx>y$.
\end{theorem}

\begin{proof}
Suppose, for a contradiction, that $nx\le y$ for all $n\in\NN$. Then $y$ is an upper bound of $A=\{nx\mid n\in\NN\}$. Since $\RR$ has the least-upper-bound property and $A\subset R$ is bounded above, $M=\sup A\in\RR$.

Consider $M-x$. Since $M-x<M=\sup A$, $M-x$ is not an upper bound of $A$. Then there exists $n_0\in\NN$ such that $M-x\le n_0x$, or $M\le(n_0+1)x$, which is a contradiction.
\end{proof}

\begin{theorem}[$\QQ$ is dense in $\RR$]
For any $x,y\in\RR$ with $x<y$, there exists $p\in\QQ$ such that $x<p<y$.
\end{theorem}

\begin{proof}
Prove by construction.

Since $x<y$, we have $y-x>0$. By the archimedian property, there exists $n\in\NN$ such that
\[n(y-x)>1.\]
Apply the archimedian property again to obtain $m_1,m_2\in\NN$ such that $m_1>nx$ and $m_2>-nx$. Then
\[-m_2<nx<m_1.\]
Hence there exists $m\in\NN$ (with $-m_2\le m\le m_1$) such that
\[m-1\le nx<m.\]
If we combine there inequalities, we obtain
\[nx<m\le1+nx<ny.\]
Since $n>0$, it follows that
\[x<\frac{m}{n}<y.\]
Take $p=\frac{m}{n}$, and we are done.
\end{proof}

\begin{theorem}[$\RR$ is closed under taking roots]
For every $x\in\RR^+$ and every $n\in\NN$, there exists a unique $x\in\RR^+$ so that $y^n=x$.
\end{theorem}

\begin{proof}
That there is at most one such $y$ is clear, since $0<y_1<y_2$ implies $y_1^n<y_2^n$. Let
\[E=\{t\in\RR^+\mid t^n<x\}.\]
We first show that $E$ has a supremum:
\begin{enumerate}[label=(\roman*)]
\item If $t=\frac{x}{1-x}$ then $0\le t<1$. Hence $t^n\le t<x$. Thus $t\in E$, and $E\neq\emptyset$.
\item If $t>1+x$ then $t^n\ge t>x$, so that $t\notin E$. Thus $1+x$ is an upper bound of $E$.
\end{enumerate}
Hence $E$ has a supremum; let $y=\sup E$.

To prove that $y^n=x$ we will show that each of the inequalities $y^n<x$ and $y^n>x$ leads to a contradiction. The identity $b^n-a^n=(b-a)\brac{n^{n-1}+b^{n-2}a+\cdots+a^{n-1}}$ yields the inequality
\[b^n-a^n<(b-a)nb^{n-1}\]
when $0<a<b$.

Assume $y^n<x$. Choose $h$ so that $0<h<1$ and
\[h<\frac{x-y^n}{n(y+1)^{n-1}}.\]
Put $a=y$, $b=y+h$. Then
\[(y+h)^n-y^n<hn(y+h)^{n-1}<hn(y+1)^{n-1}<x-y^n.\]
Thus $(y+h)^n<x$, and $y+h\in E$. Since $y+h>y$, this contradicts the fact that $y$ is an upper bound of $E$.

Now assume $y^n>x$. Put
\[k=\frac{y^n-x}{ny^{n-1}}.\]
Then $0<k<y$. If $t\ge y-k$, we conclude that
\[y^n-t^n\le y^n-(y-k)^n<kny^{n-1}=y^n-x.\]
Thus $t^n>x$, and $t\notin E$. It follows that $y-k$ is an upper bound of $E$. But $y-k<y$, which contradicts the fact that $y$ is the \emph{least} upper bound of $E$.

Hence $y^n=x$, and the proof is complete.
\end{proof}

\begin{notation}
This number $y$ is written $\sqrt[n]{x}$ or $x^\frac{1}{n}$.
\end{notation}

\begin{corollary}
If $a,b\in\RR^+$ and $n\in\NN$, then
\[(ab)^\frac{1}{n}=a^\frac{1}{n}b^\frac{1}{n}.\]
\end{corollary}

\begin{proof}
Put $\alpha=a^\frac{1}{n}$, $\beta=b^\frac{1}{n}$. Then
\[ab=\alpha^n\beta^n=(\alpha\beta)^n\]
since multiplication is commutative. The uniqueness assertion of the above result shows that
\[(ab)^\frac{1}{n}=\alpha\beta=a^\frac{1}{n}b^\frac{1}{n}.\]
\end{proof}

\begin{proposition}
Real numbers can be represented by decimal expansions.
\end{proposition}

\begin{proof}
Let $x\in\RR^+$. Let $n_0$ be the largest integer such that $n_0\le x$. (Note that the existence of $n_0$ depends on the archimedian property of $\RR$.) Having chosen $n_0,n_1,\dots,n_{k-1}$, let $n_k$ be the largest integer such that
\[n_0+\frac{n_1}{10}+\cdots+\frac{n_k}{10^k}\le x.\]
Let
\[E=\crbrac{n_0+\frac{n_1}{10}+\cdots+\frac{n_k}{10^k}\:\bigg|\:k=0,1,2,\dots}.\]
Then $x=\sup E$. The decimal expansion of $x$ is
\[n_0.n_1n_2n_3\cdots.\]
Conversely, for any infinite decimal, $E$ is bounded above, and $n_0.n_1n_2n_3\cdots$ is the decimal expansion of $\sup E$.
\end{proof}

\subsection{Extended real number system}
\begin{definition}[Extended real number system]
We add two symbols $+\infty$ and $-\infty$ to $\RR$, and call the union $\RR\cup\{\pm\infty\}$ the \vocab{extended real number system}. We preserve the original order in $\RR$, and define
\[-\infty<x<+\infty\]
for every $x\in\RR$.
\end{definition}

\begin{proposition}
Any non-empty subset $E$ of the extended real number system has a supremum and infimum.
\end{proposition}

\begin{proof}
If $E$ is bounded above in $\RR$, then we are done. If $E$ is not bounded above in $\RR$, then $\sup E=+\infty$ in the extended real number system.

Exactly the same remarks apply to lower bounds.
\end{proof}

The extended real number system does not form a field, but it is customary to make the following conventions:
\begin{enumerate}[label=(\arabic*)]
\item If $x$ is real then
\[ x+\infty=+\infty, \quad x-\infty=-\infty, \quad \frac{x}{+\infty}=\frac{x}{-\infty}=0. \]
\item If $x>0$ then $x\cdot(+\infty)=+\infty$, $x\cdot(-\infty)=-\infty$.
\item If $x<0$ then $x\cdot(+\infty)=-\infty$, $x\cdot(-\infty)=+\infty$.
\end{enumerate}
When it is desired to make the distinction between real numbers on the one hand and the symbols $+\infty$ and $-\infty$ on the other quite explicit, the former are called \textbf{finite}.

\section{Complex Field}
Consider the Cartesian product
\[\RR^2\coloneqq\RR\times\RR=\{(x_1,x_2)\mid x_1,x_2\in\RR\}.\]
We can define addition and scalar multiplication on $\RR^2$:
\begin{align*}
(x_1,x_2)+(y_1,y_2)&=(x_1+y_1,x_2+y_2),\\
c(x_1,x_2)&=(cx_1,cx_2).
\end{align*}
for all $(x_1,x_2),(y_1,y_2)\in\RR^2$, $c\in\RR$.

\begin{proposition}
$\RR^2$, with addition and scalar multiplication defined as above, is a vector space over $\RR$.
\end{proposition}

For $(x_1,x_2),(y_1,y_2)\in\RR^2$, the \textbf{inner product} is defined as 
\[\langle(x_1,x_2),(y_1,y_2)\rangle\coloneqq x_1y_1+x_2y_2.\]
The inner product induces a \textbf{norm}, defined as follows:
\[|(x_1,x_2)|\coloneqq\langle(x_1,x_2),(x_1,x_2)\rangle^\frac{1}{2}=\brac{{x_1}^2+{x_2}^2}^\frac{1}{2}. \]
\begin{notation}
From now on, we use $\vb{x}$ to denote $(x_1,x_2)$.
\end{notation}

\begin{proposition} \
\begin{enumerate}[label=(\arabic*)]
\item $|\vb{x}|\ge0$, where equality holds if and only if $\vb{x}=\vb{0}$.
\item $|c\vb{x}|=|c||\vb{x}|$
\item $|\vb{x}+\vb{y}|\le|\vb{x}|+|\vb{y}|$
\item $|\langle\vb{x},\vb{y}\rangle|\le|\vb{x}||\vb{y}|$
\end{enumerate}
\end{proposition}

Let $x=(a,b)$, $y=(c,d)$. Over $\RR^2$, we can define multiplication $\cdot$ as
\[(a,b)\cdot(c,d)=(ac-bd,ad+bc)\]
for all $(a,b),(c,d)\in\RR^2$. If we identity $\RR^2$ with
\[\CC\coloneqq\{x+yi\mid x,y\in\RR\}\]
via $(x,y)\mapsto x+yi$, where $i^2=-1$, then all structures defined above are induced to $\CC$.

\begin{proposition}
$(\CC,+,\cdot)$ is a field, with $(0,0)$ and $(1,0)$ in the role of $0$ and $1$.
\end{proposition}

\begin{proof}
Simply check the field axioms.
\end{proof}

We call $\CC$ the \vocab{complex field}. A element in $\CC$ is called a \textbf{complex number}. Usually, a complex number is denoted by $z=x+yi$ where $x,y\in\RR$. Here $x$ is called the \textbf{real part} of $z$, denoted by $x=\Re(z)$; $y$ is called the \textbf{imaginary part} of $z$, denoted by $y=\Im(z)$. The norm of $z$ is denoted by $|z|$. For $z=x+yi$, the \textbf{conjugate} of $z$ is $\bar{z}=x-yi$.

\begin{proposition}
For $z,w\in\CC$,
\begin{enumerate}[label=(\arabic*)]
\item $\overline{z+w}=\bar{z}+\bar{w}$
\item $\overline{zw}=\bar{z}\bar{w}$
\item $z+\bar{z}=2\Re(z)$, $z-\bar{z}=2i\Im(z)$
\item $z\bar{z}\in\RR$ and $z\bar{z}>0$ (except when $z=0$)
\end{enumerate}
\end{proposition}

\begin{proposition}
For $z,w\in\CC$,
\begin{enumerate}[label=(\arabic*)]
\item $z>0$ unless $z=0$, $|0|=0$
\item $|\bar{z}|=|z|$
\item $|zw|=|z||w|$
\item $|\Re(z)|\le|z|$
\item $|z+w|\le|z|+|w|$
\end{enumerate}
\end{proposition}

\begin{theorem}[Schwarz inequality]
If $a_1,\dots,a_n,b_1,\dots,b_n\in\CC$, then
\begin{equation}
\absolute{\sum_{i=1}^{n}a_ib_i}^2\le\sum_{i=1}^{n}|a_i|^2\sum_{i=1}^{n}|b_i|^2.
\end{equation}
\end{theorem}

\begin{proof}
Let $A=\sum|a_i|^2$, $B=\sum|b_i|^2$, $C=\sum a_i\bar{b_i}$. If $B=0$, then $b_1=\cdots=b_n=0$, and the conclusion is trivial. Assume therefore that $B>0$. Then we have
\begin{align*}
\sum|Ba_i-Cb_i|^2
&=\sum(Ba_i-Cb_i)(B\bar{a_i}-\overline{Cb_i})\\
&=B^2\sum|a_i|^2-B\bar{C}\sum a_i\bar{b_j}-BC\sum\bar{a_i}b_i+|C|^2\sum|b_i|^2\\
&=B^2A-B|C|^2\\
&=B(AB-|C|^2).
\end{align*}
Since each term in the first sum is non-negative, we see that
\[B(AB-|C|^2)\ge0.\]
Since $B>0$, it follows that $AB-|C|^2\ge0$. This is the desired inequality.
\end{proof}

(when does equality hold?)

\section{Euclidean Spaces}
For $n\in\ZZ^+$, 
\[\RR^n=\{(x_1,\dots,x_n)\mid x_i\in\RR\}\]
where $\vb{x}=(x_1,\dots,x_n)$, $x_i$'s are called the coordinates of $\vb{x}$. The elements of $\RR^n$ are called points, or vectors. Addition and scalar multiplication on $\RR^n$ defined as follows: for $\vb{x},\vb{y}\in\RR^n$, $\alpha\in\RR$,
\begin{align*}
\vb{x}+\vb{y}&=(x_1+y_1,\dots,x_n+y_n)\\
\alpha\vb{x}&=(\alpha x_1,\dots,\alpha x_n)
\end{align*}
These two operations satisfy the commutative, associatives, and distributive laws (the proof is trivial, in view of the analagous laws for the real numbers) and make $\RR^n$ into a vector space over the real field. The zero element of $\RR^n$ (sometimes called the origin or the null vector) is the point $\vb{0}$, all of whose coordinates are $0$.

We define the \textbf{inner product} (or scalar product) of $\vb{x}$ and $\vb{y}$ by
\[\vb{x}\cdot\vb{y}\coloneqq\sum_{i=1}^nx_iy_i.\]
The \textbf{norm} is a real-valued function $\norm{\cdot}:\RR^n\to\RR$; given $\vb{x}=(x_1,\dots,x_n)$, the norm of $\vb{x}$ is defined as
\[\norm{\vb{x}}\coloneqq\brac{\vb{x}\cdot\vb{x}}^\frac{1}{2}=\brac{\sum_{i=1}^n{x_i}^2}^\frac{1}{2}.\]
The structure now defined (the vector space $\RR^n$ with the above inner product and norm) is called the \vocab{Euclidean $n$-space}.

\begin{proposition}
Suppose $\vb{x},\vb{y},\vb{z}\in\RR^n$, $\alpha\in\RR$. Then
\begin{enumerate}[label=(\arabic*)]
\item $\norm{\vb{x}}\ge0$
\item $\norm{\vb{x}}=0$ if and only if $\vb{x}=\vb{0}$
\item $\norm{\alpha\vb{x}}=|\alpha|\norm{\vb{x}}$
\item $\norm{\vb{x}\cdot\vb{y}}\le\norm{\vb{x}}\norm{\vb{y}}$
\item $\norm{\vb{x}+\vb{y}}\le\norm{\vb{x}}+\norm{\vb{y}}$
\item $\norm{\vb{x}-\vb{z}}\le\norm{\vb{x}-\vb{y}}+\norm{\vb{y}-\vb{z}}$
\end{enumerate}
\end{proposition}

\begin{proof} \
\begin{enumerate}[label=(\arabic*)]
\item 
\item 
\item 
\item This is an immediate consequence of the Cauchy--Schwarz inequality.
\item By (4) we have
\begin{align*}
\norm{\vb{x}+\vb{y}}&=(\vb{x}+\vb{y})\cdot(\vb{x}+\vb{y})\\
&=\vb{x}\cdot\vb{x}+2\vb{x}\cdot\vb{y}+\vb{y}\cdot\vb{y}\\
&\le\norm{\vb{x}}^2+2\norm{\vb{x}}\norm{\vb{y}}+\norm{\vb{y}}^2\\
&=\brac{\norm{\vb{x}}+\norm{\vb{y}}}^2.
\end{align*}
\item This follows directly from (5) by replacing $\vb{x}$ by $\vb{x}-\vb{y}$ and $\vb{y}$ by $\vb{y}-\vb{z}$.
\end{enumerate}
\end{proof}

\begin{comment}
\begin{definition}
The \vocab{distance between sets} $E\subset\RR^n$ and $F\subset\RR^n$ is defined as
\[ d(E,F)\coloneqq\inf_{x\in E,y\in F}\norm{x-y}. \]
\end{definition}

Obviously $d(E,F)>0$ implies that $E$ and $F$ are disjoint, but $E$ and $F$ may still be disjoint even if $d(E,F)=0$. For example, the closed intervals $E=(-1,0)$ and $F=(0,1)$.

\begin{exercise}
Suppose that $E$ and $F$ are sets in $\RR^n$ where $E$ and $F$ is finite. Prove that $E$ and $F$ are disjoint if and only if $d(E,F)>0$.
\end{exercise}
\end{comment}
\pagebreak

\section*{Exercises}
\begin{prbm}
If $r\neq0$ is rational and $x$ is irrational, prove that $r+x$ and $rx$ are irrational.
\end{prbm}

\begin{solution}
We prove by contradiction. Suppose $r+x$ is rational, then $r+x=\dfrac{m}{n},m,n\in\ZZ$, and $m,n$ have no common factors. Then $m=n(r+x)$. Let $r=\frac{p}{q},p,q\in\ZZ$, the former equation implies that $m=n\brac{\frac{p}{q}+x}$, i.e., $qm=n(p+qx)$, giving
\[x=\frac{mq-np}{nq},\]
which says that $x$ can be written as the quotient of two integers, so $x$ is rational, a contradiction.

The proof for the case $rx$ is similar.
\end{solution}

\begin{prbm}
Prove that there is no rational number whose square is $12$.
\end{prbm}

\begin{solution}
If $r\in\QQ$, $r^2=12$, then $\brac{\frac{r}{2}}^2=3$, so this is equivalent to showing there is no rational number whose square is $3$. The proof is analagous to that of proving $\sqrt{2}$ is irrational.
\end{solution}

\begin{prbm}
Let $E$ be a nonempty subset of an ordered set; suppose $\alpha$ is a lower bound of $E$ and $\beta$ is an upper bound of $E$. Prove that $\alpha\le\beta$.
\end{prbm}

\begin{solution}
Let $x\in E$. By definition of lower and upper bounds, $\alpha\le x\le\beta$.
\end{solution}

\begin{prbm}
Let $A$ be a nonempty set of real numbers which is bounded below. Let $-A$ be the set of all numbers $-x$, where $x\in A$. Prove that
\[\inf A=-\sup(-A).\]
\end{prbm}

\begin{solution}
Let $\alpha=\inf A$. If $x\in(-A)$ then $-x\in A$, so $\alpha\le-x$, and so $-\alpha\le x$. This implies that $-\alpha$ is an upper bound for $-A$.

If $\beta<-\alpha$ then $-\beta>\alpha$, and there exists $x\in A$ such that $x<-\beta$. Then $-x\in(-A)$, and $-x>\beta$. This shows that $-\alpha=\sup(-A)$, and we are done.
\end{solution}

\begin{prbm}
Proe that no order can be defined in the complex field that turns it into an ordered field.
\end{prbm}

\begin{prbm}[Lexicographic order]
Suppose $z=a+bi$, $w=c+di$. Define an order on $\CC$ as follows:
\[z<w\iff\begin{cases}
a<c,\text{ or}\\
a=c,b<d.
\end{cases}\]
Prove that this turns the set of all complex numbers into an ordered set. Does this ordered set have the least uupper bound property?
\end{prbm}

\begin{prbm}
Suppose $z=a+bi$, $w+u+iv$, and
\[a=\brac{\frac{|w|+u}{2}}^\frac{1}{2},\quad b=\brac{\frac{|w|-u}{2}}^\frac{1}{2}.\]
Prove that $z^2=w$ if $v\ge0$ and that $\bar{z}^2=w$ if $v\le0$. Conclude that every complex number (with one exception!) has two complex square roots.
\end{prbm}

\begin{prbm}
If $z$ is a complex number, prove that there exists $r\ge0$ and a complex number $w$ with $|w|=1$ such that $z=rw$. Are $w$ and $r$ always uniquely determined by $z$?
\end{prbm}

\begin{prbm}[Triangle inequality]
If $z_1,\dots,z_n\in\CC$, prove that
\[|z_1+\cdots+z_n|\le|z_1|+\cdots+|z_n|.\]
\end{prbm}

\begin{prbm}
If $x,y\in\CC$, prove that
\[\absolute{|x|-|y|}\le|x-y|.\]
\end{prbm}