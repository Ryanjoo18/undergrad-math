\chapter{Sequence and Series of Functions}
\section{Uniform Convergence}
\begin{definition}[Pointwise convergence]
Suppose $\{f_n\}$, $n=1,2,3,\dots$ is a sequence of functions defined on a set $E$, and suppose that the sequence of numbers $\{f_n(x)\}$ converges for every $x\in E$. We can then define a function $f$ by
\[ f(x)=\lim_{n\to\infty}f_n(x). \]
We say that $\{f_n\}$ \vocab{converges pointwise} to $f$ on $E$, denoted by $f_n\to f$, and $f$ is the \vocab{limit}, or the \vocab{limit function}, of $\{f_n\}$.
\[\forall\epsilon>0,\forall x\in E,\exists N\in\NN\suchthat\forall n>N,|f_n(x)-f(x)|<\epsilon.\]

Similarly, if $\sum f_n(x)$ converges for every $x\in E$, and if we define
\[ f(x)=\sum_{n=1}^\infty f_n(x) \]
the function $f$ is called the \vocab{sum of the series} $\sum f_n$.
\end{definition}

Most properties are not preserved by pointwise continuity; that is, $f$ does not inherit most properties of $f_n$.

\begin{example}[$f_n$ continuous, $f$ discontinuous]
Let $f_n(x)=x^n$ for $x\in[0,1]$. Then
\[f(x)=\lim_{n\to\infty}f_n(x)=\begin{cases}
0&\text{if }x\in(0,1]\\
1&\text{if }x=1
\end{cases}\]
and so the limit function $f(x)$ is discontinuous.
\end{example}

\begin{example}[$f_n$ integrable, $f$ not integrable]
Recall that the Dirichlet function
\[D(x)=\begin{cases}
1&\text{if }x\in\QQ\\
0&\text{if }x\in\RR\setminus\QQ
\end{cases}\]
is not integrable.

\begin{proof}
Consider the interval $[0,1]$. We partition $P:0=x_0<x_1<\cdots<x_n=1$. The sum is given by $\sum_{i=1}^n D(t_i)\Delta x_i$. Then
\[M_i=\max_{t\in[x_{i-1},x_i]}D(t)=1\implies U(D;P)=1\quad\forall P\]
and
\[m_i=\min_{t\in[x_{i-1},x_i]}D(t)=0\implies L(D;P)=0\quad\forall P.\]
Hence 
\[\upperint_0^1 D(x)\dd{x}=1,\quad\lowerint_0^1 D(x)\dd{x}=0\]
so $\upperint_0^1 D(x)\dd{x}\neq\lowerint_0^1 D(x)\dd{x}$, and thus $D(x)$ is not integrable.
\end{proof}

We define a sequence of functions as follows:
\[D_n(x)=\begin{cases}
1&\text{if }x=\frac{p}{q},p\in\ZZ,q\in\ZZ\setminus\{0\},|q|\le n\\
0&\text{if otherwise}
\end{cases}\]

\end{example}

\begin{definition}[Uniform convergence]
We say that $\{f_n\}$ \vocab{uniformly converges} to $f$ over $E$, if 
\[\forall\epsilon>0, \exists N\in\NN\suchthat \forall x\in E,\forall n>N,|f_n(x)-f(x)|<\epsilon.\]
We denote this by $f_n\rightrightarrows f$.
\end{definition}

Uniform convergence is stronger than pointwise convergence, since $N$ is uniform (or ``fixed'') for all $x\in E$; for pointwise convergence, the choice of $N$ is determined by $x$.

\begin{definition}
If $X$ is a metric space, we denote the set of all complex-valued, continuous, bounded functions with domain $X$ by $C(X)$.

If $f\in C(X)$, we define 
\[\norm{f}\coloneqq\sup_{x\in X}|f(x)|,\]
known as the \vocab{suprenum norm} of $f$.
\end{definition}

\begin{lemma}
$\norm{f}$ gives a norm on $C(X)$.
\end{lemma}

\begin{proof}
Check that $\norm{f}$ satisfies the conditions for a norm:
\begin{enumerate}[label=(\arabic*)]
\item 
\end{enumerate}
\end{proof}

\begin{proposition}
$\brac{C(X),\norm{\cdot}}$ is a metric space.
\end{proposition}

\begin{lemma}[Cauchy criterion]
$\{f_n\}\rightrightarrows f$ if and only if
\[\forall\epsilon>0,\exists N\in\NN\suchthat\forall x\in E,\forall n,m>N,|f_n(x)-f_m(x)|<\epsilon.\]
\end{lemma}

\begin{proof} \

($\implies$) Suppose $f_n\rightrightarrows f$, then
\[\forall\epsilon>0, \exists N\in\NN\suchthat \forall x\in E,\forall n>N,|f_n(x)-f(x)|<\frac{\epsilon}{2}.\]
Then for all $n,m>N$,
\begin{align*}
|f_n(x)-f_m(x)|
&=\absolute{\brac{f_n(x)-f(x)}-\brac{f_m(x)-f(x)}}\\
&\le|f_n(x)-f(x)|+|f_m(x)-f(x)|\\
&<\frac{\epsilon}{2}+\frac{\epsilon}{2}=\epsilon
\end{align*}
by triangle inequality.

($\impliedby$) 
\[\forall\epsilon>0,\exists N\in\NN\suchthat\forall x\in E,\forall n,m>N,|f_n(x)-f_m(x)|<\epsilon.\]

\end{proof}

The uniform convergence of series is defined similarly: 

\begin{lemma}[Cauchy criterion]

\end{lemma}

\begin{theorem}[Weierstrass M-test]
$\sum_{n=1}^\infty f_n(x)$ uniformly converges if
\[\exists\{M_n\}\in\RR^+\suchthat|f_n(x)|<M_n,\sum_{n=1}^\infty M_n\text{ convergent}\]
where $\sum_{n=1}^\infty M_n$ is convergent if
\[\forall\epsilon>0,\exists N\in\NN\suchthat\forall n,m>N,\sum_{k=m+1}^n M_k<\epsilon.\]
\end{theorem}

\section{Uniform Convergence and Continuity}
We now consider properties preserved by uniform convergence.



\section{Uniform Convergence and Integration}
\begin{theorem}
Assume $\{f_n\}$ is a sequence of functions defined over $[a,b]$ and each $f_n\in R_\alpha[a,b]$. If $f_n\to f$, then $f\in R_\alpha[a,b]$, and
\[ \lim_{n\to\infty}\int_a^bf_n\dd{\alpha}=\int_a^bf\dd{\alpha}. \]
\end{theorem}

\begin{proof}
Define
\end{proof}

\begin{corollary}
Assume $a_n\in R_\alpha[a,b]$ and
\[ f(x)\coloneqq\sum_{n=0}^\infty a_n(x) \]
converges uniformly. Then it follows
\[ \int_a^bf\dd{\alpha}=\sum_{n=0}^\infty a_n\dd{\alpha}. \]
\end{corollary}

\begin{proof}
Consider the sequence of partial sums 
\[ f_n(x)\coloneqq\sum_{k=0}^na_k(x), \quad n=0,1,\dots \]
It follows $f_n\in R_\alpha[a,b]$ and $f_n\rightrightarrows f$. Apply above theorem to $\{f_n\}$ and the conclusion follows.
\end{proof}

\section{Uniform Convergence and Differentiation}
\begin{theorem}
$\{f_n\}$ differentiable on $[a,b]$, $\exists x_0\in[a,b]\suchthat f_n(x_0)\to y_0=f(x_0)$ and $f_n^\prime\rightrightarrows f^\prime$. Then $f_n\rightrightarrows f$ on $[a,b]$, and $f$ is differentiable, $f^\prime(x)=\lim_{n\to\infty}f_n^\prime(x)$ for any $x\in[a,b]$.
\end{theorem}

\begin{proof}
$f_n(x_0)\to y_0$ thus
\end{proof}

\section{Stone--Weierstrass Approximation Theorem}
\begin{theorem}[Weierstrass approximation theorem]
If $f$ is a continuous complex function on $[a,b]$, there exists a sequence of polynomials $P_n$ such that $P_n\rightrightarrows f$ on $[a,b]$.

If $f$ is real, then $P_n$ may be taken real.
\end{theorem}