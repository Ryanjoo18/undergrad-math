\chapter{Some Special Functions}
\section{Power Series}
\begin{definition}
\vocab{Analytic functions} are functions that can be represented by \vocab{power series}, i.e. functions of the form
\[f(x)=\sum_{n=0}^\infty c_n x^n\]
or, more generally,
\[f(x)=\sum_{n=0}^\infty c_n(x-a)^n.\]
\end{definition}

The \vocab{radius of convergence} is the maximum $R$ such that $f(x)$ converges in $(-R,R)$.

\begin{theorem}
Suppose the series 
\[ \sum_{n=0}^\infty c_nx^n \]
converges for $x\in(-R,R)$. Then
\begin{enumerate}[label=(\arabic*)]
\item $\sum_{n=0}^\infty c_nx^n$ converges uniformly on the closed interval $[-R+\epsilon,R-\epsilon]$ for all $\epsilon>0$;
\item $f(x)$ is continuous and differentiable on $(-R,R)$, and 
\[ f^\prime(x)=\sum_{n=1}^\infty nc_nx^{n-1}. \]
\end{enumerate}
\end{theorem}

\begin{proof} \
\begin{enumerate}[label=(\arabic*)]
\item Let $\epsilon>0$ be given. For $|x|\le R-\epsilon$, we have
\item 
\end{enumerate}
\end{proof}

\begin{corollary}
$f$ has derivatives of all orders in $(-R,R)$, which are given by
\[f^{(k)}(x)=\sum_{n=k}^\infty n(n-1)\cdots(n-k+1)c_nx^{n-k}.\]
In particular,
\[f^{(k)}(0)=k!c_k,\quad k=0,1,2,\dots\]
(Here $f^{(0)}$ means $f$, and $f^{(k)}$ is the $k$-th derivative of $f$, for $k=1,2,3,\dots$)
\end{corollary}

\begin{proof}
Apply theorem successively to $f,f^\prime,f^{\prime\prime},\dots$. Put $x=0$.
\end{proof}

\begin{proposition}
Suppose $\sum c_n$ converges. Put
\[f(x)=\sum_{n=0}^\infty c_n x^n\]
for $x\in(-R,R)$
\end{proposition}

\section{Exponential and Logarithmic Functions}
\begin{definition}[Exponential function]
We define the exponential function as
\begin{equation}
\exp(z)=\sum_{n=0}^\infty\frac{z^n}{n!}.
\end{equation}
\end{definition}

\begin{lemma}
$\exp(z)$ converges for every $z\in\CC$.
\end{lemma}

\begin{proof}
Ratio test.
\end{proof}



\section{Trigonometric Functions}


\section{Algebraic Completeness of the Complex Field}
We now prove that the complex field is \vocab{algebraically complete}; that is, every non-constant polynomial with complex coefficients has a complex root.

\begin{theorem}[Fundamental Theorem of Algebra]
Suppose $a_0,\dots,a_n$ are complex numbers, $n\ge1$, $a_n\neq0$,
\[P(z)=\sum_{k=0}^n a_kz^k.\]
Then $P(z)=0$ for some complex number $z$.
\end{theorem}

\begin{proof}

\end{proof}

\section{Fourier Series}
\begin{definition}
A \vocab{trigonometric polynomial} is a finite sum of the form
\[f(x)=a_0+\sum_{n=1}^\infty(a_n\cos nx+b_n\sin nx)\]
for $x\in\RR$, where $a_0,\dots,a_N,b_1,\dots,b_N\in\CC$.
\end{definition}

On account of the identities (?), we can write the above in the form
\[f(x)=\sum_{n=-N}^N c_ne^{inx}.\]
It is clear that every trigonometric polynomial is periodic, with period $2\pi$.



\section{Gamma Function}
\begin{definition}[Gamma function]
For $0<x<\infty$,
\[\Gamma(x)\coloneqq\int_0^\infty t^{x-1}e^{-t}\dd{t}.\]
The integral converges for these $x$. (When $x<1$, both $0$ and $\infty$ have to be looked at.)
\end{definition}

\begin{lemma} \
\begin{enumerate}[label=(\arabic*)]
\item The functional equation
\[\Gamma(x+1)=x\Gamma(x)\]
holds for $0<x<\infty$.
\item $\Gamma(n+1)=n!$ for $n=1,2,3,\dots$
\item $\log\Gamma$ is convex on $(0,\infty)$.
\end{enumerate}
\end{lemma}

\begin{proof} \
\begin{enumerate}[label=(\arabic*)]
\item Integrate by parts.
\item Since $\Gamma(1)=1$, (1) implies (2) by induction.
\item 
\end{enumerate}
\end{proof}

In fact, these three properties characterise $\Gamma$ completely.

\begin{lemma}[Characteristic properties of $\Gamma$] \label{lemma:gamma-char}
If $f$ is a positive function on $(0,\infty)$ such that
\begin{enumerate}[label=(\arabic*)]
\item $f(x+1)=xf(x)$,
\item $f(1)=1$,
\item $\log f$ is convex,
\end{enumerate}
then $f(x)=\Gamma(x)$.
\end{lemma}

\begin{proof}

\end{proof}

\begin{definition}[Beta function]
For $x>0$ and $y>0$, the beta function is defined as
\[B(x,y)\coloneqq\int_0^1 t^{x-1}(1-t)^{y-1}\dd{t}.\]
\end{definition}

\begin{lemma}
\[B(x,y)=\frac{\Gamma(x)\Gamma(y)}{\Gamma(x+y)}.\]
\end{lemma}

\begin{proof}
Let $f(x)=\dfrac{\Gamma(x+y)}{\Gamma(y)}B(x,y)$. We want to prove that $f(x)=\Gamma(x)$, using \cref{lemma:gamma-char}.
\begin{enumerate}[label=(\arabic*)]
\item \[B(x+1,y)=\int_0^1 t^x(1-t)^{y-1}\dd{t}.\]
Integrating by parts gives
\begin{align*}
B(x+1,y)&=\underbrace{\sqbrac{t^x\cdot\frac{(1-t)^y}{y}(-1)}_0^1}_{0}+\int_0^1 xt^{x-1}\frac{(1-t)^y}{y}\dd{t}\\
&=\frac{x}{y}\int_0^1 t^{x-1}(1-t)^{y-1}(1-t)\dd{t}\\
&=\frac{x}{y}\brac{\int_0^1 t^{x-1}(1-t)^{y-1}\dd{t}-\int_0^1 t^x(1-t)^{y-1}\dd{t}}\\
&=\frac{x}{y}\brac{B(x,y)-B(x+1,y)}
\end{align*}
which gives $B(x+1,y)=\dfrac{x}{x+y}B(x,y)$. Thus
\begin{align*}
f(x+1)&=\frac{\Gamma(x+1+y)}{\Gamma(y)}B(x+1,y)\\
&=\frac{(x+y)B(x+y)}{\Gamma(y)}\cdot\frac{x}{x+y}B(x,y)\\
&=xf(x).
\end{align*}
\item \[B(1,y)=\int_0^1(1-t)^{y-1}\dd{t}=\sqbrac{-\frac{(1-t)^y}{y}}_0^1=\frac{1}{y}\]
and thus
\[f(1)=\frac{\Gamma(1+y)}{\Gamma(y)}B(1,y)=\frac{y\Gamma(y)}{\Gamma(y)}\frac{1}{y}=1.\]
\item We now show that $\log B(x,y)$ is convex, so that
\[\log f(x)=\underbrace{\log\Gamma(x+y)}_\text{convex}+\log B(x,y)-\underbrace{\log\Gamma(y)}_\text{constant}\]
is convex with respect to $x$.
\[B(x_1,y)^\frac{1}{p}B(x_2,y)^\frac{1}{q}=\brac{\int_0^1 t^{x_1-1}(1-t)^{y-1}\dd{t}}^\frac{1}{p}\brac{\int_0^1 t^{x_2-1}(1-t)^{y-1}\dd{t}}^\frac{1}{q}\]
By H\"{o}lder's inequality,
\begin{align*}
B(x_1,y)^\frac{1}{p}B(x_2,y)^\frac{1}{q}
&=\int_0^1\sqbrac{t^{x_1-1}(1-t)^{y-1}}^\frac{1}{p}\sqbrac{t^{x_2-1}(1-t)^{y-1}}^\frac{1}{q}\dd{t}\\
&=\int_0^1 t^{\frac{x_1}{p}+\frac{x_2}{q}-1}(1-t)^{y-1}\dd{t}\\
&=B\brac{\frac{x_1}{p}+\frac{x_2}{q},y}.
\end{align*}
Taking log on both sides gives
\[\log B(x,y)^\frac{1}{p}B(x_2,y)^\frac{1}{q}\ge\log B\brac{\frac{x_1}{p}+\frac{x_2}{q},y}\]
or
\[\frac{1}{p}\log B(x,y)+\frac{1}{q}\log B(x_2,y)\ge\log B\brac{\frac{x_1}{p}+\frac{x_2}{q},y}.\]
Hence $\log B(x,y)$ is convex, so $\log f(x)$ is convex.
\end{enumerate}
Therefore $f(x)=\Gamma(x)$ which implies $B(x,y)=\dfrac{\Gamma(x)\Gamma(y)}{\Gamma(x+y)}$.
\end{proof}

An alternative form of $\Gamma$ is as follows:
\[\Gamma(x)=2\int_0^{+\infty}t^{2x-1}e^{-t^2}\dd{t}.\]
Using this form of $\Gamma$, we present an alternative proof.

\begin{proof}
\begin{align*}
\Gamma(x)\Gamma(y)
&=\brac{2\int_0^{+\infty}t^{2x-1}e^{-t^2}\dd{t}}\brac{2\int_0^{+\infty}s^{2y-1}e^{-s^2}\dd{s}}\\
&=4\iint_{[0,+\infty)\times[0,+\infty)}t^{2x-1}s^{2y-1}e^{-\brac{t^2+s^2}}\dd{t}\dd{s}
\end{align*}
Using polar coordinates transformation, let $t=r\cos\theta$, $s=r\sin\theta$. Then $\dd{t}\dd{s}=r\dd{r}\dd{\theta}$. Thus
\begin{align*}
\Gamma(x)\Gamma(y)
&=4\int_0^\frac{\pi}{2}\sqbrac{\int_0^{+\infty}r^{2x-1}\cos^{2x-1}\theta\cdot r^{2y-1}\sin^{2y-1}\theta\cdot e^{-r^2}\cdot r\dd{r}}\dd{\theta}\\
&=\underbrace{2\int_0^\frac{\pi}{2}\cos^{2x-1}\theta\sin^{2y-1}\theta\dd{\theta}}_{B(x,y)}\cdot\underbrace{2\int_0^{+\infty}r^{2(x+y)-1}e^{-r^2}\dd{r}}_{\Gamma(x+y)}
\end{align*}
since
\begin{align*}
B(x,y)&=\int_0^1 t^{x-1}(1-t)^{y-1}\dd{t}\quad t=\cos^2\theta\\
&=\int_\frac{\pi}{2}^0 \cos^{2(x-1)}\theta\sin^{2(y-1)}\theta\cdot2\cos\theta(-\sin\theta)\dd{\theta}\\
&=2\int_0^\frac{\pi}{2}\cos^{2x-1}\theta\sin^{2y-1}\theta\dd{\theta}.
\end{align*}
Hence $B(x,y)=\dfrac{\Gamma(x)\Gamma(y)}{\Gamma(x+y)}$.
\end{proof}

More on polar coordinates:
\begin{equation}
I=\int_{-\infty}^{+\infty}e^{-x^2}\dd{x}
\end{equation}

\begin{proof}
\begin{align*}
I^2&=\int_{-\infty}^{+\infty}e^{-x^2}\dd{x}\int_{-\infty}^{+\infty}e^{-y^2}\dd{y}\\
&=\iint_{\RR^2}e^{-\brac{x^2+y^2}}\dd{x}\dd{y}\quad x=r\cos\theta,y=r\sin\theta\\
&=\int_0^{2\pi}\underbrace{\int_0^{+\infty}e^{-r^2}r\dd{r}}_\text{constant w.r.t. $\theta$}\dd{\theta}\quad s=r^2,\dd{s}=2r\dd{r}\\
&=2\pi\int_0^{+\infty}e^{-s}\cdot\frac{1}{2}\dd{s}\\
&=2\pi\sqbrac{\frac{1}{2}e^{-s}(-1)}_0^\infty=\pi
\end{align*}
and thus
\[I=\int_{-\infty}^{+\infty}e^{-x^2}\dd{x}=\sqrt{\pi}.\]
\end{proof}

From this, we have
\[\Gamma\brac{\frac{1}{2}}=2\int_0^\infty e^{-t^2}\dd{t=\sqrt{\pi}.}\]

\begin{lemma}
\[\Gamma(x)=\frac{2^{x-1}}{\sqrt{\pi}}\Gamma\brac{\frac{x}{2}}\Gamma\brac{\frac{x+1}{2}}.\]
\end{lemma}

\begin{proof}
Let $\displaystyle f(x)=\frac{2^{x-1}}{\sqrt{\pi}}\Gamma\brac{\frac{x}{2}}\Gamma\brac{\frac{x+1}{2}}$. We want to prove that $f(x)=\Gamma(x)$.
\begin{enumerate}[label=(\arabic*)]
\item \begin{align*}
f(x+1)&=\frac{2^x}{\sqrt{\pi}}\Gamma\brac{\frac{x+1}{2}}\Gamma\brac{\frac{x}{2}+1}\\
&=\frac{2^x}{\sqrt{\pi}}\Gamma\brac{\frac{x+1}{2}}\frac{x}{2}\Gamma\brac{\frac{x}{2}}\\
&=xf(x)
\end{align*}
\item $f(1)=\frac{1}{\sqrt{\pi}}\Gamma\brac{\frac{1}{2}}\Gamma(1)=1$ since $\Gamma\brac{\frac{1}{2}}=\sqrt{\pi}$.
\item \[\log f(x)=\underbrace{(x-1)\log2}_\text{linear}+\underbrace{\log\Gamma\brac{\frac{x}{2}}}_\text{convex}+\underbrace{\log\Gamma\brac{\frac{x+1}{2}}}_\text{convex}-\underbrace{\log\sqrt{\pi}}_\text{constant}\]
and hence $\log f(x)$ is convex.
\end{enumerate}
Therefore $f(x)=\Gamma(x)$.
\end{proof}

\begin{theorem}[Stirling's formula]
This provides a simple approximate expression for $\Gamma(x+1)$ when $x$ is large (hence for $n!$ when $n$ is large). The formula is
\begin{equation}
\lim_{x\to\infty}\frac{\Gamma(x+1)}{(x/e)^x\sqrt{2\pi x}}=1.
\end{equation}
\end{theorem}

\begin{proof}

\end{proof}

\begin{lemma}
\[B(p,1-p)=\Gamma(p)\Gamma(1-p)=\frac{\pi}{\sin p\pi}.\]
\end{lemma}

\begin{proof}

\end{proof}