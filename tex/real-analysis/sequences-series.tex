\chapter{Numerical Sequences and Series}\label{chap:num-seq-series}
\section{Convergent Sequences}
\begin{definition}[Convergence]
Suppose that $(x_n)_{n=1}^\infty$ is a sequence of elements of a metric space $(X,d)$. Let $x\in X$. Then we say that $x_n\to x$, or that $\displaystyle\lim_{n\to\infty}x_n=x$, if
\[\forall\epsilon>0,\exists N\in\NN,\forall n\ge N,d(x_n,x)<\epsilon.\]
We call $x$ the \vocab{limit} of $(x_n)$.
\end{definition}

If $(x_n)$ does not converge, it is said to \vocab{diverge}.

\begin{exercise}
Show that $\dfrac{1}{n}\to 0$ as $n\to\infty$.
\end{exercise}

\begin{solution}
$\forall\epsilon>0$, pick $N=\frac{1}{\epsilon}+1$. Then $\forall n>N$,
\[ \frac{1}{n}<\frac{1}{N}=\frac{1}{\frac{1}{\epsilon}+1}<\frac{1}{\frac{1}{\epsilon}}=\epsilon. \]
\end{solution}

\begin{exercise}
Let $(x_n)$ be a sequence in metric space $X$, and let $x\in X$. Define what it means for $(x_n)$ to not converge to $x$.
\end{exercise}

\begin{solution}
Basically negate the definition for convergence:
\[\exists\epsilon>0\suchthat\forall N\in\NN,\exists n\ge N\suchthat d(x_n,x)\ge\epsilon.\]
\end{solution}

We now outline some important properties of convergent sequences in metric spaces.

\begin{proposition}
Let $(x_n)$ be a sequence in metric space $X$.
\begin{enumerate}[label=(\arabic*)]
\item $(x_n)$ converges to $x\in X$ if and only every neighbourhood of $x$ contains $x_n$ for all but finitely many $n$.
\item (uniqueness of the limit) If $x\in X$, $x^\prime\in X$, and if $(x_n)$ converges to $x$ and to $x^\prime$, then $x^\prime=x$.
\item (boundedness of convergent sequences) If $(x_n)$ converges, then $(x_n)$ is bounded.
\item For $E\subset X$, $x$ is a limit point of $E$, if and only if there exists a sequence $(x_n)$ in $E\setminus\{x\}$ such that $x_n\to x$.
\end{enumerate}
\end{proposition}

\begin{proof} \
\begin{enumerate}[label=(\arabic*)]
\item ($\implies$) Suppose $x_n\to x$. We want to prove that any neighbourhood $U$ of $x$ eventually contains all $x_n$.

Since $U$ is a neighbourhood of $x$, pick a ball $B_\epsilon(x)\subset U$. Corresponding to this $\epsilon$, there exists $N\in\NN$ such that $n\ge N$ implies $d(x_n,x)<\epsilon$. Thus $n\ge N$ implies $x_n\in U$.

($\impliedby$) Suppose every neighbourhood of $x$ contains all but finitely many of the $x_n$. Fix $\epsilon>0$, pick a ball $B_\epsilon(x)$. Since $B_\epsilon(x)$ is a neighbourhood of $x$, it will also eventually contain all $x_n$. By assumption, there eists $N\in\NN$ such that $x_n\in B_\epsilon(x)$ if $n\ge N$. Thus $d(x_n,x)<\epsilon$ if $n\ge N$, hence $x_n\to x$.

\item Let $\epsilon>0$ be given. There exists $N,N^\prime\in\NN$ such that
\[n\ge N\implies d(x_n,x)<\frac{\epsilon}{2}\]
and
\[n\ge N^\prime\implies d(x_n,x^\prime)<\frac{\epsilon}{2}.\]
Take $N_1\coloneqq\max\{N,N^\prime\}$. Hence if $n\ge N_1$ we have $d(x_n,x)<\frac{\epsilon}{2}$ and $d(x_n,x^\prime)<\frac{\epsilon}{2}$ at the same time. By triangle inequality,
\[ d(x,x^\prime)\le d(x,x_n)+d(x_n,x^\prime)<\frac{\epsilon}{2}+\frac{\epsilon}{2}=\epsilon.\]
Since $\epsilon$ was arbitrary (i.e. holds for all $\epsilon>0$), we must have $d(x,x^\prime)=0$ and thus $x=x^\prime$.

\item Suppose $x_n\to x$. Then there exists $N\in\NN$ such that $n>N$ implies $d(x_n,x)<1$. Take
\[r\coloneqq\max\{1,d(x_1,x),\dots,d(x_N,x)\}.\]
Then $d(x_n,x)\le r$ for $n=1,2,\dots,N$, so $(x_n)$ is in $B_r(x)$.

\item ($\implies$) If $x$ is a limit point, then for all $\epsilon>0$, $B_\epsilon\setminus\{x\}(x)$ contains points in $E$. We then construct such a sequence $(x_n)$ in $E\setminus\{x\}$: pick any $x_n\in E$ so that $x_n$ is contained in $B_\frac{1}{n}\setminus\{x\}(x)$. Then it is easy to show that $(x_n)$ is a sequence in $E\setminus\{x\}$ which converges to $x$.

($\impliedby$) Suppose that there exists a sequence $(x_n)$ in $E\setminus\{x\}$ such that $x_n\to x$. We wish to show that $B_\epsilon\setminus\{x\}(x)$ contains points in $E$ for all $\epsilon>0$.

Since $(x_n)$ converges to $x$, for all $\epsilon>0$ the sequence is eventually contained in $B_\epsilon(x)$. However because we have the precondition that $(x_n)$ has to be in $E\setminus\{x\}$, the sequence is in fact eventually contained in $B_\epsilon\setminus\{x\}(x)$.
\end{enumerate}
\end{proof}

\begin{lemma}
If $(a_n)$ and $(b_n)$ are two convergent sequences, and $a_n \le b_n$, then $\displaystyle\lim_{n\to\infty}a_n\le\lim_{n\to\infty}b_n$.
\end{lemma}

\begin{remark}
Even if you have $a_n<b_n$, you cannot say that $\displaystyle\lim_{n\to\infty}a_n<\lim_{n\to\infty}b_n$. For example, $-\frac{1}{n}<\frac{1}{n}$ but their limits are both $0$.
\end{remark}

\begin{proof}
Let $\displaystyle A=\lim_{n\to\infty}a_n$, $\displaystyle B=\lim_{n\to\infty}b_n$. Suppose otherwise that $A>B$, take $\epsilon=A-B>0$.

Since $\frac{\epsilon}{2}>0$, then there exists $N_1$ such that for $n>N_1$ we have $|a_n-A|<\frac{\epsilon}{2}$; and there exists $N_2$ such that for $n>N_2$ we have $|b_n-B|<\frac{\epsilon}{2}$.

Let $N=\max\{N_1,N_2\}$, then for any $n>N$, the two inequalities above will hold simultaneously. But then we would have
\[a_n>A-\frac{\epsilon}{2},\quad b_n<B+\frac{\epsilon}{2}\]
and thus
\[a_n-b_n>A-B-\epsilon=0\]
so $a_n>b_n$, a contradiction.
\end{proof}

\begin{theorem}[Sandwich Theorem]
Let $a_n\le c_n\le b_n$ where $(a_n),(b_n)$ are converging sequences such that $\displaystyle\lim_{n\to\infty}a_n=\lim_{n\to\infty}b_n=L$, then $(c_n)$ is also a converging sequence and $\displaystyle\lim_{n\to\infty}c_n=L$.
\end{theorem}

\begin{proof}

\end{proof}

\begin{lemma}[Arithmetic properties]
Suppose $(a_n)$ and $(b_n)$ are convergent seqeunces of real numbers, $k\in\RR$. Then
\begin{enumerate}[label=(\arabic*)]
\item Scalar multiplication: $\displaystyle\lim_{n\to\infty} ka_n=k\lim_{n\to\infty}a_n$
\item Addition: $\displaystyle\lim_{n\to\infty}(a_n+b_n)=\lim_{n\to\infty}a_n+\lim_{n\to\infty}b_n$
\item Multiplication: $\displaystyle\lim_{n\to\infty}(a_n b_n)=\lim_{n\to\infty}a_n\cdot\lim_{n\to\infty}b_n$
\item Division: $\displaystyle\lim_{n\to\infty}\frac{a_n}{b_n}=\frac{\lim_{n\to\infty} a_n}{\lim_{n\to\infty} b_n}$ ($b_n\neq0$, $\displaystyle\lim_{n\to\infty}b_n\neq0$)
\end{enumerate}
\end{lemma}

\begin{proof} \
\begin{enumerate}[label=(\arabic*)]
\item The proof is left as an exercise. You will need to consider three cases, when $k$ is positive, negative or $0$.

\item Let $\displaystyle A=\lim_{n\to\infty}a_n$, $\displaystyle B=\lim_{n\to\infty}b_n$.

$\forall\epsilon>0$, $\exists N_1\in\NN$, $\forall n>N_1$
\[ |a_n-A|<\frac{\epsilon}{2}. \]
$\forall\epsilon>0$, $\exists N_2\in\NN$, $\forall n>N_2$, 
\[ |b_n-B|<\frac{\epsilon}{2}. \]

Let $N=\max\{N_1,N_2\}$, then for all $n>N$, by the triangle inequality we have
\[ \absolute{(a_n+b_n)-(A+B)}\le|a_n-A|+|b_n-B|<\epsilon. \]

\item Let $\displaystyle A=\lim_{n\to\infty}a_n$, $\displaystyle B=\lim_{n\to\infty}b_n$.

Consider the limit $\displaystyle\lim_{n\to\infty}(a_nb_n-AB)$. We want to prove that this equals to $0$. We write
\[ \lim_{n\to\infty}(a_nb_n-AB)=\lim_{n\to\infty}(a_nb_n-Ab_n+Ab_n-AB) \]
The idea is to show that this is equal to
\[ \lim_{n\to\infty}(a_nb_n-Ab_n)+\lim_{n\to\infty}(Ab_n-AB) \]
(Note that we cannot write this yet because we have not shown that these two sequences are convergent)

So let's examine these two sequences. The second one is easier since we have proved addition:
\[ \lim_{n\to\infty} b_n=B \implies \lim_{n\to\infty}(b_n-B)=0 \]
Thus $\displaystyle\lim_{n\to\infty}(Ab_n-AB)=A\lim_{n\to\infty}(b_n-B)=0$.

As for the first one, we want to show that $\displaystyle\lim_{n\to\infty}(a_n-A)b_n=0$. Since $b_n$ is convergent, $b_n$ is bounded. Let $M>0$ be a bound of $b_n$: $\forall n\in\NN$,
\[|b_n|\le M.\]

Since $\displaystyle\lim_{n\to\infty} a_n=a$, $\forall\epsilon>0\exists N\in\NN\suchthat\forall n>N$,
\[|a_n-a|<\frac{\epsilon}{M}.\]

Combining the two above, we then conclude that $\forall\epsilon>0\exists N\in\NN\suchthat\forall n>N$,
\[ |a_nb_n-Ab_n|=|(a_n-A)b_n|<\frac{\epsilon}{M}\cdot M=\epsilon. \]
Thus $\displaystyle\lim_{n\to\infty}(a_nb_n-Ab_n)=0$.

Since $\displaystyle\lim_{n\to\infty}(Ab_n-AB)=0$ and $\displaystyle\lim_{n\to\infty}(a_nb_n-Ab_n)=0$, we can conclude that $\displaystyle\lim_{n\to\infty}(a_nb_n-AB)=0$.

\item Since we have proven multiplication, it suffices to show that $\displaystyle\lim_{n\to\infty}\frac{1}{b_n}=\frac{1}{\lim_{n\to\infty} b_n}$.

Let $\displaystyle b=\lim_{n\to\infty} b_n$. Consider the limit
\[ \lim_{n\to\infty}\brac{\frac{1}{b_n}-\frac{1}{b}}=\lim_{n\to\infty}\brac{\frac{b-b_n}{b_nb}}. \]

Again, the important term here is $b-b_n$, but there is an extra term of $\frac{1}{b_nb}$, so we'll need to control this.

Since we need this to be bounded, we actually cannot have $b_n$ to be close to $0$. The good thing here is that $b\neq0$, so we can restrict $b_n$ to be close enough to $b$ so that it stays away from $0$.

Pick $N_1$ such that for all $n>N_1$,
\[|b_n-b|<\frac{|b|}{2}.\]

Then
\begin{align*}
|b_nb-b^2|&<\frac{b^2}{2}\\
\frac{b^2}{2}<b_nb<\frac{3b^2}{2}
\end{align*}

This show that if $n>N_1$, $b_nb$ would always be positive, and $\frac{1}{b_nb}<\frac{2}{b^2}$.

Let $M=\frac{2}{b^2}$, then we may refer back to the original statement
\[ \absolute{\frac{b-b_n}{b_nb}}<M|b-b_n| \]
We pick $N_2$ such that for all $n>N_2$, $|b_n-b|<\frac{\epsilon}{M}$.

Let $N=\max\{N_1,N_2\}$, then for all $n>N$,
\[ \absolute{\frac{b-b_n}{b_nb}}<M\cdot\frac{\epsilon}{M}=\epsilon. \]
\end{enumerate}
\end{proof}

\begin{exercise}
Let $(x_n)$ be a sequence of real numbers and let $\alpha\ge2$ be a constant. Define the sequence $(y_n)$ as follows:
\[y_n=x_n+\alpha x_{n+1}\quad(n=1,2,\dots)\]
Show that if $(y_n)$ is convergent, then $(x_n)$ is also convergent.
\end{exercise}

\begin{exercise} \
\begin{enumerate}[label=(\arabic*)]
\item $\displaystyle\lim_{n\to\infty}\frac{1}{n_p}=0$ ($p>0$).
\item $\displaystyle\lim_{n\to\infty}\sqrt[n]{p}=1$ ($p>0$).
\item $\displaystyle\lim_{n\to\infty}\sqrt[n]{n}=1$.
\item $\displaystyle\lim_{n\to\infty}\frac{n^\alpha}{(1+p)^n}=0$ ($p>0$, $\alpha\in\RR$).
\item $\displaystyle\lim_{n\to\infty}x^n=0$ ($|x|<1$).
\end{enumerate}
\end{exercise}

\section{Subsequences}
\begin{definition}[Subsequence]
Given a sequence $(x_n)$, consider a sequence $(n_k)$ of positive integers such that $n_1<n_2<\cdots$. Then $(x_{n_i})$ is called a \vocab{subsequence} of $(x_n)$. If $(x_{n_i})$ converges, its limit is called a \vocab{subsequential limit} of $(x_n)$.
\end{definition}

\begin{proposition}
$(x_n)$ converges to $x$ if and only if every subsequence of $(x_n)$ converges to $x$.
\end{proposition}

\begin{proof}
Suppose $(x_n)$ converges to $x$. Then $\forall\epsilon>0$, $\exists N\in\NN$, $\forall n>N$, $d(x_n,x)<\epsilon$. Every subsequence of $(x_n)$ can be written in the form $(x_{n_i})$ where $n_1<n_2<\cdots$ is a strictly increasing sequence of positive integers. Pick $M$ such that $n_M>N$, then $\forall i>M$, $d(x_{n_i},x)<\epsilon$. Hence every subsequence of $(x_n)$ converges to $x$.

Intuitively, if every neighbourhood of $x$ eventually contains all $x_n$, then since $(x_{n_i})$ is a subset of $(x_n)$ they should all be contained in the neighbourhood eventually as well.
\end{proof}

\begin{proposition}
Subsequential limits of a sequence are precisely the limit points of the sequence (viewed as a set)
\end{proposition}

\begin{proof}
This is just part (d) of the previous section.

Again, to make this work, we need to assume that nothing funny is going on at subsequential limits
If the limits appear due to eventually constant subsequences, then they need not be limit points of the original sequence when viewed as a set

3.6, 3.7 are precisely the statements we've prepared for last week
\end{proof}

\begin{proposition}
If $(x_n)$ is a sequence in a compact set (bounded closed set), then there exists a convergent subsequence of $(x_n)$ (which converges to some number in the set).
\end{proposition}

\begin{proof}
This is Bolzano--Weierstrass together with part (b)

Essentially, compact sets satisfies the property akin to the statement in Heine-Borel:
Given a topological space $(X,\tau)$, a compact set $K$ in $X$ is a set satisfying that, given any open covering $\{U_i\}$ of $X$, there exists a finite open cover $\{U_1,\dots,U_n\}$ of $X$

This is difficult to process at this stage
Since we're currently only working with Euclidean spaces it would be more beneficial if you consider the Heine-Borel Theorem as a property first
It would be a lot easier to accept the definition after you're more accustomed to applying the theorem
\end{proof}

\begin{proposition}
The subsequential limits of a sequence $(x_n)$ in metric space $X$ form a closed subset of $X$.
\end{proposition}

\begin{proof}
Let $E$ be the set of all subsequential limits of $(x_n)$, and let $q$ be a limit point of $E$. We want to show that $q\in E$.

Choose $n_1$ so that $x_{n_1}\neq q$. (If no such $n_1$ exists, then $E$ has only one point, and there is nothing to prove.) Put $\delta=d(q,x_{n_1})$. Suppose $n_1,\dots,n_{i-1}$ are chosen. Since $q$ is a limit point of $E$, there is an $x\in E$ with $d(x,q)<2^{-1}\delta$. Since $x\in E$, there is an $n_i>n_{i-1}$ such that $d(x,x_{n_i})<2^{-i}\delta$. Thus
\[d(q,x_{n_i})<2^{1-i}\delta\]
for $i=1,2,3,\dots$. This says that $(x_{n_i})$ converges to $q$. Hence $q\in E$.
\end{proof}

\section{Cauchy Sequences}
This is a very helpful way to determine whether a sequence is convergent or divergent, as it does not require the limit to be known. In the future you will see many instances where the convergence of all sorts of limits are compared with similar counterparts; generally we describe such properties as \vocab{Cauchy criteria}.

\begin{definition}[Cauchy sequence]
A sequence $(x_n)$ in a metric space $X$ is said to be a \vocab{Cauchy sequence} if 
\[\forall\epsilon>0,\exists N\in\NN,\forall n,m\ge N,d(x_n,x_m)<\epsilon.\]
\end{definition}

\begin{remark}
This simply means that the distances between any two terms is sufficiently small after a certain point.
\end{remark}

It is easy to prove that a converging sequence is Cauchy using the triangle inequality. The idea is that, if all the points are becoming arbitrarily close to a given point $x$, then they are also becoming close to each other. The converse is not always true, however.

\begin{proposition}
A sequence $(x_n)$ in $\RR^n$ is convergent if and only if it is Cauchy.
\end{proposition}

\begin{proof} \

($\implies$) Suppose that $(x_n)$ converges to $x$, then there exists $N\in\NN$ such that $\forall n>N$, $|x_n-x|<\dfrac{\epsilon}{2}$. Then for $n,m>N$, by triangle inequality,
\[|x_n-x_m|\le|x_n-x|+|x_m-x|<\frac{\epsilon}{2}+\frac{\epsilon}{2}=\epsilon.\]
Hence $(x_n)$ is a Cauchy sequence.

($\impliedby$) First, we show that $(x_n)$ must be bounded. 
Pick $N\in\NN$ such that $\forall n,m>N$ we have $|x_n-x_m|<1$. 
Centered at $x_n$, we show that $(x_n)$ is bounded; to do this we pick
\[r=\max\{1,|x_n-x_1|,\dots,|x_n-x_N|\}.\]
Then the sequence ${x_n}$ is in $B_r(x_n)$ and thus is bounded.

Since $(x_n)$ is bounded, by the corollary of Bolzano--Weierstrass we know that $(x_n)$ contains a subsequence $(x_{n_i})$ that converges to $x$.

Then $\forall\epsilon>0$, pick $N_1\in\NN$ such that for all $n,m>N$, $|x_n-x_m|<\dfrac{\epsilon}{2}$.

Simultaneously, since $\{x_{n_i}\}$ converges to $x$, pick $M$ such that for $i>M$, $|x_{n_i}-x|<\dfrac{\epsilon}{2}$.

Now, since $n_1<n_2<\cdots$ is a sequence of strictly increasing natural numbers, we can pick $i>M$ such that $n_i>N$. Then $\forall n>N$, by setting $m=n_i$ we obtain
\[ |x_n-x_{n_i}| < \frac{\epsilon}{2}, \quad |x_{n_i}-x| < \frac{\epsilon}{2} \]
and hence
\[|x_n-x|\le|x_n-x_{n_i}|+|x_{n_i}-x|<\epsilon\]
by triangle inequality. Hence $(x_n)$ is convergent.
\end{proof}

\begin{definition}
Let nonempty $E\subseteq X$. Let $S$ be the set of all real numbers of the form $d(x,y)$, with $x,y\in E$. Then the \vocab{diameter} of $E$ is 
\[ \diam E\coloneqq\sup S. \]

\end{definition}

\begin{definition}
A sequence $(x_n)$ of real number is said to be
\begin{enumerate}[label=(\roman*)]
\item \vocab{monotonically increasing} if $x_n\le x_{n+1}$ ($n=1,2,\dots$);
\item \vocab{monotonically decreasing} if $x_n\ge x_{n+1}$ ($n=1,2,\dots$).
\end{enumerate}
The class of monotonic sequences consists of the increasing and decreasing sequences.
\end{definition}

\begin{lemma}
Suppose $(x_n)$ is monotonic. Then $(x_n)$ converges if and only if it is bounded.
\end{lemma}

\begin{proof}
Suppose $x_n\le x_{n+1}$ (the proof is analogous in the other case). Let $E$ be the range of $(x_n)$. Suppose $(x_n)$ is bounded, then let $x=\sup E$. Then
\[x_n\le x\quad(n=1,2,\dots)\]
...
\end{proof}

\section{Upper and Lower Limits}
\begin{definition}
Let $(x_n)$ be a real sequence with the following property: $\forall M\in\RR\exists N\in\NN\suchthat n\ge N\implies x_n\ge M$. We then write $x_n\to\infty$.

Similarly, if $\forall M\in\RR\exists N\in\NN\suchthat n\ge N\implies x_n\le M$, we write $x_n\to\infty$.
\end{definition}

\begin{definition}[Upper and lower limits]\label{defn:upper-lower-limit}
Let $(x_n)$ be a real sequence. Let $E$ be the set of real numbers $x$ (in the extended real number system) such that $x_{n_k}\to x$ for some subsequence $(x_{n_k})$. This set $E$ contains all subsequential limits, plus possibly the numbers $+\infty$ and $-\infty$.

Put $x^*=\sup E$, $x_*=\inf E$. The numbers $x^*$ and $x_*$ are called the \vocab{upper limit} and \vocab{lower limit} of $(x_n)$, denoted by
\[\limsup_{n\to\infty}x_n=x^*,\quad\liminf_{n\to\infty}x_n=x_*.\]
\end{definition}

\begin{proposition}
Let $(x_n)$ be a real sequence. Let $E$ and $x^*$ have the same meaning as in \cref{defn:upper-lower-limit}. Then $x^*$ has the following two properties:
\begin{enumerate}[label=(\arabic*)]
\item $x^*\in E$.
\item If $x>x^*$, there exists $N\in\NN$ such that $n\ge N$ implies $x_n<x$.
\end{enumerate}
Moreover, $x^*$ is the only number with the properties (1) and (2).
\end{proposition}

\begin{example} \
\begin{itemize}
\item Let $(x_n)$ be a sequence containing all rationals. Then every real number is a subsequential limit, and $\limsup_{n\to\infty}x_n=+\infty$, $\liminf_{n\to\infty}=-\infty$.
\item For a real-valued seqeunce $(x_n)$, $\lim_{n\to\infty}x_n=x$ if and only if $\limsup_{n\to\infty}x_n=\liminf_{n\to\infty}x_n=x$.
\end{itemize}
\end{example}

\begin{lemma}
If $a_n\le b_n$ for $n\ge N$ where $N$ is fixed, then
\begin{align*}
\liminf_{n\to\infty}a_n&\le\liminf_{n\to\infty}b_n,\\
\limsup_{n\to\infty}a_n&\le\limsup_{n\to\infty}b_n.
\end{align*}
\end{lemma}

\begin{lemma}[Arithmetic properties] \
\begin{enumerate}[label=(\arabic*)]
\item If $k>0$, $\limsup_{n\to\infty}ka_n=k\limsup_{n\to\infty}a_n$.

If $k<0$, $\limsup_{n\to\infty}ka_n=k\liminf_{n\to\infty}a_n$.

\item $\limsup(a_n+b_n)\le\limsup a_n+\limsup b_n$

Moreover, $\limsup(a_n+b_n)$ may be bounded from below as follows:
\[ \limsup(a_n+b_n)\ge\limsup a_n+\liminf b_n \]

Your homework for today is to write down the analogous properties for liminf, and to prove (i) and (ii)
\end{enumerate}
\end{lemma}

Now you should try to prove (i) for liminf as well; as for (ii), try to explain why properties (i),(ii) for limsup and property (i) for liminf would imply property (ii) for $\liminf$

\section{Series}
\begin{definition}[Series]
Given a sequence $(a_n)$ in metric space $X$, we associate a sequence $(S_n)$, where
\[S_n=\sum_{k=1}^n a_k\]
which we call a \vocab{series}. $S_n$ is the \vocab{$n$-th partial sum} of the series.

We say that the (infinite) series \vocab{converges} if the sequence of partial sums $(S_n)$ converges. We then define the \vocab{sum} of a convergent infinite series to be the limit of the convergent sequence $(S_n)$; that is, given $S\in X$, $\sum_{n=1}^\infty a_n=S$ if
\[\forall\epsilon>0,\exists N\in\NN,\forall n>N,\absolute{\sum_{k=1}^n a_k-S}<\epsilon.\]

If $(S_n)$ diverges, the series is said to \vocab{diverge}.
\end{definition}

3.22
\begin{proposition}[Cauchy criterion]
$\sum_{n=1}^\infty a_n$ converges if and only if $\forall\epsilon>0$, $\exists N\in\NN$ such that $\forall m\ge n>N$,
\[\absolute{\sum_{k=m}^n a_k}<\epsilon.\]
\end{proposition}

\begin{corollary}
If $\sum_{n=1}^\infty a_n$ converges, then $\lim_{n\to\infty}a_n=0$.
\end{corollary}

\begin{remark}
The converse is not true; we have the very well known counterexample of the harmonic series $\sum_{n=1}^\infty\frac{1}{n}$.
\end{remark}

Now we talk about various methods to determine whether an infinite series converges or diverges.

\begin{lemma}[Comparison test]
We consider two sequences $(a_n)$ and $(b_n)$.
\begin{enumerate}[label=(\arabic*)]
\item Suppose $|a_n|\le b_n$ for all $n$ (or for all sufficiently large $n$), if $\sum_{n=1}^\infty b_n$ converges, then $\sum_{n=1}^\infty a_n$ converges.
\item Suppose $a_n\ge b_n\ge0$ for all $n$ (or for all sufficiently large $n$), if $\sum_{n=1}^\infty b_n$ diverges, then $\sum_{n=1}^\infty a_n$ diverges.
\end{enumerate}
\end{lemma}
Refer to 3.25 for the proof

\begin{lemma}[Root test]
Given $\sum a_n$, put $\displaystyle\alpha=\limsup_{n\to\infty}\sqrt[n]{|a_n|}$. Then
\begin{enumerate}[label=(\roman*)]
\item if $\alpha<1$, $\sum a_n$ converges;
\item if $\alpha>1$, $\sum a_n$ diverges;
\item if $\alpha=1$, the rest gives no information.
\end{enumerate}
\end{lemma}

\begin{lemma}[Ratio test]
The series $\sum a_n$
\begin{enumerate}[label=(\roman*)]
\item converges if $\displaystyle\limsup_{n\to\infty}\absolute{\frac{a_{n+1}}{a_n}}<1$;
\item diverges if $\displaystyle\absolute{\frac{a_{n+1}}{a_n}}\ge1$ for all $n\ge n_0$, where $n_0$ is some fixed integer.
\end{enumerate}
\end{lemma}

The series $\sum a_n$ is said to \vocab{converge absolutely} if the series $\sum|a_n|$ converges.

\begin{lemma}
If $\sum a_n$ converges absolutely, then $\sum a_n$ converges.
\end{lemma}

\begin{definition}
\[e\coloneqq\sum_{n=0}^\infty\frac{1}{n!}\]
\end{definition}

\begin{lemma}
\[\lim_{n\to\infty}\brac{1+\frac{1}{n}}^n=e.\]
\end{lemma}

\begin{proof}
Let
\[s_n=\sum_{k=0}^n\frac{1}{k!},\quad t_n=\brac{1+\frac{1}{n}}^n.\]
Expanding $t_n$ using the binomial theorem, and comparing $s_n$ and $t_n$ term by term, we have $t_n\le s_n$, so
\[\limsup_{n\to\infty}t_n\le e.\]

\end{proof}

\begin{proposition}
$e$ is irrational.
\end{proposition}

\begin{proof}
Suppose $e$ is rational. Then $e=\frac{p}{q}$, where 
\end{proof}